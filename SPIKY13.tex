



<!DOCTYPE html>
<html lang="en" class="   ">
  <head prefix="og: http://ogp.me/ns# fb: http://ogp.me/ns/fb# object: http://ogp.me/ns/object# article: http://ogp.me/ns/article# profile: http://ogp.me/ns/profile#">
    <meta charset='utf-8'>
    <meta http-equiv="X-UA-Compatible" content="IE=edge">
    <meta http-equiv="Content-Language" content="en">
    
    
    <title>SPIKY-paper/SPIKY13.tex at b26e086bba9f6529f3199194b9f241500d468bb8 · thomaskreuz/SPIKY-paper</title>
    <link rel="search" type="application/opensearchdescription+xml" href="/opensearch.xml" title="GitHub">
    <link rel="fluid-icon" href="https://github.com/fluidicon.png" title="GitHub">
    <link rel="apple-touch-icon" sizes="57x57" href="/apple-touch-icon-114.png">
    <link rel="apple-touch-icon" sizes="114x114" href="/apple-touch-icon-114.png">
    <link rel="apple-touch-icon" sizes="72x72" href="/apple-touch-icon-144.png">
    <link rel="apple-touch-icon" sizes="144x144" href="/apple-touch-icon-144.png">
    <meta property="fb:app_id" content="1401488693436528">

      <meta content="@github" name="twitter:site" /><meta content="summary" name="twitter:card" /><meta content="thomaskreuz/SPIKY-paper" name="twitter:title" /><meta content="Contribute to SPIKY-paper development by creating an account on GitHub." name="twitter:description" /><meta content="https://avatars1.githubusercontent.com/u/7440733?v=2&amp;s=400" name="twitter:image:src" />
<meta content="GitHub" property="og:site_name" /><meta content="object" property="og:type" /><meta content="https://avatars1.githubusercontent.com/u/7440733?v=2&amp;s=400" property="og:image" /><meta content="thomaskreuz/SPIKY-paper" property="og:title" /><meta content="https://github.com/thomaskreuz/SPIKY-paper" property="og:url" /><meta content="Contribute to SPIKY-paper development by creating an account on GitHub." property="og:description" />

    <link rel="assets" href="https://assets-cdn.github.com/">
    <link rel="conduit-xhr" href="https://ghconduit.com:25035">
    <link rel="xhr-socket" href="/_sockets">

    <meta name="msapplication-TileImage" content="/windows-tile.png">
    <meta name="msapplication-TileColor" content="#ffffff">
    <meta name="selected-link" value="repo_source" data-pjax-transient>
      <meta name="google-analytics" content="UA-3769691-2">

    <meta content="collector.githubapp.com" name="octolytics-host" /><meta content="collector-cdn.github.com" name="octolytics-script-host" /><meta content="github" name="octolytics-app-id" /><meta content="958B0F41:4A6A:B3FD562:53FB6687" name="octolytics-dimension-request_id" /><meta content="7440733" name="octolytics-actor-id" /><meta content="thomaskreuz" name="octolytics-actor-login" /><meta content="5f84cb923c1e84b94fdbf424050c5b87af62119a2c0392ea67f450d0638cd669" name="octolytics-actor-hash" />
    

    
    
    <link rel="icon" type="image/x-icon" href="https://assets-cdn.github.com/favicon.ico">


    <meta content="authenticity_token" name="csrf-param" />
<meta content="mQreK6mksjC2puqs7J9gEfbXP19No7SB/c+Q0UmoXcbmBw2UwHWvD2rEpsVL428ZlShI15VVHgq2GspQsOdF3w==" name="csrf-token" />

    <link href="https://assets-cdn.github.com/assets/github-c42e52ca7947fa1d1ee79edddeebf592213d7224.css" media="all" rel="stylesheet" type="text/css" />
    <link href="https://assets-cdn.github.com/assets/github2-690d1864b28aa6d2e824fea6953c01baa82cc813.css" media="all" rel="stylesheet" type="text/css" />
    


    <meta http-equiv="x-pjax-version" content="0de96c373fd14f20605f14146b163b7e">

      
  <meta name="description" content="Contribute to SPIKY-paper development by creating an account on GitHub.">
  <meta name="go-import" content="github.com/thomaskreuz/SPIKY-paper git https://github.com/thomaskreuz/SPIKY-paper.git">

  <meta content="7440733" name="octolytics-dimension-user_id" /><meta content="thomaskreuz" name="octolytics-dimension-user_login" /><meta content="19281968" name="octolytics-dimension-repository_id" /><meta content="thomaskreuz/SPIKY-paper" name="octolytics-dimension-repository_nwo" /><meta content="true" name="octolytics-dimension-repository_public" /><meta content="false" name="octolytics-dimension-repository_is_fork" /><meta content="19281968" name="octolytics-dimension-repository_network_root_id" /><meta content="thomaskreuz/SPIKY-paper" name="octolytics-dimension-repository_network_root_nwo" />
  <link href="https://github.com/thomaskreuz/SPIKY-paper/commits/b26e086bba9f6529f3199194b9f241500d468bb8.atom" rel="alternate" title="Recent Commits to SPIKY-paper:b26e086bba9f6529f3199194b9f241500d468bb8" type="application/atom+xml">

  </head>


  <body class="logged_in  env-production macintosh vis-public page-blob">
    <a href="#start-of-content" tabindex="1" class="accessibility-aid js-skip-to-content">Skip to content</a>
    <div class="wrapper">
      
      
      
      


      <div class="header header-logged-in true">
  <div class="container clearfix">

    <a class="header-logo-invertocat" href="https://github.com/" aria-label="Homepage" ga-data-click="Header, go to dashboard, icon:logo">
  <span class="mega-octicon octicon-mark-github"></span>
</a>


      <div class="site-search repo-scope js-site-search">
          <form accept-charset="UTF-8" action="/thomaskreuz/SPIKY-paper/search" class="js-site-search-form" data-global-search-url="/search" data-repo-search-url="/thomaskreuz/SPIKY-paper/search" method="get"><div style="margin:0;padding:0;display:inline"><input name="utf8" type="hidden" value="&#x2713;" /></div>
  <input type="text"
    class="js-site-search-field is-clearable"
    data-hotkey="s"
    name="q"
    placeholder="Search"
    data-global-scope-placeholder="Search GitHub"
    data-repo-scope-placeholder="Search"
    tabindex="1"
    autocapitalize="off">
  <div class="scope-badge">This repository</div>
</form>
      </div>
      <ul class="header-nav left">
        <li class="header-nav-item explore" data-ga-click="Header, go to explore, text:explore">
          <a class="header-nav-link" href="/explore">Explore</a>
        </li>
          <li class="header-nav-item">
            <a class="header-nav-link" href="https://gist.github.com" data-ga-click="Header, go to gist, text:gist">Gist</a>
          </li>
          <li class="header-nav-item">
            <a class="header-nav-link" href="/blog" data-ga-click="Header, go to blog, text:blog">Blog</a>
          </li>
        <li class="header-nav-item">
          <a class="header-nav-link" href="https://help.github.com" data-ga-click="Header, go to help, text:help">Help</a>
        </li>
      </ul>

    
<ul class="header-nav user-nav right" id="user-links">
  <li class="header-nav-item dropdown js-menu-container">
    <a class="header-nav-link name" href="/thomaskreuz" data-ga-click="Header, go to profile, text:username">
      <img alt="thomaskreuz" class="avatar" data-user="7440733" height="20" src="https://avatars2.githubusercontent.com/u/7440733?v=2&amp;s=40" width="20" />
      <span class="css-truncate">
        <span class="css-truncate-target">thomaskreuz</span>
      </span>
    </a>
  </li>

  <li class="header-nav-item dropdown js-menu-container">
    <a class="header-nav-link js-menu-target tooltipped tooltipped-s" href="#" aria-label="Create new..." data-ga-click="Header, create new, icon:add">
      <span class="octicon octicon-plus"></span>
      <span class="dropdown-caret"></span>
    </a>

    <div class="dropdown-menu-content js-menu-content">
      
<ul class="dropdown-menu">
  <li>
    <a href="/new"><span class="octicon octicon-repo"></span> New repository</a>
  </li>
  <li>
    <a href="/organizations/new"><span class="octicon octicon-organization"></span> New organization</a>
  </li>


    <li class="dropdown-divider"></li>
    <li class="dropdown-header">
      <span title="thomaskreuz/SPIKY-paper">This repository</span>
    </li>
      <li>
        <a href="/thomaskreuz/SPIKY-paper/issues/new"><span class="octicon octicon-issue-opened"></span> New issue</a>
      </li>
      <li>
        <a href="/thomaskreuz/SPIKY-paper/settings/collaboration"><span class="octicon octicon-person"></span> New collaborator</a>
      </li>
</ul>

    </div>
  </li>

  <li class="header-nav-item">
        <a href="/notifications" aria-label="You have no unread notifications" class="header-nav-link notification-indicator tooltipped tooltipped-s" data-ga-click="Header, go to notifications, icon:read" data-hotkey="g n">
        <span class="mail-status all-read"></span>
        <span class="octicon octicon-inbox"></span>
</a>
  </li>

  <li class="header-nav-item">
    <a class="header-nav-link tooltipped tooltipped-s" href="/settings/profile" id="account_settings" aria-label="Settings" data-ga-click="Header, go to settings, icon:settings">
      <span class="octicon octicon-gear"></span>
    </a>
  </li>

  <li class="header-nav-item">
    <form accept-charset="UTF-8" action="/logout" class="logout-form" method="post"><div style="margin:0;padding:0;display:inline"><input name="utf8" type="hidden" value="&#x2713;" /><input name="authenticity_token" type="hidden" value="o3qDM/kiazoiw7/bwJnAFAMUcBc3PPMj6PjKbx8gyyJTLi3+J2sYkAOtxmzEuuln76ueKViENnrfihBhYxNx9A==" /></div>
      <button class="header-nav-link sign-out-button tooltipped tooltipped-s" aria-label="Sign out" data-ga-click="Header, sign out, icon:logout">
        <span class="octicon octicon-sign-out"></span>
      </button>
</form>  </li>

</ul>


    
  </div>
</div>

      

        


      <div id="start-of-content" class="accessibility-aid"></div>
          <div class="site" itemscope itemtype="http://schema.org/WebPage">
    <div id="js-flash-container">
      
    </div>
    <div class="pagehead repohead instapaper_ignore readability-menu">
      <div class="container">
        
<ul class="pagehead-actions">

    <li class="subscription">
      <form accept-charset="UTF-8" action="/notifications/subscribe" class="js-social-container" data-autosubmit="true" data-remote="true" method="post"><div style="margin:0;padding:0;display:inline"><input name="utf8" type="hidden" value="&#x2713;" /><input name="authenticity_token" type="hidden" value="dQhmhxzLMLxEDqCYaUM6S+tCWw8uWNgyAOWptcA0ROqel1NcKakBVlThv5dyrelXJIVtpyBDiVeS12PWN+pIoA==" /></div>  <input id="repository_id" name="repository_id" type="hidden" value="19281968" />

    <div class="select-menu js-menu-container js-select-menu">
      <a class="social-count js-social-count" href="/thomaskreuz/SPIKY-paper/watchers">
        2
      </a>
      <a href="/thomaskreuz/SPIKY-paper/subscription"
        class="minibutton select-menu-button with-count js-menu-target" role="button" tabindex="0" aria-haspopup="true">
        <span class="js-select-button">
          <span class="octicon octicon-eye"></span>
          Unwatch
        </span>
      </a>

      <div class="select-menu-modal-holder">
        <div class="select-menu-modal subscription-menu-modal js-menu-content" aria-hidden="true">
          <div class="select-menu-header">
            <span class="select-menu-title">Notifications</span>
            <span class="octicon octicon-x js-menu-close" role="button" aria-label="Close"></span>
          </div> <!-- /.select-menu-header -->

          <div class="select-menu-list js-navigation-container" role="menu">

            <div class="select-menu-item js-navigation-item " role="menuitem" tabindex="0">
              <span class="select-menu-item-icon octicon octicon-check"></span>
              <div class="select-menu-item-text">
                <input id="do_included" name="do" type="radio" value="included" />
                <h4>Not watching</h4>
                <span class="description">Be notified when participating or @mentioned.</span>
                <span class="js-select-button-text hidden-select-button-text">
                  <span class="octicon octicon-eye"></span>
                  Watch
                </span>
              </div>
            </div> <!-- /.select-menu-item -->

            <div class="select-menu-item js-navigation-item selected" role="menuitem" tabindex="0">
              <span class="select-menu-item-icon octicon octicon octicon-check"></span>
              <div class="select-menu-item-text">
                <input checked="checked" id="do_subscribed" name="do" type="radio" value="subscribed" />
                <h4>Watching</h4>
                <span class="description">Be notified of all conversations.</span>
                <span class="js-select-button-text hidden-select-button-text">
                  <span class="octicon octicon-eye"></span>
                  Unwatch
                </span>
              </div>
            </div> <!-- /.select-menu-item -->

            <div class="select-menu-item js-navigation-item " role="menuitem" tabindex="0">
              <span class="select-menu-item-icon octicon octicon-check"></span>
              <div class="select-menu-item-text">
                <input id="do_ignore" name="do" type="radio" value="ignore" />
                <h4>Ignoring</h4>
                <span class="description">Never be notified.</span>
                <span class="js-select-button-text hidden-select-button-text">
                  <span class="octicon octicon-mute"></span>
                  Stop ignoring
                </span>
              </div>
            </div> <!-- /.select-menu-item -->

          </div> <!-- /.select-menu-list -->

        </div> <!-- /.select-menu-modal -->
      </div> <!-- /.select-menu-modal-holder -->
    </div> <!-- /.select-menu -->

</form>
    </li>

  <li>
    
  <div class="js-toggler-container js-social-container starring-container ">

    <form accept-charset="UTF-8" action="/thomaskreuz/SPIKY-paper/unstar" class="js-toggler-form starred js-unstar-button" data-remote="true" method="post"><div style="margin:0;padding:0;display:inline"><input name="utf8" type="hidden" value="&#x2713;" /><input name="authenticity_token" type="hidden" value="ZjBXNC8dAjgguWxSjVBNsSyUF8e9WfzeZhrtuN7eByJVf80drARCluppXIPhme3oMhNpjWNFQe74wN+J1QGLMA==" /></div>
      <button
        class="minibutton with-count js-toggler-target star-button"
        aria-label="Unstar this repository" title="Unstar thomaskreuz/SPIKY-paper">
        <span class="octicon octicon-star"></span>
        Unstar
      </button>
        <a class="social-count js-social-count" href="/thomaskreuz/SPIKY-paper/stargazers">
          0
        </a>
</form>
    <form accept-charset="UTF-8" action="/thomaskreuz/SPIKY-paper/star" class="js-toggler-form unstarred js-star-button" data-remote="true" method="post"><div style="margin:0;padding:0;display:inline"><input name="utf8" type="hidden" value="&#x2713;" /><input name="authenticity_token" type="hidden" value="kikt6QJLUkHC7p6LYgQJtWnKGAcJTaCNPOlfSeycMsN2V3HtBJdn9F1T6P00L2+KDCKD/CXMldINmF9XEZ9a9Q==" /></div>
      <button
        class="minibutton with-count js-toggler-target star-button"
        aria-label="Star this repository" title="Star thomaskreuz/SPIKY-paper">
        <span class="octicon octicon-star"></span>
        Star
      </button>
        <a class="social-count js-social-count" href="/thomaskreuz/SPIKY-paper/stargazers">
          0
        </a>
</form>  </div>

  </li>


        <li>
          <a href="/thomaskreuz/SPIKY-paper/fork" class="minibutton with-count js-toggler-target fork-button tooltipped-n" title="Fork your own copy of thomaskreuz/SPIKY-paper to your account" aria-label="Fork your own copy of thomaskreuz/SPIKY-paper to your account" rel="nofollow" data-method="post">
            <span class="octicon octicon-repo-forked"></span>
            Fork
          </a>
          <a href="/thomaskreuz/SPIKY-paper/network" class="social-count">0</a>
        </li>

</ul>

        <h1 itemscope itemtype="http://data-vocabulary.org/Breadcrumb" class="entry-title public">
          <span class="mega-octicon octicon-repo"></span>
          <span class="author"><a href="/thomaskreuz" class="url fn" itemprop="url" rel="author"><span itemprop="title">thomaskreuz</span></a></span><!--
       --><span class="path-divider">/</span><!--
       --><strong><a href="/thomaskreuz/SPIKY-paper" class="js-current-repository js-repo-home-link">SPIKY-paper</a></strong>

          <span class="page-context-loader">
            <img alt="" height="16" src="https://assets-cdn.github.com/images/spinners/octocat-spinner-32.gif" width="16" />
          </span>

        </h1>
      </div><!-- /.container -->
    </div><!-- /.repohead -->

    <div class="container">
      <div class="repository-with-sidebar repo-container new-discussion-timeline  ">
        <div class="repository-sidebar clearfix">
            
<div class="sunken-menu vertical-right repo-nav js-repo-nav js-repository-container-pjax js-octicon-loaders" data-issue-count-url="/thomaskreuz/SPIKY-paper/issues/counts">
  <div class="sunken-menu-contents">
    <ul class="sunken-menu-group">
      <li class="tooltipped tooltipped-w" aria-label="Code">
        <a href="/thomaskreuz/SPIKY-paper" aria-label="Code" class="selected js-selected-navigation-item sunken-menu-item" data-hotkey="g c" data-pjax="true" data-selected-links="repo_source repo_downloads repo_commits repo_releases repo_tags repo_branches /thomaskreuz/SPIKY-paper">
          <span class="octicon octicon-code"></span> <span class="full-word">Code</span>
          <img alt="" class="mini-loader" height="16" src="https://assets-cdn.github.com/images/spinners/octocat-spinner-32.gif" width="16" />
</a>      </li>

        <li class="tooltipped tooltipped-w" aria-label="Issues">
          <a href="/thomaskreuz/SPIKY-paper/issues" aria-label="Issues" class="js-selected-navigation-item sunken-menu-item js-disable-pjax" data-hotkey="g i" data-selected-links="repo_issues repo_labels repo_milestones /thomaskreuz/SPIKY-paper/issues">
            <span class="octicon octicon-issue-opened"></span> <span class="full-word">Issues</span>
            <span class="js-issue-replace-counter"></span>
            <img alt="" class="mini-loader" height="16" src="https://assets-cdn.github.com/images/spinners/octocat-spinner-32.gif" width="16" />
</a>        </li>

      <li class="tooltipped tooltipped-w" aria-label="Pull Requests">
        <a href="/thomaskreuz/SPIKY-paper/pulls" aria-label="Pull Requests" class="js-selected-navigation-item sunken-menu-item js-disable-pjax" data-hotkey="g p" data-selected-links="repo_pulls /thomaskreuz/SPIKY-paper/pulls">
            <span class="octicon octicon-git-pull-request"></span> <span class="full-word">Pull Requests</span>
            <span class="js-pull-replace-counter"></span>
            <img alt="" class="mini-loader" height="16" src="https://assets-cdn.github.com/images/spinners/octocat-spinner-32.gif" width="16" />
</a>      </li>


        <li class="tooltipped tooltipped-w" aria-label="Wiki">
          <a href="/thomaskreuz/SPIKY-paper/wiki" aria-label="Wiki" class="js-selected-navigation-item sunken-menu-item js-disable-pjax" data-hotkey="g w" data-selected-links="repo_wiki /thomaskreuz/SPIKY-paper/wiki">
            <span class="octicon octicon-book"></span> <span class="full-word">Wiki</span>
            <img alt="" class="mini-loader" height="16" src="https://assets-cdn.github.com/images/spinners/octocat-spinner-32.gif" width="16" />
</a>        </li>
    </ul>
    <div class="sunken-menu-separator"></div>
    <ul class="sunken-menu-group">

      <li class="tooltipped tooltipped-w" aria-label="Pulse">
        <a href="/thomaskreuz/SPIKY-paper/pulse/weekly" aria-label="Pulse" class="js-selected-navigation-item sunken-menu-item" data-pjax="true" data-selected-links="pulse /thomaskreuz/SPIKY-paper/pulse/weekly">
          <span class="octicon octicon-pulse"></span> <span class="full-word">Pulse</span>
          <img alt="" class="mini-loader" height="16" src="https://assets-cdn.github.com/images/spinners/octocat-spinner-32.gif" width="16" />
</a>      </li>

      <li class="tooltipped tooltipped-w" aria-label="Graphs">
        <a href="/thomaskreuz/SPIKY-paper/graphs" aria-label="Graphs" class="js-selected-navigation-item sunken-menu-item" data-pjax="true" data-selected-links="repo_graphs repo_contributors /thomaskreuz/SPIKY-paper/graphs">
          <span class="octicon octicon-graph"></span> <span class="full-word">Graphs</span>
          <img alt="" class="mini-loader" height="16" src="https://assets-cdn.github.com/images/spinners/octocat-spinner-32.gif" width="16" />
</a>      </li>
    </ul>


      <div class="sunken-menu-separator"></div>
      <ul class="sunken-menu-group">
        <li class="tooltipped tooltipped-w" aria-label="Settings">
          <a href="/thomaskreuz/SPIKY-paper/settings" aria-label="Settings" class="js-selected-navigation-item sunken-menu-item" data-pjax="true" data-selected-links="repo_settings /thomaskreuz/SPIKY-paper/settings">
            <span class="octicon octicon-tools"></span> <span class="full-word">Settings</span>
            <img alt="" class="mini-loader" height="16" src="https://assets-cdn.github.com/images/spinners/octocat-spinner-32.gif" width="16" />
</a>        </li>
      </ul>
  </div>
</div>

              <div class="only-with-full-nav">
                
  
<div class="clone-url open"
  data-protocol-type="http"
  data-url="/users/set_protocol?protocol_selector=http&amp;protocol_type=push">
  <h3><span class="text-emphasized">HTTPS</span> clone URL</h3>
  <div class="input-group">
    <input type="text" class="input-mini input-monospace js-url-field"
           value="https://github.com/thomaskreuz/SPIKY-paper.git" readonly="readonly">
    <span class="input-group-button">
      <button aria-label="Copy to clipboard" class="js-zeroclipboard minibutton zeroclipboard-button" data-clipboard-text="https://github.com/thomaskreuz/SPIKY-paper.git" data-copied-hint="Copied!" type="button"><span class="octicon octicon-clippy"></span></button>
    </span>
  </div>
</div>

  
<div class="clone-url "
  data-protocol-type="ssh"
  data-url="/users/set_protocol?protocol_selector=ssh&amp;protocol_type=push">
  <h3><span class="text-emphasized">SSH</span> clone URL</h3>
  <div class="input-group">
    <input type="text" class="input-mini input-monospace js-url-field"
           value="git@github.com:thomaskreuz/SPIKY-paper.git" readonly="readonly">
    <span class="input-group-button">
      <button aria-label="Copy to clipboard" class="js-zeroclipboard minibutton zeroclipboard-button" data-clipboard-text="git@github.com:thomaskreuz/SPIKY-paper.git" data-copied-hint="Copied!" type="button"><span class="octicon octicon-clippy"></span></button>
    </span>
  </div>
</div>

  
<div class="clone-url "
  data-protocol-type="subversion"
  data-url="/users/set_protocol?protocol_selector=subversion&amp;protocol_type=push">
  <h3><span class="text-emphasized">Subversion</span> checkout URL</h3>
  <div class="input-group">
    <input type="text" class="input-mini input-monospace js-url-field"
           value="https://github.com/thomaskreuz/SPIKY-paper" readonly="readonly">
    <span class="input-group-button">
      <button aria-label="Copy to clipboard" class="js-zeroclipboard minibutton zeroclipboard-button" data-clipboard-text="https://github.com/thomaskreuz/SPIKY-paper" data-copied-hint="Copied!" type="button"><span class="octicon octicon-clippy"></span></button>
    </span>
  </div>
</div>


<p class="clone-options">You can clone with
      <a href="#" class="js-clone-selector" data-protocol="http">HTTPS</a>,
      <a href="#" class="js-clone-selector" data-protocol="ssh">SSH</a>,
      or <a href="#" class="js-clone-selector" data-protocol="subversion">Subversion</a>.
  <a href="https://help.github.com/articles/which-remote-url-should-i-use" class="help tooltipped tooltipped-n" aria-label="Get help on which URL is right for you.">
    <span class="octicon octicon-question"></span>
  </a>
</p>

  <a href="http://mac.github.com" data-url="github-mac://openRepo/https://github.com/thomaskreuz/SPIKY-paper" class="minibutton sidebar-button js-conduit-rewrite-url" title="Save thomaskreuz/SPIKY-paper to your computer and use it in GitHub Desktop." aria-label="Save thomaskreuz/SPIKY-paper to your computer and use it in GitHub Desktop.">
    <span class="octicon octicon-device-desktop"></span>
    Clone in Desktop
  </a>


                <a href="/thomaskreuz/SPIKY-paper/archive/b26e086bba9f6529f3199194b9f241500d468bb8.zip"
                   class="minibutton sidebar-button"
                   aria-label="Download thomaskreuz/SPIKY-paper as a zip file"
                   title="Download thomaskreuz/SPIKY-paper as a zip file"
                   rel="nofollow">
                  <span class="octicon octicon-cloud-download"></span>
                  Download ZIP
                </a>
              </div>
        </div><!-- /.repository-sidebar -->

        <div id="js-repo-pjax-container" class="repository-content context-loader-container" data-pjax-container>
          

<a href="/thomaskreuz/SPIKY-paper/blob/b26e086bba9f6529f3199194b9f241500d468bb8/SPIKY13.tex" class="hidden js-permalink-shortcut" data-hotkey="y">Permalink</a>

<!-- blob contrib key: blob_contributors:v21:20b162576d844fab94ae1a094056eb75 -->

<div class="file-navigation">
  
<div class="select-menu js-menu-container js-select-menu left">
  <span class="minibutton select-menu-button js-menu-target css-truncate" data-hotkey="w"
    data-master-branch="master"
    data-ref=""
    title=""
    role="button" aria-label="Switch branches or tags" tabindex="0" aria-haspopup="true">
    <span class="octicon octicon-git-branch"></span>
    <i>tree:</i>
    <span class="js-select-button css-truncate-target">b26e086bba</span>
  </span>

  <div class="select-menu-modal-holder js-menu-content js-navigation-container" data-pjax aria-hidden="true">

    <div class="select-menu-modal">
      <div class="select-menu-header">
        <span class="select-menu-title">Switch branches/tags</span>
        <span class="octicon octicon-x js-menu-close" role="button" aria-label="Close"></span>
      </div> <!-- /.select-menu-header -->

      <div class="select-menu-filters">
        <div class="select-menu-text-filter">
          <input type="text" aria-label="Find or create a branch…" id="context-commitish-filter-field" class="js-filterable-field js-navigation-enable" placeholder="Find or create a branch…">
        </div>
        <div class="select-menu-tabs">
          <ul>
            <li class="select-menu-tab">
              <a href="#" data-tab-filter="branches" class="js-select-menu-tab">Branches</a>
            </li>
            <li class="select-menu-tab">
              <a href="#" data-tab-filter="tags" class="js-select-menu-tab">Tags</a>
            </li>
          </ul>
        </div><!-- /.select-menu-tabs -->
      </div><!-- /.select-menu-filters -->

      <div class="select-menu-list select-menu-tab-bucket js-select-menu-tab-bucket" data-tab-filter="branches">

        <div data-filterable-for="context-commitish-filter-field" data-filterable-type="substring">


            <div class="select-menu-item js-navigation-item ">
              <span class="select-menu-item-icon octicon octicon-check"></span>
              <a href="/thomaskreuz/SPIKY-paper/blob/master/SPIKY13.tex"
                 data-name="master"
                 data-skip-pjax="true"
                 rel="nofollow"
                 class="js-navigation-open select-menu-item-text css-truncate-target"
                 title="master">master</a>
            </div> <!-- /.select-menu-item -->
        </div>

          <form accept-charset="UTF-8" action="/thomaskreuz/SPIKY-paper/branches" class="js-create-branch select-menu-item select-menu-new-item-form js-navigation-item js-new-item-form" method="post"><div style="margin:0;padding:0;display:inline"><input name="utf8" type="hidden" value="&#x2713;" /><input name="authenticity_token" type="hidden" value="twmFuLNdiAMDrsez+AqhqFK9hraJ8q2cwaK99nJssM5c1epRPXAeeC4vwR/d/X0SrX0k0pcFIA3L4OS5LEksuA==" /></div>
            <span class="octicon octicon-git-branch select-menu-item-icon"></span>
            <div class="select-menu-item-text">
              <h4>Create branch: <span class="js-new-item-name"></span></h4>
              <span class="description">from ‘b26e086’</span>
            </div>
            <input type="hidden" name="name" id="name" class="js-new-item-value">
            <input type="hidden" name="branch" id="branch" value="b26e086bba9f6529f3199194b9f241500d468bb8">
            <input type="hidden" name="path" id="path" value="SPIKY13.tex">
          </form> <!-- /.select-menu-item -->

      </div> <!-- /.select-menu-list -->

      <div class="select-menu-list select-menu-tab-bucket js-select-menu-tab-bucket" data-tab-filter="tags">
        <div data-filterable-for="context-commitish-filter-field" data-filterable-type="substring">


        </div>

        <div class="select-menu-no-results">Nothing to show</div>
      </div> <!-- /.select-menu-list -->

    </div> <!-- /.select-menu-modal -->
  </div> <!-- /.select-menu-modal-holder -->
</div> <!-- /.select-menu -->

  <div class="button-group right">
    <a href="/thomaskreuz/SPIKY-paper/find/b26e086bba9f6529f3199194b9f241500d468bb8"
          class="js-show-file-finder minibutton empty-icon tooltipped tooltipped-s"
          data-pjax
          data-hotkey="t"
          aria-label="Quickly jump between files">
      <span class="octicon octicon-list-unordered"></span>
    </a>
    <button class="js-zeroclipboard minibutton zeroclipboard-button"
          data-clipboard-text="SPIKY13.tex"
          aria-label="Copy to clipboard"
          data-copied-hint="Copied!">
      <span class="octicon octicon-clippy"></span>
    </button>
  </div>

  <div class="breadcrumb">
    <span class='repo-root js-repo-root'><span itemscope="" itemtype="http://data-vocabulary.org/Breadcrumb"><a href="/thomaskreuz/SPIKY-paper/tree/b26e086bba9f6529f3199194b9f241500d468bb8" class="" data-branch="b26e086bba9f6529f3199194b9f241500d468bb8" data-direction="back" data-pjax="true" itemscope="url" rel="nofollow"><span itemprop="title">SPIKY-paper</span></a></span></span><span class="separator"> / </span><strong class="final-path">SPIKY13.tex</strong>
  </div>
</div>


  <div class="commit file-history-tease">
      <img alt="thomaskreuz" class="main-avatar" data-user="7440733" height="24" src="https://avatars0.githubusercontent.com/u/7440733?v=2&amp;s=48" width="24" />
      <span class="author"><a href="/thomaskreuz" rel="author">thomaskreuz</a></span>
      <time datetime="2014-08-12T03:04:21+02:00" is="relative-time">August 12, 2014</time>
      <div class="commit-title">
          <a href="/thomaskreuz/SPIKY-paper/commit/452cca44f5e17c15c0de83d3d63597547df661e3" class="message" data-pjax="true" title="More references Fig2">More references Fig2</a>
      </div>

    <div class="participation">
      <p class="quickstat"><a href="#blob_contributors_box" rel="facebox"><strong>1</strong>  contributor</a></p>
      

    </div>
    <div id="blob_contributors_box" style="display:none">
      <h2 class="facebox-header">Users who have contributed to this file</h2>
      <ul class="facebox-user-list">
          <li class="facebox-user-list-item">
            <img alt="thomaskreuz" data-user="7440733" height="24" src="https://avatars0.githubusercontent.com/u/7440733?v=2&amp;s=48" width="24" />
            <a href="/thomaskreuz">thomaskreuz</a>
          </li>
      </ul>
    </div>
  </div>

<div class="file-box">
  <div class="file">
    <div class="meta clearfix">
      <div class="info file-name">
          <span>641 lines (453 sloc)</span>
          <span class="meta-divider"></span>
        <span>64.432 kb</span>
      </div>
      <div class="actions">
        <div class="button-group">
          <a href="/thomaskreuz/SPIKY-paper/raw/b26e086bba9f6529f3199194b9f241500d468bb8/SPIKY13.tex" class="minibutton " id="raw-url">Raw</a>
            <a href="/thomaskreuz/SPIKY-paper/blame/b26e086bba9f6529f3199194b9f241500d468bb8/SPIKY13.tex" class="minibutton js-update-url-with-hash">Blame</a>
          <a href="/thomaskreuz/SPIKY-paper/commits/b26e086bba9f6529f3199194b9f241500d468bb8/SPIKY13.tex" class="minibutton " rel="nofollow">History</a>
        </div><!-- /.button-group -->


            <a class="octicon-button disabled tooltipped tooltipped-w" href="#"
               aria-label="You must be on a branch to make or propose changes to this file"><span class="octicon octicon-pencil"></span></a>

          <a class="octicon-button danger disabled tooltipped tooltipped-w" href="#"
             aria-label="You must be on a branch to make or propose changes to this file">
          <span class="octicon octicon-trashcan"></span>
        </a>
      </div><!-- /.actions -->
    </div>
      
  <div class="blob-wrapper data type-tex">
      <table class="highlight tab-size-8 js-file-line-container">
      <tr>
        <td id="L1" class="blob-line-num js-line-number" data-line-number="1"></td>
        <td id="LC1" class="blob-line-code js-file-line"><span class="k">\documentclass</span><span class="na">[10pt,twocolumn]</span><span class="nb">{</span>elsart5p<span class="nb">}</span></td>
      </tr>
      <tr>
        <td id="L2" class="blob-line-num js-line-number" data-line-number="2"></td>
        <td id="LC2" class="blob-line-code js-file-line"><span class="k">\usepackage</span><span class="nb">{</span>graphicx<span class="nb">}</span></td>
      </tr>
      <tr>
        <td id="L3" class="blob-line-num js-line-number" data-line-number="3"></td>
        <td id="LC3" class="blob-line-code js-file-line"><span class="k">\usepackage</span><span class="nb">{</span>dcolumn<span class="nb">}</span></td>
      </tr>
      <tr>
        <td id="L4" class="blob-line-num js-line-number" data-line-number="4"></td>
        <td id="LC4" class="blob-line-code js-file-line"><span class="k">\usepackage</span><span class="nb">{</span>amsmath<span class="nb">}</span></td>
      </tr>
      <tr>
        <td id="L5" class="blob-line-num js-line-number" data-line-number="5"></td>
        <td id="LC5" class="blob-line-code js-file-line"><span class="k">\usepackage</span><span class="nb">{</span>amssymb<span class="nb">}</span></td>
      </tr>
      <tr>
        <td id="L6" class="blob-line-num js-line-number" data-line-number="6"></td>
        <td id="LC6" class="blob-line-code js-file-line"><span class="k">\usepackage</span><span class="nb">{</span>bm<span class="nb">}</span></td>
      </tr>
      <tr>
        <td id="L7" class="blob-line-num js-line-number" data-line-number="7"></td>
        <td id="LC7" class="blob-line-code js-file-line"><span class="k">\usepackage</span><span class="nb">{</span>natbib<span class="nb">}</span></td>
      </tr>
      <tr>
        <td id="L8" class="blob-line-num js-line-number" data-line-number="8"></td>
        <td id="LC8" class="blob-line-code js-file-line"><span class="k">\usepackage</span><span class="nb">{</span>hyperref<span class="nb">}</span></td>
      </tr>
      <tr>
        <td id="L9" class="blob-line-num js-line-number" data-line-number="9"></td>
        <td id="LC9" class="blob-line-code js-file-line"><span class="c">%\usepackage[noperiod]{jabbrv}</span></td>
      </tr>
      <tr>
        <td id="L10" class="blob-line-num js-line-number" data-line-number="10"></td>
        <td id="LC10" class="blob-line-code js-file-line">
</td>
      </tr>
      <tr>
        <td id="L11" class="blob-line-num js-line-number" data-line-number="11"></td>
        <td id="LC11" class="blob-line-code js-file-line"><span class="c">%</span></td>
      </tr>
      <tr>
        <td id="L12" class="blob-line-num js-line-number" data-line-number="12"></td>
        <td id="LC12" class="blob-line-code js-file-line"><span class="c">% Introduction</span></td>
      </tr>
      <tr>
        <td id="L13" class="blob-line-num js-line-number" data-line-number="13"></td>
        <td id="LC13" class="blob-line-code js-file-line"><span class="c">% ============</span></td>
      </tr>
      <tr>
        <td id="L14" class="blob-line-num js-line-number" data-line-number="14"></td>
        <td id="LC14" class="blob-line-code js-file-line"><span class="c">%</span></td>
      </tr>
      <tr>
        <td id="L15" class="blob-line-num js-line-number" data-line-number="15"></td>
        <td id="LC15" class="blob-line-code js-file-line"><span class="c">% Measures</span></td>
      </tr>
      <tr>
        <td id="L16" class="blob-line-num js-line-number" data-line-number="16"></td>
        <td id="LC16" class="blob-line-code js-file-line"><span class="c">% ========</span></td>
      </tr>
      <tr>
        <td id="L17" class="blob-line-num js-line-number" data-line-number="17"></td>
        <td id="LC17" class="blob-line-code js-file-line"><span class="c">% ISI        																	Done</span></td>
      </tr>
      <tr>
        <td id="L18" class="blob-line-num js-line-number" data-line-number="18"></td>
        <td id="LC18" class="blob-line-code js-file-line"><span class="c">% SPIKE      																	Done</span></td>
      </tr>
      <tr>
        <td id="L19" class="blob-line-num js-line-number" data-line-number="19"></td>
        <td id="LC19" class="blob-line-code js-file-line"><span class="c">% SPIKE-realtime      															Done</span></td>
      </tr>
      <tr>
        <td id="L20" class="blob-line-num js-line-number" data-line-number="20"></td>
        <td id="LC20" class="blob-line-code js-file-line"><span class="c">% SPIKE-future      																Done</span></td>
      </tr>
      <tr>
        <td id="L21" class="blob-line-num js-line-number" data-line-number="21"></td>
        <td id="LC21" class="blob-line-code js-file-line"><span class="c">% Edge effect																	Done</span></td>
      </tr>
      <tr>
        <td id="L22" class="blob-line-num js-line-number" data-line-number="22"></td>
        <td id="LC22" class="blob-line-code js-file-line"><span class="c">% Sampling   																	Done</span></td>
      </tr>
      <tr>
        <td id="L23" class="blob-line-num js-line-number" data-line-number="23"></td>
        <td id="LC23" class="blob-line-code js-file-line"><span class="c">% Representations      															Done</span></td>
      </tr>
      <tr>
        <td id="L24" class="blob-line-num js-line-number" data-line-number="24"></td>
        <td id="LC24" class="blob-line-code js-file-line"><span class="c">%</span></td>
      </tr>
      <tr>
        <td id="L25" class="blob-line-num js-line-number" data-line-number="25"></td>
        <td id="LC25" class="blob-line-code js-file-line"><span class="c">% SPIKY</span></td>
      </tr>
      <tr>
        <td id="L26" class="blob-line-num js-line-number" data-line-number="26"></td>
        <td id="LC26" class="blob-line-code js-file-line"><span class="c">% =====</span></td>
      </tr>
      <tr>
        <td id="L27" class="blob-line-num js-line-number" data-line-number="27"></td>
        <td id="LC27" class="blob-line-code js-file-line"><span class="c">% SPIKY-Intro   																	Done</span></td>
      </tr>
      <tr>
        <td id="L28" class="blob-line-num js-line-number" data-line-number="28"></td>
        <td id="LC28" class="blob-line-code js-file-line"><span class="c">% Structure     																	Done</span></td>
      </tr>
      <tr>
        <td id="L29" class="blob-line-num js-line-number" data-line-number="29"></td>
        <td id="LC29" class="blob-line-code js-file-line"><span class="c">% Input (data format), Spike Train Generator (Patterns, by hand)					Done </span></td>
      </tr>
      <tr>
        <td id="L30" class="blob-line-num js-line-number" data-line-number="30"></td>
        <td id="LC30" class="blob-line-code js-file-line"><span class="c">% Output      																	Done</span></td>
      </tr>
      <tr>
        <td id="L31" class="blob-line-num js-line-number" data-line-number="31"></td>
        <td id="LC31" class="blob-line-code js-file-line"><span class="c">% Figure Layout      															Done</span></td>
      </tr>
      <tr>
        <td id="L32" class="blob-line-num js-line-number" data-line-number="32"></td>
        <td id="LC32" class="blob-line-code js-file-line"><span class="c">% GUI vs Loop      																Done</span></td>
      </tr>
      <tr>
        <td id="L33" class="blob-line-num js-line-number" data-line-number="33"></td>
        <td id="LC33" class="blob-line-code js-file-line"><span class="c">% Spike train surrogates (Significance), Comparison with Poisson					Done</span></td>
      </tr>
      <tr>
        <td id="L34" class="blob-line-num js-line-number" data-line-number="34"></td>
        <td id="LC34" class="blob-line-code js-file-line"><span class="c">% Comparison with other measures/implementations, Computational Cost (Memory, Speed)   Neb</span></td>
      </tr>
      <tr>
        <td id="L35" class="blob-line-num js-line-number" data-line-number="35"></td>
        <td id="LC35" class="blob-line-code js-file-line"><span class="c">%</span></td>
      </tr>
      <tr>
        <td id="L36" class="blob-line-num js-line-number" data-line-number="36"></td>
        <td id="LC36" class="blob-line-code js-file-line"><span class="c">% Discussion</span></td>
      </tr>
      <tr>
        <td id="L37" class="blob-line-num js-line-number" data-line-number="37"></td>
        <td id="LC37" class="blob-line-code js-file-line"><span class="c">% ==========</span></td>
      </tr>
      <tr>
        <td id="L38" class="blob-line-num js-line-number" data-line-number="38"></td>
        <td id="LC38" class="blob-line-code js-file-line"><span class="c">% Limitations</span></td>
      </tr>
      <tr>
        <td id="L39" class="blob-line-num js-line-number" data-line-number="39"></td>
        <td id="LC39" class="blob-line-code js-file-line"><span class="c">% Outlook</span></td>
      </tr>
      <tr>
        <td id="L40" class="blob-line-num js-line-number" data-line-number="40"></td>
        <td id="LC40" class="blob-line-code js-file-line"><span class="c">%</span></td>
      </tr>
      <tr>
        <td id="L41" class="blob-line-num js-line-number" data-line-number="41"></td>
        <td id="LC41" class="blob-line-code js-file-line">
</td>
      </tr>
      <tr>
        <td id="L42" class="blob-line-num js-line-number" data-line-number="42"></td>
        <td id="LC42" class="blob-line-code js-file-line"><span class="k">\begin</span><span class="nb">{</span>document<span class="nb">}</span></td>
      </tr>
      <tr>
        <td id="L43" class="blob-line-num js-line-number" data-line-number="43"></td>
        <td id="LC43" class="blob-line-code js-file-line">
</td>
      </tr>
      <tr>
        <td id="L44" class="blob-line-num js-line-number" data-line-number="44"></td>
        <td id="LC44" class="blob-line-code js-file-line"><span class="k">\begin</span><span class="nb">{</span>frontmatter<span class="nb">}</span></td>
      </tr>
      <tr>
        <td id="L45" class="blob-line-num js-line-number" data-line-number="45"></td>
        <td id="LC45" class="blob-line-code js-file-line">
</td>
      </tr>
      <tr>
        <td id="L46" class="blob-line-num js-line-number" data-line-number="46"></td>
        <td id="LC46" class="blob-line-code js-file-line"><span class="k">\title</span><span class="nb">{</span>SPIKY: A graphical user interface for monitoring spike train synchrony<span class="nb">}</span></td>
      </tr>
      <tr>
        <td id="L47" class="blob-line-num js-line-number" data-line-number="47"></td>
        <td id="LC47" class="blob-line-code js-file-line">
</td>
      </tr>
      <tr>
        <td id="L48" class="blob-line-num js-line-number" data-line-number="48"></td>
        <td id="LC48" class="blob-line-code js-file-line"><span class="k">\author</span><span class="nb">{</span>Nebojsa Bozanic<span class="nb">}</span>,</td>
      </tr>
      <tr>
        <td id="L49" class="blob-line-num js-line-number" data-line-number="49"></td>
        <td id="LC49" class="blob-line-code js-file-line"><span class="k">\ead</span><span class="nb">{</span>strance@gmail.com<span class="nb">}</span></td>
      </tr>
      <tr>
        <td id="L50" class="blob-line-num js-line-number" data-line-number="50"></td>
        <td id="LC50" class="blob-line-code js-file-line"><span class="k">\author</span><span class="nb">{</span>Thomas Kreuz<span class="nb">}</span></td>
      </tr>
      <tr>
        <td id="L51" class="blob-line-num js-line-number" data-line-number="51"></td>
        <td id="LC51" class="blob-line-code js-file-line"><span class="k">\ead</span><span class="nb">{</span>thomas.kreuz@cnr.it<span class="nb">}</span></td>
      </tr>
      <tr>
        <td id="L52" class="blob-line-num js-line-number" data-line-number="52"></td>
        <td id="LC52" class="blob-line-code js-file-line">
</td>
      </tr>
      <tr>
        <td id="L53" class="blob-line-num js-line-number" data-line-number="53"></td>
        <td id="LC53" class="blob-line-code js-file-line">
</td>
      </tr>
      <tr>
        <td id="L54" class="blob-line-num js-line-number" data-line-number="54"></td>
        <td id="LC54" class="blob-line-code js-file-line"><span class="k">\address</span><span class="nb">{</span>Institute for complex systems, CNR, Sesto Fiorentino, Italy<span class="nb">}</span></td>
      </tr>
      <tr>
        <td id="L55" class="blob-line-num js-line-number" data-line-number="55"></td>
        <td id="LC55" class="blob-line-code js-file-line">
</td>
      </tr>
      <tr>
        <td id="L56" class="blob-line-num js-line-number" data-line-number="56"></td>
        <td id="LC56" class="blob-line-code js-file-line">
</td>
      </tr>
      <tr>
        <td id="L57" class="blob-line-num js-line-number" data-line-number="57"></td>
        <td id="LC57" class="blob-line-code js-file-line"><span class="k">\date</span><span class="nb">{</span><span class="k">\today</span><span class="nb">}</span></td>
      </tr>
      <tr>
        <td id="L58" class="blob-line-num js-line-number" data-line-number="58"></td>
        <td id="LC58" class="blob-line-code js-file-line">
</td>
      </tr>
      <tr>
        <td id="L59" class="blob-line-num js-line-number" data-line-number="59"></td>
        <td id="LC59" class="blob-line-code js-file-line"><span class="k">\begin</span><span class="nb">{</span>abstract<span class="nb">}</span></td>
      </tr>
      <tr>
        <td id="L60" class="blob-line-num js-line-number" data-line-number="60"></td>
        <td id="LC60" class="blob-line-code js-file-line">
</td>
      </tr>
      <tr>
        <td id="L61" class="blob-line-num js-line-number" data-line-number="61"></td>
        <td id="LC61" class="blob-line-code js-file-line">XXXXX Preliminary, to be edited in the end XXXXX</td>
      </tr>
      <tr>
        <td id="L62" class="blob-line-num js-line-number" data-line-number="62"></td>
        <td id="LC62" class="blob-line-code js-file-line">Recently, the SPIKE-distance has been proposed as a measure of spike train synchrony which is both parameter-free and time-scale independent. Since it relies on instantaneous estimates of spike train dissimilarity, it is also time-resolved which makes it possible to track changes in instantaneous clustering, i.e., time-localized patterns of (dis)similarity among multiple spike trains. Further features include selective and triggered temporal averaging as well as the instantaneous comparison of spike train groups. Besides the regular SPIKE-distance, there also exists a causal variant which is defined such that the instantaneous values of dissimilarity rely on past information only so that time-resolved spike train synchrony can be estimated in real-time. Finally, here we introduce the future SPIKE-distance which can be used in triggered temporal averaging in order to evaluate the effect of certain spikes or of certain stimuli features on future spiking. In the first part of this report we address some of the computational aspects in the calculation and implementation of the SPIKE-distance while in the second part we present SPIKY, a graphical user interface which facilitates the application of the SPIKE-distance and all its variants to both simulated and real data.</td>
      </tr>
      <tr>
        <td id="L63" class="blob-line-num js-line-number" data-line-number="63"></td>
        <td id="LC63" class="blob-line-code js-file-line"> </td>
      </tr>
      <tr>
        <td id="L64" class="blob-line-num js-line-number" data-line-number="64"></td>
        <td id="LC64" class="blob-line-code js-file-line"><span class="k">\end</span><span class="nb">{</span>abstract<span class="nb">}</span></td>
      </tr>
      <tr>
        <td id="L65" class="blob-line-num js-line-number" data-line-number="65"></td>
        <td id="LC65" class="blob-line-code js-file-line">
</td>
      </tr>
      <tr>
        <td id="L66" class="blob-line-num js-line-number" data-line-number="66"></td>
        <td id="LC66" class="blob-line-code js-file-line">
</td>
      </tr>
      <tr>
        <td id="L67" class="blob-line-num js-line-number" data-line-number="67"></td>
        <td id="LC67" class="blob-line-code js-file-line"><span class="c">%\begin{keyword}</span></td>
      </tr>
      <tr>
        <td id="L68" class="blob-line-num js-line-number" data-line-number="68"></td>
        <td id="LC68" class="blob-line-code js-file-line"><span class="c">%    time series analysis; spike trains; clustering; neuronal coding; SPIKE-distance; ISI-distance; graphical user interface; Matlab; SPIKY</span></td>
      </tr>
      <tr>
        <td id="L69" class="blob-line-num js-line-number" data-line-number="69"></td>
        <td id="LC69" class="blob-line-code js-file-line"><span class="c">%\end{keyword}</span></td>
      </tr>
      <tr>
        <td id="L70" class="blob-line-num js-line-number" data-line-number="70"></td>
        <td id="LC70" class="blob-line-code js-file-line">
</td>
      </tr>
      <tr>
        <td id="L71" class="blob-line-num js-line-number" data-line-number="71"></td>
        <td id="LC71" class="blob-line-code js-file-line"><span class="k">\end</span><span class="nb">{</span>frontmatter<span class="nb">}</span></td>
      </tr>
      <tr>
        <td id="L72" class="blob-line-num js-line-number" data-line-number="72"></td>
        <td id="LC72" class="blob-line-code js-file-line">
</td>
      </tr>
      <tr>
        <td id="L73" class="blob-line-num js-line-number" data-line-number="73"></td>
        <td id="LC73" class="blob-line-code js-file-line"><span class="k">\newcommand</span><span class="nb">{</span><span class="k">\abb</span><span class="nb">}{</span><span class="k">\small\sf</span><span class="nb">}</span></td>
      </tr>
      <tr>
        <td id="L74" class="blob-line-num js-line-number" data-line-number="74"></td>
        <td id="LC74" class="blob-line-code js-file-line"><span class="k">\maketitle</span></td>
      </tr>
      <tr>
        <td id="L75" class="blob-line-num js-line-number" data-line-number="75"></td>
        <td id="LC75" class="blob-line-code js-file-line">
</td>
      </tr>
      <tr>
        <td id="L76" class="blob-line-num js-line-number" data-line-number="76"></td>
        <td id="LC76" class="blob-line-code js-file-line"><span class="c">%</span></td>
      </tr>
      <tr>
        <td id="L77" class="blob-line-num js-line-number" data-line-number="77"></td>
        <td id="LC77" class="blob-line-code js-file-line"><span class="c">%</span></td>
      </tr>
      <tr>
        <td id="L78" class="blob-line-num js-line-number" data-line-number="78"></td>
        <td id="LC78" class="blob-line-code js-file-line"><span class="c">% *************************************************************************************</span></td>
      </tr>
      <tr>
        <td id="L79" class="blob-line-num js-line-number" data-line-number="79"></td>
        <td id="LC79" class="blob-line-code js-file-line"><span class="c">% ******************************************************** Introduction ***************</span></td>
      </tr>
      <tr>
        <td id="L80" class="blob-line-num js-line-number" data-line-number="80"></td>
        <td id="LC80" class="blob-line-code js-file-line"><span class="c">% *************************************************************************************</span></td>
      </tr>
      <tr>
        <td id="L81" class="blob-line-num js-line-number" data-line-number="81"></td>
        <td id="LC81" class="blob-line-code js-file-line"><span class="c">%</span></td>
      </tr>
      <tr>
        <td id="L82" class="blob-line-num js-line-number" data-line-number="82"></td>
        <td id="LC82" class="blob-line-code js-file-line"><span class="c">%</span></td>
      </tr>
      <tr>
        <td id="L83" class="blob-line-num js-line-number" data-line-number="83"></td>
        <td id="LC83" class="blob-line-code js-file-line"><span class="k">\section</span><span class="nb">{</span><span class="k">\label</span><span class="nb">{</span>s:Intro<span class="nb">}</span> Introduction<span class="nb">}</span></td>
      </tr>
      <tr>
        <td id="L84" class="blob-line-num js-line-number" data-line-number="84"></td>
        <td id="LC84" class="blob-line-code js-file-line">
</td>
      </tr>
      <tr>
        <td id="L85" class="blob-line-num js-line-number" data-line-number="85"></td>
        <td id="LC85" class="blob-line-code js-file-line">Spike train distances are measures that when applied to a set of spike trains yield low (high) values for very similar (dissimilar) spike trains. Essentially there are two major scenarios for using spike train distances to estimate the degree of synchrony between two or more spike trains. The most common scenario is the simultaneous recording of a neuronal population, typically in some kind of spatial multi-channel setup. If different neurons emit spikes simultaneously, these spikes are &quot;synchronous&quot; in the classical sense of the word, which is derived from Greek and describes events &quot;occurring at the same time&quot; <span class="k">\citep</span><span class="nb">{</span>Pikovsky01<span class="nb">}</span>. Synchronization between individual neurons has been proven to be of high prevalence in many different neuronal circuits. Examples include the visual cortex <span class="k">\citep</span><span class="nb">{</span>Tiesinga08<span class="nb">}</span> and the retina <span class="k">\citep</span><span class="nb">{</span>Shlens08<span class="nb">}</span> for which numbers of spike coincidences considerably above chance level have been reported. As of now many open questions remain regarding the spatial scale and the nature of interactions (e.g., pairwise or higher order, see <span class="k">\citep</span><span class="nb">{</span>Nirenberg07<span class="nb">}</span>) as well as their functional significance for neuronal coding and information processing <span class="k">\citep</span><span class="nb">{</span>Kumar10<span class="nb">}</span>.</td>
      </tr>
      <tr>
        <td id="L86" class="blob-line-num js-line-number" data-line-number="86"></td>
        <td id="LC86" class="blob-line-code js-file-line">
</td>
      </tr>
      <tr>
        <td id="L87" class="blob-line-num js-line-number" data-line-number="87"></td>
        <td id="LC87" class="blob-line-code js-file-line">In the other major scenario the neuronal spiking response is recorded in different time intervals. In order to allow a meaningful confrontation there has to be a temporal reference point which is typically set by some kind of trigger (e.g., the onset of an external stimulation). There are two prominent applications for this successive trials scenario. The first one is to test for the reliability of individual neurons by presenting the same stimulus repeatedly (e.g., <span class="k">\citet</span><span class="nb">{</span>Mainen95<span class="nb">}</span>). In contrast, when different stimuli are used one is mostly interested in the features of the response that provide the optimal discrimination since this might allow drawing conclusions regarding the nature of the neuronal coding (e.g., <span class="k">\citet</span><span class="nb">{</span>Victor05<span class="nb">}</span>). These two applications are related since for a good clustering performance one needs not only a pronounced discrimination between stimuli (high inter-stimulus spike train distances) but also a high reliability for the same stimulus (low intra-stimulus spike train distances).</td>
      </tr>
      <tr>
        <td id="L88" class="blob-line-num js-line-number" data-line-number="88"></td>
        <td id="LC88" class="blob-line-code js-file-line">
</td>
      </tr>
      <tr>
        <td id="L89" class="blob-line-num js-line-number" data-line-number="89"></td>
        <td id="LC89" class="blob-line-code js-file-line">In both population and successive trial recordings a high temporal resolution would be very desirable. Analyzing the changing spiking patterns of large ensembles of single neurons in epilepsy patients could lead to a better understanding of the mechanisms of seizure generation, propagation, and termination <span class="k">\citep</span><span class="nb">{</span>Truccolo11, Bower12<span class="nb">}</span> in the same way as the analysis of neuronal responses to successive presentations of time-dependent stimuli could help to understand the role of synchronous firing in neuronal coding <span class="k">\citep</span><span class="nb">{</span>Miller08<span class="nb">}</span>. Moreover, in population recordings it would be even more advantageous to be able to monitor spike train synchrony in realtime. In epilepsy this would be a necessary condition for the implementation of a prospective seizure prediction algorithm <span class="k">\citep</span><span class="nb">{</span>Mormann07<span class="nb">}</span>, but it could also be very useful for the rapid online decoding of neural signals needed to control prosthetics <span class="k">\citep</span><span class="nb">{</span>Hochberg06, Sanchez08<span class="nb">}</span>.</td>
      </tr>
      <tr>
        <td id="L90" class="blob-line-num js-line-number" data-line-number="90"></td>
        <td id="LC90" class="blob-line-code js-file-line">
</td>
      </tr>
      <tr>
        <td id="L91" class="blob-line-num js-line-number" data-line-number="91"></td>
        <td id="LC91" class="blob-line-code js-file-line">XXXXX to be continued XXXXX</td>
      </tr>
      <tr>
        <td id="L92" class="blob-line-num js-line-number" data-line-number="92"></td>
        <td id="LC92" class="blob-line-code js-file-line">
</td>
      </tr>
      <tr>
        <td id="L93" class="blob-line-num js-line-number" data-line-number="93"></td>
        <td id="LC93" class="blob-line-code js-file-line">Many different representations, i.e. degrees of information reduction</td>
      </tr>
      <tr>
        <td id="L94" class="blob-line-num js-line-number" data-line-number="94"></td>
        <td id="LC94" class="blob-line-code js-file-line">
</td>
      </tr>
      <tr>
        <td id="L95" class="blob-line-num js-line-number" data-line-number="95"></td>
        <td id="LC95" class="blob-line-code js-file-line">Need for a graphical user interface (GUI)</td>
      </tr>
      <tr>
        <td id="L96" class="blob-line-num js-line-number" data-line-number="96"></td>
        <td id="LC96" class="blob-line-code js-file-line">
</td>
      </tr>
      <tr>
        <td id="L97" class="blob-line-num js-line-number" data-line-number="97"></td>
        <td id="LC97" class="blob-line-code js-file-line">Another advantage of having a graphical user interface is the easy accessibility to users that are not familiar with programming environments such as Matlab.</td>
      </tr>
      <tr>
        <td id="L98" class="blob-line-num js-line-number" data-line-number="98"></td>
        <td id="LC98" class="blob-line-code js-file-line">
</td>
      </tr>
      <tr>
        <td id="L99" class="blob-line-num js-line-number" data-line-number="99"></td>
        <td id="LC99" class="blob-line-code js-file-line">Improvements regarding the possible size of the dataset (memory and speed)</td>
      </tr>
      <tr>
        <td id="L100" class="blob-line-num js-line-number" data-line-number="100"></td>
        <td id="LC100" class="blob-line-code js-file-line">
</td>
      </tr>
      <tr>
        <td id="L101" class="blob-line-num js-line-number" data-line-number="101"></td>
        <td id="LC101" class="blob-line-code js-file-line">edge effect</td>
      </tr>
      <tr>
        <td id="L102" class="blob-line-num js-line-number" data-line-number="102"></td>
        <td id="LC102" class="blob-line-code js-file-line">
</td>
      </tr>
      <tr>
        <td id="L103" class="blob-line-num js-line-number" data-line-number="103"></td>
        <td id="LC103" class="blob-line-code js-file-line">Computational aspects</td>
      </tr>
      <tr>
        <td id="L104" class="blob-line-num js-line-number" data-line-number="104"></td>
        <td id="LC104" class="blob-line-code js-file-line">
</td>
      </tr>
      <tr>
        <td id="L105" class="blob-line-num js-line-number" data-line-number="105"></td>
        <td id="LC105" class="blob-line-code js-file-line">XXXXX to be continued XXXXX</td>
      </tr>
      <tr>
        <td id="L106" class="blob-line-num js-line-number" data-line-number="106"></td>
        <td id="LC106" class="blob-line-code js-file-line">
</td>
      </tr>
      <tr>
        <td id="L107" class="blob-line-num js-line-number" data-line-number="107"></td>
        <td id="LC107" class="blob-line-code js-file-line">The remainder of this paper is organized as follows. In Section <span class="k">\ref</span><span class="nb">{</span>s:Measures<span class="nb">}</span> we introduce and discuss the different measures available in SPIKY. These include the ISI-distance and the SPIKE-distance (as well as its realtime and future variants). In Subsection <span class="k">\ref</span><span class="nb">{</span>ss:Improvements<span class="nb">}</span> we present the improvements we have realized in this new implementation of the measures such as the correction of the edge-effect and the increase of memory efficiency by avoiding sampling. An overview of the different levels of information reduction is given in Subsection <span class="k">\ref</span><span class="nb">{</span>ss:Information-reduction<span class="nb">}</span>. These range from the most detailed representation in which one instantaneous value is obtained for each pair of spike trains to the most condensed representation in which successive temporal and spatial averaging leads to one single distance value which describes the overall level of synchrony for a group of spike trains over a given time interval.</td>
      </tr>
      <tr>
        <td id="L108" class="blob-line-num js-line-number" data-line-number="108"></td>
        <td id="LC108" class="blob-line-code js-file-line">
</td>
      </tr>
      <tr>
        <td id="L109" class="blob-line-num js-line-number" data-line-number="109"></td>
        <td id="LC109" class="blob-line-code js-file-line">SPIKY, our graphical user interface (GUI) for monitoring spike train synchrony, is presented in Section <span class="k">\ref</span><span class="nb">{</span>s:SPIKY<span class="nb">}</span>. In Subsection <span class="k">\ref</span><span class="nb">{</span>ss:Access<span class="nb">}</span> explain how to get access to the codes and the large library of documentation (which includes text files, images, movies, as well as screen recordings with voice-over in which the user is guided step by step through some of the most important features of SPIKY). Subsequently, in Subsection <span class="k">\ref</span><span class="nb">{</span>ss:Structure<span class="nb">}</span> we introduce the structure and the workflow of the program. We show how to input spike train data, how to change the layout of the figures and how to export results. We also present two programs included in the SPIKY-package which are complementary to the graphical user interface itself. While the program `SPIKY<span class="k">\_</span>loop&#39; (Subsection <span class="k">\ref</span><span class="nb">{</span>ss:GUI-vs-loop<span class="nb">}</span>) is meant to be used for the collective analysis of many different datasets (e.g. when the statistics of a certain quantity should be evaluated over all available datasets), the program `SPIKY<span class="k">\_</span>loop<span class="k">\_</span>surro&#39; (Subsection <span class="k">\ref</span><span class="nb">{</span>ss:Spike-train-surrogates<span class="nb">}</span>) was designed to evaluate the statistical significance of the results obtained for the original dataset by comparing them against the results obtained for spike train surrogates generated from that dataset. In Subsection <span class="k">\ref</span><span class="nb">{</span>ss:Comparison<span class="nb">}</span> we investigate the performance of our new implementation of the time-resolved measures of spike train synchrony and compare it with previously published codes (e.g. <span class="k">\citet</span><span class="nb">{</span>Rusu14<span class="nb">}</span>). Finally, in Section <span class="k">\ref</span><span class="nb">{</span>s:Discussion<span class="nb">}</span> we summarize both the methods and the program, discuss their limitations and present an outlook on future developments.</td>
      </tr>
      <tr>
        <td id="L110" class="blob-line-num js-line-number" data-line-number="110"></td>
        <td id="LC110" class="blob-line-code js-file-line">
</td>
      </tr>
      <tr>
        <td id="L111" class="blob-line-num js-line-number" data-line-number="111"></td>
        <td id="LC111" class="blob-line-code js-file-line"><span class="c">%</span></td>
      </tr>
      <tr>
        <td id="L112" class="blob-line-num js-line-number" data-line-number="112"></td>
        <td id="LC112" class="blob-line-code js-file-line"><span class="c">%</span></td>
      </tr>
      <tr>
        <td id="L113" class="blob-line-num js-line-number" data-line-number="113"></td>
        <td id="LC113" class="blob-line-code js-file-line"><span class="c">% *************************************************************************************</span></td>
      </tr>
      <tr>
        <td id="L114" class="blob-line-num js-line-number" data-line-number="114"></td>
        <td id="LC114" class="blob-line-code js-file-line"><span class="c">% ********************************************************** Measures *****************</span></td>
      </tr>
      <tr>
        <td id="L115" class="blob-line-num js-line-number" data-line-number="115"></td>
        <td id="LC115" class="blob-line-code js-file-line"><span class="c">% *************************************************************************************</span></td>
      </tr>
      <tr>
        <td id="L116" class="blob-line-num js-line-number" data-line-number="116"></td>
        <td id="LC116" class="blob-line-code js-file-line"><span class="c">%</span></td>
      </tr>
      <tr>
        <td id="L117" class="blob-line-num js-line-number" data-line-number="117"></td>
        <td id="LC117" class="blob-line-code js-file-line"><span class="c">%</span></td>
      </tr>
      <tr>
        <td id="L118" class="blob-line-num js-line-number" data-line-number="118"></td>
        <td id="LC118" class="blob-line-code js-file-line"><span class="k">\section</span><span class="nb">{</span><span class="k">\label</span><span class="nb">{</span>s:Measures<span class="nb">}</span> Measures<span class="nb">}</span></td>
      </tr>
      <tr>
        <td id="L119" class="blob-line-num js-line-number" data-line-number="119"></td>
        <td id="LC119" class="blob-line-code js-file-line">
</td>
      </tr>
      <tr>
        <td id="L120" class="blob-line-num js-line-number" data-line-number="120"></td>
        <td id="LC120" class="blob-line-code js-file-line">Both the bivariate ISI- and the bivariate SPIKE-distance rely on instantaneous values in the sense that in a first step the two sequences of discrete spike times are transformed into dissimilarity profiles, <span class="s">$</span><span class="nb">I </span><span class="o">(</span><span class="nb">t</span><span class="o">)</span><span class="s">$</span> and <span class="s">$</span><span class="nb">S </span><span class="o">(</span><span class="nb">t</span><span class="o">)</span><span class="s">$</span>. These dissimilarity profiles are based on three piecewise constant quantities which for each neuron <span class="s">$</span><span class="nb">n </span><span class="o">=</span><span class="nb"> </span><span class="m">1</span><span class="nb">, </span><span class="m">2</span><span class="s">$</span> are assigned to every time instant (see Fig. <span class="k">\ref</span><span class="nb">{</span>fig:Fig1-SPIKE-Illustration<span class="nb">}</span>). These are the time of the preceding spike</td>
      </tr>
      <tr>
        <td id="L121" class="blob-line-num js-line-number" data-line-number="121"></td>
        <td id="LC121" class="blob-line-code js-file-line"><span class="c">%</span></td>
      </tr>
      <tr>
        <td id="L122" class="blob-line-num js-line-number" data-line-number="122"></td>
        <td id="LC122" class="blob-line-code js-file-line"><span class="k">\begin</span><span class="nb">{</span>equation<span class="nb">}</span> <span class="k">\label</span><span class="nb">{</span>eq:Prev-Spike<span class="nb">}</span></td>
      </tr>
      <tr>
        <td id="L123" class="blob-line-num js-line-number" data-line-number="123"></td>
        <td id="LC123" class="blob-line-code js-file-line">    t<span class="nb">_{</span><span class="k">\mathrm</span> <span class="nb">{</span>P<span class="nb">}}^{</span>(n)<span class="nb">}</span> (t) = <span class="k">\max</span>(t<span class="nb">_</span>i<span class="nb">^{</span>(n)<span class="nb">}</span> | t<span class="nb">_</span>i<span class="nb">^{</span>(n)<span class="nb">}</span> <span class="k">\leq</span> t)  <span class="k">\quad</span> t<span class="nb">_</span>1<span class="nb">^{</span>(n)<span class="nb">}</span> <span class="k">\leq</span> t <span class="k">\leq</span> t<span class="nb">_{</span>M<span class="nb">_</span>n<span class="nb">}^{</span>(n)<span class="nb">}</span>,</td>
      </tr>
      <tr>
        <td id="L124" class="blob-line-num js-line-number" data-line-number="124"></td>
        <td id="LC124" class="blob-line-code js-file-line"><span class="k">\end</span><span class="nb">{</span>equation<span class="nb">}</span></td>
      </tr>
      <tr>
        <td id="L125" class="blob-line-num js-line-number" data-line-number="125"></td>
        <td id="LC125" class="blob-line-code js-file-line"><span class="c">%</span></td>
      </tr>
      <tr>
        <td id="L126" class="blob-line-num js-line-number" data-line-number="126"></td>
        <td id="LC126" class="blob-line-code js-file-line">the time of the following spike</td>
      </tr>
      <tr>
        <td id="L127" class="blob-line-num js-line-number" data-line-number="127"></td>
        <td id="LC127" class="blob-line-code js-file-line"><span class="c">%</span></td>
      </tr>
      <tr>
        <td id="L128" class="blob-line-num js-line-number" data-line-number="128"></td>
        <td id="LC128" class="blob-line-code js-file-line"><span class="k">\begin</span><span class="nb">{</span>equation<span class="nb">}</span> <span class="k">\label</span><span class="nb">{</span>eq:Foll-Spike<span class="nb">}</span></td>
      </tr>
      <tr>
        <td id="L129" class="blob-line-num js-line-number" data-line-number="129"></td>
        <td id="LC129" class="blob-line-code js-file-line">    t<span class="nb">_{</span><span class="k">\mathrm</span> <span class="nb">{</span>F<span class="nb">}}^{</span>(n)<span class="nb">}</span> (t) = <span class="k">\min</span>(t<span class="nb">_</span>i<span class="nb">^{</span>(n)<span class="nb">}</span> | t<span class="nb">_</span>i<span class="nb">^{</span>(n)<span class="nb">}</span> &gt; t)  <span class="k">\quad</span> t<span class="nb">_</span>1<span class="nb">^{</span>(n)<span class="nb">}</span> <span class="k">\leq</span> t <span class="k">\leq</span> t<span class="nb">_{</span>M<span class="nb">_</span>n<span class="nb">}^{</span>(n)<span class="nb">}</span>,</td>
      </tr>
      <tr>
        <td id="L130" class="blob-line-num js-line-number" data-line-number="130"></td>
        <td id="LC130" class="blob-line-code js-file-line"><span class="k">\end</span><span class="nb">{</span>equation<span class="nb">}</span></td>
      </tr>
      <tr>
        <td id="L131" class="blob-line-num js-line-number" data-line-number="131"></td>
        <td id="LC131" class="blob-line-code js-file-line"><span class="c">%</span></td>
      </tr>
      <tr>
        <td id="L132" class="blob-line-num js-line-number" data-line-number="132"></td>
        <td id="LC132" class="blob-line-code js-file-line">as well as the interspike interval</td>
      </tr>
      <tr>
        <td id="L133" class="blob-line-num js-line-number" data-line-number="133"></td>
        <td id="LC133" class="blob-line-code js-file-line"><span class="c">%</span></td>
      </tr>
      <tr>
        <td id="L134" class="blob-line-num js-line-number" data-line-number="134"></td>
        <td id="LC134" class="blob-line-code js-file-line"><span class="k">\begin</span><span class="nb">{</span>equation<span class="nb">}</span> <span class="k">\label</span><span class="nb">{</span>eq:ISI<span class="nb">}</span></td>
      </tr>
      <tr>
        <td id="L135" class="blob-line-num js-line-number" data-line-number="135"></td>
        <td id="LC135" class="blob-line-code js-file-line">    x<span class="nb">_{</span><span class="k">\mathrm</span> <span class="nb">{</span>ISI<span class="nb">}}^{</span>(n)<span class="nb">}</span> (t) = t<span class="nb">_{</span><span class="k">\mathrm</span> <span class="nb">{</span>F<span class="nb">}}^{</span>(n)<span class="nb">}</span> (t) - t<span class="nb">_{</span><span class="k">\mathrm</span> <span class="nb">{</span>P<span class="nb">}}^{</span>(n)<span class="nb">}</span> (t).</td>
      </tr>
      <tr>
        <td id="L136" class="blob-line-num js-line-number" data-line-number="136"></td>
        <td id="LC136" class="blob-line-code js-file-line"><span class="k">\end</span><span class="nb">{</span>equation<span class="nb">}</span></td>
      </tr>
      <tr>
        <td id="L137" class="blob-line-num js-line-number" data-line-number="137"></td>
        <td id="LC137" class="blob-line-code js-file-line"><span class="c">%</span></td>
      </tr>
      <tr>
        <td id="L138" class="blob-line-num js-line-number" data-line-number="138"></td>
        <td id="LC138" class="blob-line-code js-file-line">The ambiguity regarding the definition of the very first and the very last interspike interval is resolved by placing for each spike train auxiliary leading spikes at time <span class="s">$</span><span class="nb">t </span><span class="o">=</span><span class="nb"> </span><span class="m">0</span><span class="s">$</span> and auxiliary trailing spikes at time <span class="s">$</span><span class="nb">t </span><span class="o">=</span><span class="nb"> T</span><span class="s">$</span> (but see also Section <span class="k">\ref</span><span class="nb">{</span>sss:Edge-effect<span class="nb">}</span>).</td>
      </tr>
      <tr>
        <td id="L139" class="blob-line-num js-line-number" data-line-number="139"></td>
        <td id="LC139" class="blob-line-code js-file-line">
</td>
      </tr>
      <tr>
        <td id="L140" class="blob-line-num js-line-number" data-line-number="140"></td>
        <td id="LC140" class="blob-line-code js-file-line">
</td>
      </tr>
      <tr>
        <td id="L141" class="blob-line-num js-line-number" data-line-number="141"></td>
        <td id="LC141" class="blob-line-code js-file-line"><span class="k">\subsection</span><span class="nb">{</span><span class="k">\label</span><span class="nb">{</span>ss:ISI-Distance<span class="nb">}</span> The ISI-distance<span class="nb">}</span></td>
      </tr>
      <tr>
        <td id="L142" class="blob-line-num js-line-number" data-line-number="142"></td>
        <td id="LC142" class="blob-line-code js-file-line">
</td>
      </tr>
      <tr>
        <td id="L143" class="blob-line-num js-line-number" data-line-number="143"></td>
        <td id="LC143" class="blob-line-code js-file-line">The ISI-distance, proposed as a bivariate measure in <span class="k">\citep</span><span class="nb">{</span>Kreuz07c<span class="nb">}</span> and extended to the multi-neuron case in <span class="k">\citep</span><span class="nb">{</span>Kreuz09<span class="nb">}</span>, was the first spike train distance directly defined as the temporal average of an instantaneous dissimilarity profile. This profile is calculated as the instantaneous ratio between the interspike intervals <span class="s">$</span><span class="nb">x_{</span><span class="nv">\mathrm</span><span class="nb"> {ISI}}^{</span><span class="o">(</span><span class="m">1</span><span class="o">)</span><span class="nb">}</span><span class="s">$</span> and <span class="s">$</span><span class="nb">x_{</span><span class="nv">\mathrm</span><span class="nb"> {ISI}}^{</span><span class="o">(</span><span class="m">2</span><span class="o">)</span><span class="nb">}</span><span class="s">$</span> (see Eq. <span class="k">\ref</span><span class="nb">{</span>eq:ISI<span class="nb">}</span> and Fig. <span class="k">\ref</span><span class="nb">{</span>fig:Fig1-SPIKE-Illustration<span class="nb">}</span>) according to:</td>
      </tr>
      <tr>
        <td id="L144" class="blob-line-num js-line-number" data-line-number="144"></td>
        <td id="LC144" class="blob-line-code js-file-line"><span class="c">%</span></td>
      </tr>
      <tr>
        <td id="L145" class="blob-line-num js-line-number" data-line-number="145"></td>
        <td id="LC145" class="blob-line-code js-file-line"><span class="k">\begin</span><span class="nb">{</span>equation<span class="nb">}</span> <span class="k">\label</span><span class="nb">{</span>eq:ISI-Ratio<span class="nb">}</span></td>
      </tr>
      <tr>
        <td id="L146" class="blob-line-num js-line-number" data-line-number="146"></td>
        <td id="LC146" class="blob-line-code js-file-line">    I (t) = <span class="k">\begin</span><span class="nb">{</span>cases<span class="nb">}</span></td>
      </tr>
      <tr>
        <td id="L147" class="blob-line-num js-line-number" data-line-number="147"></td>
        <td id="LC147" class="blob-line-code js-file-line">           x<span class="nb">_{</span><span class="k">\mathrm</span> <span class="nb">{</span>ISI<span class="nb">}}^{</span>(1)<span class="nb">}</span> (t) / x<span class="nb">_{</span><span class="k">\mathrm</span> <span class="nb">{</span>ISI<span class="nb">}}^{</span>(2)<span class="nb">}</span> (t) - 1 <span class="nb">&amp;</span> <span class="nb">{</span><span class="k">\rm</span> if<span class="nb">}</span> ~~ x<span class="nb">_{</span><span class="k">\mathrm</span> <span class="nb">{</span>ISI<span class="nb">}}^{</span>(1)<span class="nb">}</span> (t) <span class="k">\leq</span> x<span class="nb">_{</span><span class="k">\mathrm</span> <span class="nb">{</span>ISI<span class="nb">}}^{</span>(2)<span class="nb">}</span> (t) <span class="k">\cr</span></td>
      </tr>
      <tr>
        <td id="L148" class="blob-line-num js-line-number" data-line-number="148"></td>
        <td id="LC148" class="blob-line-code js-file-line">                      - (x<span class="nb">_{</span><span class="k">\mathrm</span> <span class="nb">{</span>ISI<span class="nb">}}^{</span>(2)<span class="nb">}</span> (t) / x<span class="nb">_{</span><span class="k">\mathrm</span> <span class="nb">{</span>ISI<span class="nb">}}^{</span>(1)<span class="nb">}</span> (t) -1)     <span class="nb">&amp;</span> <span class="nb">{</span><span class="k">\rm</span> otherwise<span class="nb">}</span>.</td>
      </tr>
      <tr>
        <td id="L149" class="blob-line-num js-line-number" data-line-number="149"></td>
        <td id="LC149" class="blob-line-code js-file-line">                  <span class="k">\end</span><span class="nb">{</span>cases<span class="nb">}</span></td>
      </tr>
      <tr>
        <td id="L150" class="blob-line-num js-line-number" data-line-number="150"></td>
        <td id="LC150" class="blob-line-code js-file-line"><span class="k">\end</span><span class="nb">{</span>equation<span class="nb">}</span></td>
      </tr>
      <tr>
        <td id="L151" class="blob-line-num js-line-number" data-line-number="151"></td>
        <td id="LC151" class="blob-line-code js-file-line"><span class="c">%</span></td>
      </tr>
      <tr>
        <td id="L152" class="blob-line-num js-line-number" data-line-number="152"></td>
        <td id="LC152" class="blob-line-code js-file-line">Since the ISI-values only change at the times of spikes, the dissimilarity profile is piecewise constant (with potential discontinuities at the spikes). The ISI-ratio equals <span class="s">$</span><span class="m">0</span><span class="s">$</span> for identical ISI in the two spike trains, and approaches <span class="s">$</span><span class="o">-</span><span class="m">1</span><span class="s">$</span> and <span class="s">$</span><span class="m">1</span><span class="s">$</span>, respectively, in intervals in which the first or the second spike train is much faster than the other. In order to treat both kinds of deviations equally, the temporal averaging is performed on the absolute value of the ISI-ratio:</td>
      </tr>
      <tr>
        <td id="L153" class="blob-line-num js-line-number" data-line-number="153"></td>
        <td id="LC153" class="blob-line-code js-file-line"><span class="c">%</span></td>
      </tr>
      <tr>
        <td id="L154" class="blob-line-num js-line-number" data-line-number="154"></td>
        <td id="LC154" class="blob-line-code js-file-line"><span class="k">\begin</span><span class="nb">{</span>equation<span class="nb">}</span> <span class="k">\label</span><span class="nb">{</span>eq:Temporal-Average<span class="nb">}</span></td>
      </tr>
      <tr>
        <td id="L155" class="blob-line-num js-line-number" data-line-number="155"></td>
        <td id="LC155" class="blob-line-code js-file-line">    D<span class="nb">_</span>I = <span class="k">\frac</span><span class="nb">{</span>1<span class="nb">}{</span>T<span class="nb">}</span> <span class="k">\int</span><span class="nb">_{</span>t=0<span class="nb">}^</span>T dt |I (t)|.</td>
      </tr>
      <tr>
        <td id="L156" class="blob-line-num js-line-number" data-line-number="156"></td>
        <td id="LC156" class="blob-line-code js-file-line"><span class="k">\end</span><span class="nb">{</span>equation<span class="nb">}</span></td>
      </tr>
      <tr>
        <td id="L157" class="blob-line-num js-line-number" data-line-number="157"></td>
        <td id="LC157" class="blob-line-code js-file-line">
</td>
      </tr>
      <tr>
        <td id="L158" class="blob-line-num js-line-number" data-line-number="158"></td>
        <td id="LC158" class="blob-line-code js-file-line">
</td>
      </tr>
      <tr>
        <td id="L159" class="blob-line-num js-line-number" data-line-number="159"></td>
        <td id="LC159" class="blob-line-code js-file-line"><span class="k">\subsection</span><span class="nb">{</span><span class="k">\label</span><span class="nb">{</span>ss:SPIKE-Distance<span class="nb">}</span> The SPIKE-distance<span class="nb">}</span></td>
      </tr>
      <tr>
        <td id="L160" class="blob-line-num js-line-number" data-line-number="160"></td>
        <td id="LC160" class="blob-line-code js-file-line">
</td>
      </tr>
      <tr>
        <td id="L161" class="blob-line-num js-line-number" data-line-number="161"></td>
        <td id="LC161" class="blob-line-code js-file-line">In contrast to the ISI-distance, the SPIKE-distance (see <span class="k">\citet</span><span class="nb">{</span>Kreuz11<span class="nb">}</span> for the original proposal and <span class="k">\citet</span><span class="nb">{</span>Kreuz13<span class="nb">}</span> for the definite version presented here) considers the exact timing of the spikes since it is extracted from differences between the spike times of the two spike trains. </td>
      </tr>
      <tr>
        <td id="L162" class="blob-line-num js-line-number" data-line-number="162"></td>
        <td id="LC162" class="blob-line-code js-file-line"><span class="c">%</span></td>
      </tr>
      <tr>
        <td id="L163" class="blob-line-num js-line-number" data-line-number="163"></td>
        <td id="LC163" class="blob-line-code js-file-line"><span class="c">% #########################################################################################</span></td>
      </tr>
      <tr>
        <td id="L164" class="blob-line-num js-line-number" data-line-number="164"></td>
        <td id="LC164" class="blob-line-code js-file-line"><span class="c">% #########################################################################################</span></td>
      </tr>
      <tr>
        <td id="L165" class="blob-line-num js-line-number" data-line-number="165"></td>
        <td id="LC165" class="blob-line-code js-file-line"><span class="c">% #################################################### Figure 1 ###########################</span></td>
      </tr>
      <tr>
        <td id="L166" class="blob-line-num js-line-number" data-line-number="166"></td>
        <td id="LC166" class="blob-line-code js-file-line"><span class="c">% #########################################################################################</span></td>
      </tr>
      <tr>
        <td id="L167" class="blob-line-num js-line-number" data-line-number="167"></td>
        <td id="LC167" class="blob-line-code js-file-line"><span class="c">% #########################################################################################</span></td>
      </tr>
      <tr>
        <td id="L168" class="blob-line-num js-line-number" data-line-number="168"></td>
        <td id="LC168" class="blob-line-code js-file-line"><span class="c">%</span></td>
      </tr>
      <tr>
        <td id="L169" class="blob-line-num js-line-number" data-line-number="169"></td>
        <td id="LC169" class="blob-line-code js-file-line"><span class="k">\begin</span><span class="nb">{</span>figure<span class="nb">}</span></td>
      </tr>
      <tr>
        <td id="L170" class="blob-line-num js-line-number" data-line-number="170"></td>
        <td id="LC170" class="blob-line-code js-file-line">    <span class="k">\includegraphics</span><span class="na">[width=85mm]</span><span class="nb">{</span>Fig1<span class="nb">_</span>SPIKE<span class="nb">_</span>Illustration.eps<span class="nb">}</span></td>
      </tr>
      <tr>
        <td id="L171" class="blob-line-num js-line-number" data-line-number="171"></td>
        <td id="LC171" class="blob-line-code js-file-line">    <span class="k">\caption</span><span class="nb">{</span><span class="k">\abb\label</span><span class="nb">{</span>fig:Fig1-SPIKE-Illustration<span class="nb">}</span> SPIKE-distance. Illustration of the local</td>
      </tr>
      <tr>
        <td id="L172" class="blob-line-num js-line-number" data-line-number="172"></td>
        <td id="LC172" class="blob-line-code js-file-line">	    quantities needed to define the dissimilarity profile <span class="s">$</span><span class="nb">S </span><span class="o">(</span><span class="nb">t</span><span class="o">)</span><span class="s">$</span> for an arbitrary time </td>
      </tr>
      <tr>
        <td id="L173" class="blob-line-num js-line-number" data-line-number="173"></td>
        <td id="LC173" class="blob-line-code js-file-line">    		instant <span class="s">$</span><span class="nb">t</span><span class="s">$</span>.<span class="nb">}</span></td>
      </tr>
      <tr>
        <td id="L174" class="blob-line-num js-line-number" data-line-number="174"></td>
        <td id="LC174" class="blob-line-code js-file-line"><span class="k">\end</span><span class="nb">{</span>figure<span class="nb">}</span></td>
      </tr>
      <tr>
        <td id="L175" class="blob-line-num js-line-number" data-line-number="175"></td>
        <td id="LC175" class="blob-line-code js-file-line"><span class="c">%</span></td>
      </tr>
      <tr>
        <td id="L176" class="blob-line-num js-line-number" data-line-number="176"></td>
        <td id="LC176" class="blob-line-code js-file-line"><span class="c">% #########################################################################################</span></td>
      </tr>
      <tr>
        <td id="L177" class="blob-line-num js-line-number" data-line-number="177"></td>
        <td id="LC177" class="blob-line-code js-file-line"><span class="c">% #########################################################################################</span></td>
      </tr>
      <tr>
        <td id="L178" class="blob-line-num js-line-number" data-line-number="178"></td>
        <td id="LC178" class="blob-line-code js-file-line"><span class="c">% #################################################### Figure 1 ###########################</span></td>
      </tr>
      <tr>
        <td id="L179" class="blob-line-num js-line-number" data-line-number="179"></td>
        <td id="LC179" class="blob-line-code js-file-line"><span class="c">% #########################################################################################</span></td>
      </tr>
      <tr>
        <td id="L180" class="blob-line-num js-line-number" data-line-number="180"></td>
        <td id="LC180" class="blob-line-code js-file-line"><span class="c">% #########################################################################################</span></td>
      </tr>
      <tr>
        <td id="L181" class="blob-line-num js-line-number" data-line-number="181"></td>
        <td id="LC181" class="blob-line-code js-file-line">
</td>
      </tr>
      <tr>
        <td id="L182" class="blob-line-num js-line-number" data-line-number="182"></td>
        <td id="LC182" class="blob-line-code js-file-line">The dissimilarity profile is calculated in two steps: First for each spike the distance to the nearest spike in the other spike train is calculated, then for each time instant the relevant spike time differences are selected, weighted, and normalized. Here `relevant&#39; means local; each time instant is uniquely surrounded by four corner spikes: the preceding spike from the first spike train <span class="s">$</span><span class="nb">t_{</span><span class="nv">\mathrm</span><span class="nb"> {P}}^{</span><span class="o">(</span><span class="m">1</span><span class="o">)</span><span class="nb">}</span><span class="s">$</span>, the following spike from the first spike train <span class="s">$</span><span class="nb">t_{</span><span class="nv">\mathrm</span><span class="nb"> {F}}^{</span><span class="o">(</span><span class="m">1</span><span class="o">)</span><span class="nb">}</span><span class="s">$</span>, the preceding spike from the second spike train <span class="s">$</span><span class="nb">t_{</span><span class="nv">\mathrm</span><span class="nb"> {P}}^{</span><span class="o">(</span><span class="m">2</span><span class="o">)</span><span class="nb">}</span><span class="s">$</span>, and, finally, the following spike from the second spike train <span class="s">$</span><span class="nb">t_{</span><span class="nv">\mathrm</span><span class="nb"> {F}}^{</span><span class="o">(</span><span class="m">2</span><span class="o">)</span><span class="nb">}</span><span class="s">$</span>. Each of these corner spikes can be identified with a spike time difference, for example, for the previous spike of the first spike train</td>
      </tr>
      <tr>
        <td id="L183" class="blob-line-num js-line-number" data-line-number="183"></td>
        <td id="LC183" class="blob-line-code js-file-line"><span class="c">%</span></td>
      </tr>
      <tr>
        <td id="L184" class="blob-line-num js-line-number" data-line-number="184"></td>
        <td id="LC184" class="blob-line-code js-file-line"><span class="k">\begin</span><span class="nb">{</span>equation<span class="nb">}</span> <span class="k">\label</span><span class="nb">{</span>eq:Delta-Corner-Spike<span class="nb">}</span></td>
      </tr>
      <tr>
        <td id="L185" class="blob-line-num js-line-number" data-line-number="185"></td>
        <td id="LC185" class="blob-line-code js-file-line">     <span class="k">\Delta</span> t<span class="nb">_{</span><span class="k">\mathrm</span> <span class="nb">{</span>P<span class="nb">}}^{</span>(1)<span class="nb">}</span> = <span class="k">\min</span><span class="nb">_</span>i (| t<span class="nb">_{</span><span class="k">\mathrm</span> <span class="nb">{</span>P<span class="nb">}}^{</span>(1)<span class="nb">}</span> - t<span class="nb">_</span>i<span class="nb">^{</span>(2)<span class="nb">}</span> |)</td>
      </tr>
      <tr>
        <td id="L186" class="blob-line-num js-line-number" data-line-number="186"></td>
        <td id="LC186" class="blob-line-code js-file-line"><span class="k">\end</span><span class="nb">{</span>equation<span class="nb">}</span></td>
      </tr>
      <tr>
        <td id="L187" class="blob-line-num js-line-number" data-line-number="187"></td>
        <td id="LC187" class="blob-line-code js-file-line"><span class="c">%</span></td>
      </tr>
      <tr>
        <td id="L188" class="blob-line-num js-line-number" data-line-number="188"></td>
        <td id="LC188" class="blob-line-code js-file-line">and analogously for <span class="s">$</span><span class="nb">t_{</span><span class="nv">\mathrm</span><span class="nb"> {F}}^{</span><span class="o">(</span><span class="m">1</span><span class="o">)</span><span class="nb">}</span><span class="s">$</span>, <span class="s">$</span><span class="nb">t_{</span><span class="nv">\mathrm</span><span class="nb"> {P}}^{</span><span class="o">(</span><span class="m">2</span><span class="o">)</span><span class="nb">}</span><span class="s">$</span>, and <span class="s">$</span><span class="nb">t_{</span><span class="nv">\mathrm</span><span class="nb"> {F}}^{</span><span class="o">(</span><span class="m">2</span><span class="o">)</span><span class="nb">}</span><span class="s">$</span>  (see Fig. <span class="k">\ref</span><span class="nb">{</span>fig:Fig1-SPIKE-Illustration<span class="nb">}</span>). For each spike train separately a locally weighted average is employed such that the differences for the closer spike dominate; the weighting factors depend on</td>
      </tr>
      <tr>
        <td id="L189" class="blob-line-num js-line-number" data-line-number="189"></td>
        <td id="LC189" class="blob-line-code js-file-line"><span class="c">%</span></td>
      </tr>
      <tr>
        <td id="L190" class="blob-line-num js-line-number" data-line-number="190"></td>
        <td id="LC190" class="blob-line-code js-file-line"><span class="k">\begin</span><span class="nb">{</span>equation<span class="nb">}</span> <span class="k">\label</span><span class="nb">{</span>eq:Prev-Spike-Dist<span class="nb">}</span></td>
      </tr>
      <tr>
        <td id="L191" class="blob-line-num js-line-number" data-line-number="191"></td>
        <td id="LC191" class="blob-line-code js-file-line">     x<span class="nb">_{</span><span class="k">\mathrm</span> <span class="nb">{</span>P<span class="nb">}}^{</span>(n)<span class="nb">}</span> (t) = t - t<span class="nb">_{</span><span class="k">\mathrm</span> <span class="nb">{</span>P<span class="nb">}}^{</span>(n)<span class="nb">}</span> (t)</td>
      </tr>
      <tr>
        <td id="L192" class="blob-line-num js-line-number" data-line-number="192"></td>
        <td id="LC192" class="blob-line-code js-file-line"><span class="k">\end</span><span class="nb">{</span>equation<span class="nb">}</span></td>
      </tr>
      <tr>
        <td id="L193" class="blob-line-num js-line-number" data-line-number="193"></td>
        <td id="LC193" class="blob-line-code js-file-line"><span class="c">%</span></td>
      </tr>
      <tr>
        <td id="L194" class="blob-line-num js-line-number" data-line-number="194"></td>
        <td id="LC194" class="blob-line-code js-file-line">and</td>
      </tr>
      <tr>
        <td id="L195" class="blob-line-num js-line-number" data-line-number="195"></td>
        <td id="LC195" class="blob-line-code js-file-line"><span class="c">%</span></td>
      </tr>
      <tr>
        <td id="L196" class="blob-line-num js-line-number" data-line-number="196"></td>
        <td id="LC196" class="blob-line-code js-file-line"><span class="k">\begin</span><span class="nb">{</span>equation<span class="nb">}</span> <span class="k">\label</span><span class="nb">{</span>eq:Foll-Spike-Dist<span class="nb">}</span></td>
      </tr>
      <tr>
        <td id="L197" class="blob-line-num js-line-number" data-line-number="197"></td>
        <td id="LC197" class="blob-line-code js-file-line">     x<span class="nb">_{</span><span class="k">\mathrm</span> <span class="nb">{</span>F<span class="nb">}}^{</span>(n)<span class="nb">}</span> (t) = t<span class="nb">_{</span><span class="k">\mathrm</span> <span class="nb">{</span>F<span class="nb">}}^{</span>(n)<span class="nb">}</span> (t) - t,</td>
      </tr>
      <tr>
        <td id="L198" class="blob-line-num js-line-number" data-line-number="198"></td>
        <td id="LC198" class="blob-line-code js-file-line"><span class="k">\end</span><span class="nb">{</span>equation<span class="nb">}</span></td>
      </tr>
      <tr>
        <td id="L199" class="blob-line-num js-line-number" data-line-number="199"></td>
        <td id="LC199" class="blob-line-code js-file-line"><span class="c">%</span></td>
      </tr>
      <tr>
        <td id="L200" class="blob-line-num js-line-number" data-line-number="200"></td>
        <td id="LC200" class="blob-line-code js-file-line">the intervals to the previous and the following spikes for each neuron <span class="s">$</span><span class="nb">n </span><span class="o">=</span><span class="nb"> </span><span class="m">1</span><span class="nb">, </span><span class="m">2</span><span class="s">$</span>. The local weighting for the spike time differences of the first spike train reads</td>
      </tr>
      <tr>
        <td id="L201" class="blob-line-num js-line-number" data-line-number="201"></td>
        <td id="LC201" class="blob-line-code js-file-line"><span class="c">%</span></td>
      </tr>
      <tr>
        <td id="L202" class="blob-line-num js-line-number" data-line-number="202"></td>
        <td id="LC202" class="blob-line-code js-file-line"><span class="k">\begin</span><span class="nb">{</span>equation<span class="nb">}</span> <span class="k">\label</span><span class="nb">{</span>eq:Bi-Spike-Diss-First<span class="nb">}</span></td>
      </tr>
      <tr>
        <td id="L203" class="blob-line-num js-line-number" data-line-number="203"></td>
        <td id="LC203" class="blob-line-code js-file-line">     S<span class="nb">_</span>1 (t) = <span class="k">\frac</span><span class="nb">{</span><span class="k">\Delta</span> t<span class="nb">_{</span><span class="k">\mathrm</span> <span class="nb">{</span>P<span class="nb">}}^{</span>(1)<span class="nb">}</span> x<span class="nb">_{</span><span class="k">\mathrm</span> <span class="nb">{</span>F<span class="nb">}}^{</span>(1)<span class="nb">}</span> + <span class="k">\Delta</span> t<span class="nb">_{</span><span class="k">\mathrm</span> <span class="nb">{</span>F<span class="nb">}}^{</span>(1)<span class="nb">}</span> x<span class="nb">_{</span><span class="k">\mathrm</span> <span class="nb">{</span>P<span class="nb">}}^{</span>(1)<span class="nb">}}{</span>x<span class="nb">_{</span><span class="k">\mathrm</span> <span class="nb">{</span>ISI<span class="nb">}}^{</span>(1)<span class="nb">}}</span></td>
      </tr>
      <tr>
        <td id="L204" class="blob-line-num js-line-number" data-line-number="204"></td>
        <td id="LC204" class="blob-line-code js-file-line"><span class="k">\end</span><span class="nb">{</span>equation<span class="nb">}</span></td>
      </tr>
      <tr>
        <td id="L205" class="blob-line-num js-line-number" data-line-number="205"></td>
        <td id="LC205" class="blob-line-code js-file-line"><span class="c">%</span></td>
      </tr>
      <tr>
        <td id="L206" class="blob-line-num js-line-number" data-line-number="206"></td>
        <td id="LC206" class="blob-line-code js-file-line">and analogously <span class="s">$</span><span class="nb">S_</span><span class="m">2</span><span class="nb"> </span><span class="o">(</span><span class="nb">t</span><span class="o">)</span><span class="s">$</span> is obtained for the second spike train. Averaging over the two spike train contributions and normalizing by the mean interspike interval yields</td>
      </tr>
      <tr>
        <td id="L207" class="blob-line-num js-line-number" data-line-number="207"></td>
        <td id="LC207" class="blob-line-code js-file-line"><span class="c">%</span></td>
      </tr>
      <tr>
        <td id="L208" class="blob-line-num js-line-number" data-line-number="208"></td>
        <td id="LC208" class="blob-line-code js-file-line"><span class="k">\begin</span><span class="nb">{</span>equation<span class="nb">}</span> <span class="k">\label</span><span class="nb">{</span>eq:Bi-Spike-Diss-Intermediate<span class="nb">}</span></td>
      </tr>
      <tr>
        <td id="L209" class="blob-line-num js-line-number" data-line-number="209"></td>
        <td id="LC209" class="blob-line-code js-file-line">     S&#39;&#39; (t) = <span class="k">\frac</span><span class="nb">{</span>S<span class="nb">_</span>1 (t) + S<span class="nb">_</span>2 (t)<span class="nb">}{</span>2 <span class="k">\langle</span> x<span class="nb">_{</span><span class="k">\mathrm</span> <span class="nb">{</span>ISI<span class="nb">}}^{</span>(n)<span class="nb">}</span> <span class="k">\rangle</span><span class="nb">_</span>n<span class="nb">}</span>.</td>
      </tr>
      <tr>
        <td id="L210" class="blob-line-num js-line-number" data-line-number="210"></td>
        <td id="LC210" class="blob-line-code js-file-line"><span class="k">\end</span><span class="nb">{</span>equation<span class="nb">}</span></td>
      </tr>
      <tr>
        <td id="L211" class="blob-line-num js-line-number" data-line-number="211"></td>
        <td id="LC211" class="blob-line-code js-file-line">
</td>
      </tr>
      <tr>
        <td id="L212" class="blob-line-num js-line-number" data-line-number="212"></td>
        <td id="LC212" class="blob-line-code js-file-line">This quantity weights the spike time differences for each spike train according to the relative distance of the corner spike from the time instant under investigation. This way relative distances within each spike train are taken care of, while relative distances between spike trains are not. In order to get these ratios straight and to account for differences in firing rate, in a last step the two contributions from the two spike trains are locally weighted by their instantaneous interspike intervals. This leads to the definition of the dissimilarity profile</td>
      </tr>
      <tr>
        <td id="L213" class="blob-line-num js-line-number" data-line-number="213"></td>
        <td id="LC213" class="blob-line-code js-file-line"><span class="c">%</span></td>
      </tr>
      <tr>
        <td id="L214" class="blob-line-num js-line-number" data-line-number="214"></td>
        <td id="LC214" class="blob-line-code js-file-line"><span class="k">\begin</span><span class="nb">{</span>equation<span class="nb">}</span> <span class="k">\label</span><span class="nb">{</span>eq:Bi-Spike-Diss-Improved<span class="nb">}</span></td>
      </tr>
      <tr>
        <td id="L215" class="blob-line-num js-line-number" data-line-number="215"></td>
        <td id="LC215" class="blob-line-code js-file-line">     S (t) = <span class="k">\frac</span><span class="nb">{</span>S<span class="nb">_</span>1 (t) x<span class="nb">_{</span><span class="k">\mathrm</span> <span class="nb">{</span>ISI<span class="nb">}}^{</span>(2)<span class="nb">}</span> + S<span class="nb">_</span>2 (t) x<span class="nb">_{</span><span class="k">\mathrm</span> <span class="nb">{</span>ISI<span class="nb">}}^{</span>(1)<span class="nb">}}{</span>2 <span class="k">\langle</span> x<span class="nb">_{</span><span class="k">\mathrm</span> <span class="nb">{</span>ISI<span class="nb">}}^{</span>(n)<span class="nb">}</span> <span class="k">\rangle</span><span class="nb">_</span>n<span class="nb">^</span>2<span class="nb">}</span>.</td>
      </tr>
      <tr>
        <td id="L216" class="blob-line-num js-line-number" data-line-number="216"></td>
        <td id="LC216" class="blob-line-code js-file-line"><span class="k">\end</span><span class="nb">{</span>equation<span class="nb">}</span></td>
      </tr>
      <tr>
        <td id="L217" class="blob-line-num js-line-number" data-line-number="217"></td>
        <td id="LC217" class="blob-line-code js-file-line"><span class="c">%</span></td>
      </tr>
      <tr>
        <td id="L218" class="blob-line-num js-line-number" data-line-number="218"></td>
        <td id="LC218" class="blob-line-code js-file-line">Since the dissimilarity profile is obtained from a linear interpolation of piecewise constant quantities, it is piecewise linear (with potential discontinuities at the spikes).</td>
      </tr>
      <tr>
        <td id="L219" class="blob-line-num js-line-number" data-line-number="219"></td>
        <td id="LC219" class="blob-line-code js-file-line">
</td>
      </tr>
      <tr>
        <td id="L220" class="blob-line-num js-line-number" data-line-number="220"></td>
        <td id="LC220" class="blob-line-code js-file-line">Similar to the ISI-distance, the SPIKE-distance is defined as the temporal average of this dissimilarity profile:</td>
      </tr>
      <tr>
        <td id="L221" class="blob-line-num js-line-number" data-line-number="221"></td>
        <td id="LC221" class="blob-line-code js-file-line"><span class="c">%</span></td>
      </tr>
      <tr>
        <td id="L222" class="blob-line-num js-line-number" data-line-number="222"></td>
        <td id="LC222" class="blob-line-code js-file-line"><span class="k">\begin</span><span class="nb">{</span>equation<span class="nb">}</span> <span class="k">\label</span><span class="nb">{</span>eq:Temporal-Average2<span class="nb">}</span></td>
      </tr>
      <tr>
        <td id="L223" class="blob-line-num js-line-number" data-line-number="223"></td>
        <td id="LC223" class="blob-line-code js-file-line">    D<span class="nb">_</span>S = <span class="k">\frac</span><span class="nb">{</span>1<span class="nb">}{</span>T<span class="nb">}</span> <span class="k">\int</span><span class="nb">_{</span>t=0<span class="nb">}^</span>T dt S (t).</td>
      </tr>
      <tr>
        <td id="L224" class="blob-line-num js-line-number" data-line-number="224"></td>
        <td id="LC224" class="blob-line-code js-file-line"><span class="k">\end</span><span class="nb">{</span>equation<span class="nb">}</span></td>
      </tr>
      <tr>
        <td id="L225" class="blob-line-num js-line-number" data-line-number="225"></td>
        <td id="LC225" class="blob-line-code js-file-line"><span class="c">%</span></td>
      </tr>
      <tr>
        <td id="L226" class="blob-line-num js-line-number" data-line-number="226"></td>
        <td id="LC226" class="blob-line-code js-file-line">The dissimilarity profile <span class="s">$</span><span class="nb">S </span><span class="o">(</span><span class="nb">t</span><span class="o">)</span><span class="s">$</span> and the SPIKE-distance <span class="s">$</span><span class="nb">D_S</span><span class="s">$</span> as its average are bounded in the interval <span class="s">$</span><span class="o">[</span><span class="m">0</span><span class="nb">, </span><span class="m">1</span><span class="o">]</span><span class="s">$</span>. The distance value <span class="s">$</span><span class="nb">D_S </span><span class="o">=</span><span class="nb"> </span><span class="m">0</span><span class="s">$</span> is obtained for identical spike trains only.</td>
      </tr>
      <tr>
        <td id="L227" class="blob-line-num js-line-number" data-line-number="227"></td>
        <td id="LC227" class="blob-line-code js-file-line">
</td>
      </tr>
      <tr>
        <td id="L228" class="blob-line-num js-line-number" data-line-number="228"></td>
        <td id="LC228" class="blob-line-code js-file-line">	</td>
      </tr>
      <tr>
        <td id="L229" class="blob-line-num js-line-number" data-line-number="229"></td>
        <td id="LC229" class="blob-line-code js-file-line">For both the ISI- and the SPIKE-distance there exists a straightforward extension to the case <span class="s">$</span><span class="nb">N &gt; </span><span class="m">2</span><span class="s">$</span> (with <span class="s">$</span><span class="nb">N</span><span class="s">$</span> denoting the number of spike trains), the averaged bivariate distance. This average over all pairs of neurons commutes with the average over time, so it is possible to achieve the same kind of time-resolved visualization as in the bivariate case by first calculating the instantaneous average <span class="s">$</span><span class="nb">S^{</span><span class="nv">\mathrm</span><span class="nb"> {a}} </span><span class="o">(</span><span class="nb">t</span><span class="o">)</span><span class="s">$</span> (for the SPIKE-distance) over all pairwise instantaneous values <span class="s">$</span><span class="nb">S^{mn} </span><span class="o">(</span><span class="nb">t</span><span class="o">)</span><span class="s">$</span>,</td>
      </tr>
      <tr>
        <td id="L230" class="blob-line-num js-line-number" data-line-number="230"></td>
        <td id="LC230" class="blob-line-code js-file-line"><span class="c">%</span></td>
      </tr>
      <tr>
        <td id="L231" class="blob-line-num js-line-number" data-line-number="231"></td>
        <td id="LC231" class="blob-line-code js-file-line"><span class="k">\begin</span><span class="nb">{</span>equation<span class="nb">}</span> <span class="k">\label</span><span class="nb">{</span>eq:Bivariate-Average<span class="nb">}</span></td>
      </tr>
      <tr>
        <td id="L232" class="blob-line-num js-line-number" data-line-number="232"></td>
        <td id="LC232" class="blob-line-code js-file-line">    S<span class="nb">^{</span><span class="k">\mathrm</span> <span class="nb">{</span>a<span class="nb">}}</span> (t) = <span class="k">\frac</span><span class="nb">{</span>1<span class="nb">}{</span>N(N-1)/2<span class="nb">}</span><span class="k">\sum</span><span class="nb">_{</span>n=1<span class="nb">}^{</span>N-1<span class="nb">}</span> <span class="k">\sum</span><span class="nb">_{</span>m=n+1<span class="nb">}^</span>N S<span class="nb">^{</span>mn<span class="nb">}</span> (t).</td>
      </tr>
      <tr>
        <td id="L233" class="blob-line-num js-line-number" data-line-number="233"></td>
        <td id="LC233" class="blob-line-code js-file-line"><span class="k">\end</span><span class="nb">{</span>equation<span class="nb">}</span></td>
      </tr>
      <tr>
        <td id="L234" class="blob-line-num js-line-number" data-line-number="234"></td>
        <td id="LC234" class="blob-line-code js-file-line">
</td>
      </tr>
      <tr>
        <td id="L235" class="blob-line-num js-line-number" data-line-number="235"></td>
        <td id="LC235" class="blob-line-code js-file-line">
</td>
      </tr>
      <tr>
        <td id="L236" class="blob-line-num js-line-number" data-line-number="236"></td>
        <td id="LC236" class="blob-line-code js-file-line">
</td>
      </tr>
      <tr>
        <td id="L237" class="blob-line-num js-line-number" data-line-number="237"></td>
        <td id="LC237" class="blob-line-code js-file-line"><span class="k">\subsubsection</span><span class="nb">{</span><span class="k">\label</span><span class="nb">{</span>sss:Realtime-Spike-Distance<span class="nb">}</span> Realtime SPIKE-distance<span class="nb">}</span></td>
      </tr>
      <tr>
        <td id="L238" class="blob-line-num js-line-number" data-line-number="238"></td>
        <td id="LC238" class="blob-line-code js-file-line">
</td>
      </tr>
      <tr>
        <td id="L239" class="blob-line-num js-line-number" data-line-number="239"></td>
        <td id="LC239" class="blob-line-code js-file-line">The realtime SPIKE-distance <span class="s">$</span><span class="nb">D_{S_r}</span><span class="s">$</span> is a modification of the SPIKE-distance with the key difference that the corresponding time profile <span class="s">$</span><span class="nb">S_r</span><span class="o">(</span><span class="nb">t</span><span class="o">)</span><span class="s">$</span> can be calculated online because it relies on past information only <span class="k">\footnote</span><span class="nb">{</span>For the ISI-distance no such causal realtime extension is possible, since it relies on the instantaneous ISI-values and thus requires knowledge about the following spikes.<span class="nb">}</span>. From the perspective of an online measure, the information provided by the following spikes, both their position and the length of the interspike interval, is not yet available. Like the regular (improved) SPIKE-distance <span class="s">$</span><span class="nb">D_S</span><span class="s">$</span>, this causal variant is also based on local spike time differences but now only two corner spikes are available, and the spikes of comparison are restricted to past spikes, e.g., for the preceding spike of the first spike train</td>
      </tr>
      <tr>
        <td id="L240" class="blob-line-num js-line-number" data-line-number="240"></td>
        <td id="LC240" class="blob-line-code js-file-line"><span class="c">%</span></td>
      </tr>
      <tr>
        <td id="L241" class="blob-line-num js-line-number" data-line-number="241"></td>
        <td id="LC241" class="blob-line-code js-file-line"><span class="k">\begin</span><span class="nb">{</span>equation<span class="nb">}</span> <span class="k">\label</span><span class="nb">{</span>eq:Delta-Corner-Spike-Realtime<span class="nb">}</span></td>
      </tr>
      <tr>
        <td id="L242" class="blob-line-num js-line-number" data-line-number="242"></td>
        <td id="LC242" class="blob-line-code js-file-line">     <span class="k">\Delta</span> t<span class="nb">_{</span><span class="k">\mathrm</span> <span class="nb">{</span>P<span class="nb">}}^{</span>(1)<span class="nb">}</span> = <span class="k">\min</span><span class="nb">_</span>i (| t<span class="nb">_{</span><span class="k">\mathrm</span> <span class="nb">{</span>P<span class="nb">}}^{</span>(1)<span class="nb">}</span> - t<span class="nb">_</span>i<span class="nb">^{</span>(2)<span class="nb">}</span> |), t<span class="nb">_</span>i &lt; t.</td>
      </tr>
      <tr>
        <td id="L243" class="blob-line-num js-line-number" data-line-number="243"></td>
        <td id="LC243" class="blob-line-code js-file-line"><span class="k">\end</span><span class="nb">{</span>equation<span class="nb">}</span></td>
      </tr>
      <tr>
        <td id="L244" class="blob-line-num js-line-number" data-line-number="244"></td>
        <td id="LC244" class="blob-line-code js-file-line"><span class="c">%</span></td>
      </tr>
      <tr>
        <td id="L245" class="blob-line-num js-line-number" data-line-number="245"></td>
        <td id="LC245" class="blob-line-code js-file-line">Since there are no following spikes available, there is no local weighting, and since there is no interspike interval, the normalization is achieved by dividing the average corner spike difference by twice the average time interval to the preceding spikes (Eq. <span class="k">\ref</span><span class="nb">{</span>eq:Prev-Spike-Dist<span class="nb">}</span>). This yields a causal indicator of local spike train dissimilarity:</td>
      </tr>
      <tr>
        <td id="L246" class="blob-line-num js-line-number" data-line-number="246"></td>
        <td id="LC246" class="blob-line-code js-file-line"><span class="c">%</span></td>
      </tr>
      <tr>
        <td id="L247" class="blob-line-num js-line-number" data-line-number="247"></td>
        <td id="LC247" class="blob-line-code js-file-line"><span class="k">\begin</span><span class="nb">{</span>equation<span class="nb">}</span> <span class="k">\label</span><span class="nb">{</span>eq:Bi-Spike-Diss-RT<span class="nb">}</span></td>
      </tr>
      <tr>
        <td id="L248" class="blob-line-num js-line-number" data-line-number="248"></td>
        <td id="LC248" class="blob-line-code js-file-line">    S<span class="nb">_</span>r (t) = <span class="k">\frac</span><span class="nb">{</span> <span class="k">\Delta</span> t<span class="nb">_{</span><span class="k">\mathrm</span> <span class="nb">{</span>P<span class="nb">}}^{</span>(1)<span class="nb">}</span> + <span class="k">\Delta</span> t<span class="nb">_{</span><span class="k">\mathrm</span> <span class="nb">{</span>P<span class="nb">}}^{</span>(2)<span class="nb">}}</span> <span class="nb">{</span>4 <span class="k">\langle</span> x<span class="nb">_{</span><span class="k">\mathrm</span> <span class="nb">{</span>P<span class="nb">}}^{</span>(n)<span class="nb">}</span> <span class="k">\rangle</span><span class="nb">_</span>n<span class="nb">}</span>.</td>
      </tr>
      <tr>
        <td id="L249" class="blob-line-num js-line-number" data-line-number="249"></td>
        <td id="LC249" class="blob-line-code js-file-line"><span class="k">\end</span><span class="nb">{</span>equation<span class="nb">}</span></td>
      </tr>
      <tr>
        <td id="L250" class="blob-line-num js-line-number" data-line-number="250"></td>
        <td id="LC250" class="blob-line-code js-file-line">
</td>
      </tr>
      <tr>
        <td id="L251" class="blob-line-num js-line-number" data-line-number="251"></td>
        <td id="LC251" class="blob-line-code js-file-line">
</td>
      </tr>
      <tr>
        <td id="L252" class="blob-line-num js-line-number" data-line-number="252"></td>
        <td id="LC252" class="blob-line-code js-file-line"><span class="k">\subsubsection</span><span class="nb">{</span><span class="k">\label</span><span class="nb">{</span>sss:Future-Spike-Distance<span class="nb">}</span> Future SPIKE-distance<span class="nb">}</span></td>
      </tr>
      <tr>
        <td id="L253" class="blob-line-num js-line-number" data-line-number="253"></td>
        <td id="LC253" class="blob-line-code js-file-line">
</td>
      </tr>
      <tr>
        <td id="L254" class="blob-line-num js-line-number" data-line-number="254"></td>
        <td id="LC254" class="blob-line-code js-file-line">The future SPIKE-distance <span class="s">$</span><span class="nb">D_{S_f}</span><span class="s">$</span> can be used in triggered temporal averaging in order to evaluate the effect of certain spikes or of certain stimuli features on future spiking. It is the inverse measure to the realtime SPIKE-distance but instead of relying on past information only it relies on future information only. Again for each time instant there are just two corner spikes and the potential nearest spikes in the other spike train are future spikes only. Thus the spike time difference for the following spike of the first spike train reads</td>
      </tr>
      <tr>
        <td id="L255" class="blob-line-num js-line-number" data-line-number="255"></td>
        <td id="LC255" class="blob-line-code js-file-line"><span class="c">%</span></td>
      </tr>
      <tr>
        <td id="L256" class="blob-line-num js-line-number" data-line-number="256"></td>
        <td id="LC256" class="blob-line-code js-file-line"><span class="k">\begin</span><span class="nb">{</span>equation<span class="nb">}</span> <span class="k">\label</span><span class="nb">{</span>eq:Delta-Corner-Spike-Future<span class="nb">}</span></td>
      </tr>
      <tr>
        <td id="L257" class="blob-line-num js-line-number" data-line-number="257"></td>
        <td id="LC257" class="blob-line-code js-file-line">     <span class="k">\Delta</span> t<span class="nb">_{</span><span class="k">\mathrm</span> <span class="nb">{</span>F<span class="nb">}}^{</span>(1)<span class="nb">}</span> = <span class="k">\min</span><span class="nb">_</span>i (| t<span class="nb">_{</span><span class="k">\mathrm</span> <span class="nb">{</span>F<span class="nb">}}^{</span>(1)<span class="nb">}</span> - t<span class="nb">_</span>i<span class="nb">^{</span>(2)<span class="nb">}</span> |), t<span class="nb">_</span>i &gt; t,</td>
      </tr>
      <tr>
        <td id="L258" class="blob-line-num js-line-number" data-line-number="258"></td>
        <td id="LC258" class="blob-line-code js-file-line"><span class="k">\end</span><span class="nb">{</span>equation<span class="nb">}</span></td>
      </tr>
      <tr>
        <td id="L259" class="blob-line-num js-line-number" data-line-number="259"></td>
        <td id="LC259" class="blob-line-code js-file-line"><span class="c">%</span></td>
      </tr>
      <tr>
        <td id="L260" class="blob-line-num js-line-number" data-line-number="260"></td>
        <td id="LC260" class="blob-line-code js-file-line">and accordingly for the following spike of the second spike train. In analogy to Eq. <span class="k">\ref</span><span class="nb">{</span>eq:Bi-Spike-Diss-RT<span class="nb">}</span>, an indicator of local spike train dissimilarity is obtained as follows:</td>
      </tr>
      <tr>
        <td id="L261" class="blob-line-num js-line-number" data-line-number="261"></td>
        <td id="LC261" class="blob-line-code js-file-line"><span class="c">%</span></td>
      </tr>
      <tr>
        <td id="L262" class="blob-line-num js-line-number" data-line-number="262"></td>
        <td id="LC262" class="blob-line-code js-file-line"><span class="k">\begin</span><span class="nb">{</span>equation<span class="nb">}</span> <span class="k">\label</span><span class="nb">{</span>eq:Bi-Spike-Diss-FT<span class="nb">}</span></td>
      </tr>
      <tr>
        <td id="L263" class="blob-line-num js-line-number" data-line-number="263"></td>
        <td id="LC263" class="blob-line-code js-file-line">    S<span class="nb">_</span>f (t) = <span class="k">\frac</span><span class="nb">{</span> <span class="k">\Delta</span> t<span class="nb">_{</span><span class="k">\mathrm</span> <span class="nb">{</span>F<span class="nb">}}^{</span>(1)<span class="nb">}</span> + <span class="k">\Delta</span> t<span class="nb">_{</span><span class="k">\mathrm</span> <span class="nb">{</span>F<span class="nb">}}^{</span>(2)<span class="nb">}}</span> <span class="nb">{</span>4 <span class="k">\langle</span> x<span class="nb">_{</span><span class="k">\mathrm</span> <span class="nb">{</span>F<span class="nb">}}^{</span>(n)<span class="nb">}</span> <span class="k">\rangle</span><span class="nb">_</span>n<span class="nb">}</span>.</td>
      </tr>
      <tr>
        <td id="L264" class="blob-line-num js-line-number" data-line-number="264"></td>
        <td id="LC264" class="blob-line-code js-file-line"><span class="k">\end</span><span class="nb">{</span>equation<span class="nb">}</span>	</td>
      </tr>
      <tr>
        <td id="L265" class="blob-line-num js-line-number" data-line-number="265"></td>
        <td id="LC265" class="blob-line-code js-file-line">
</td>
      </tr>
      <tr>
        <td id="L266" class="blob-line-num js-line-number" data-line-number="266"></td>
        <td id="LC266" class="blob-line-code js-file-line">
</td>
      </tr>
      <tr>
        <td id="L267" class="blob-line-num js-line-number" data-line-number="267"></td>
        <td id="LC267" class="blob-line-code js-file-line"><span class="k">\subsection</span><span class="nb">{</span><span class="k">\label</span><span class="nb">{</span>ss:Improvements<span class="nb">}</span> Improvements in this implementation<span class="nb">}</span></td>
      </tr>
      <tr>
        <td id="L268" class="blob-line-num js-line-number" data-line-number="268"></td>
        <td id="LC268" class="blob-line-code js-file-line">
</td>
      </tr>
      <tr>
        <td id="L269" class="blob-line-num js-line-number" data-line-number="269"></td>
        <td id="LC269" class="blob-line-code js-file-line"><span class="k">\subsubsection</span><span class="nb">{</span><span class="k">\label</span><span class="nb">{</span>sss:Edge-effect<span class="nb">}</span> Correction of the edge effect<span class="nb">}</span></td>
      </tr>
      <tr>
        <td id="L270" class="blob-line-num js-line-number" data-line-number="270"></td>
        <td id="LC270" class="blob-line-code js-file-line">
</td>
      </tr>
      <tr>
        <td id="L271" class="blob-line-num js-line-number" data-line-number="271"></td>
        <td id="LC271" class="blob-line-code js-file-line">In previous versions of the SPIKE-distance the ambiguity regarding the definition of the initial (final) distance to the preceding (following) spike as well as the very first and the very last interspike intervals was resolved by adding to each spike train an auxiliary leading spike at time <span class="s">$</span><span class="nb">t </span><span class="o">=</span><span class="nb"> </span><span class="m">0</span><span class="s">$</span> and an auxiliary trailing spike at time <span class="s">$</span><span class="nb">t </span><span class="o">=</span><span class="nb"> T</span><span class="s">$</span>. This lead to spurious synchrony at the edges where by construction the dissimilarity profile reached the zero value. Here we follow a suggestion by Conor Houghton (personal communication) and at least partly correct this edge effect. We describe the correction for the beginning of the recording, it is an analogous mirror image at the end of the recording.</td>
      </tr>
      <tr>
        <td id="L272" class="blob-line-num js-line-number" data-line-number="272"></td>
        <td id="LC272" class="blob-line-code js-file-line">
</td>
      </tr>
      <tr>
        <td id="L273" class="blob-line-num js-line-number" data-line-number="273"></td>
        <td id="LC273" class="blob-line-code js-file-line">We count the auxiliary spikes as normal spikes which can be nearest neighbor to other spikes. But instead of calculating their spike time distance (which is always zero) we use the spike time difference of the first real spike. For the first interspike interval we know that it is at least the distance to the first spike <span class="s">$</span><span class="nb">t_</span><span class="m">1</span><span class="o">-</span><span class="nb">t_</span><span class="m">0</span><span class="nb"> </span><span class="o">=</span><span class="nb"> t_</span><span class="m">1</span><span class="s">$</span> but it could be longer. So to take the local firing rate (or its inverse) into consideration we set</td>
      </tr>
      <tr>
        <td id="L274" class="blob-line-num js-line-number" data-line-number="274"></td>
        <td id="LC274" class="blob-line-code js-file-line"><span class="c">%</span></td>
      </tr>
      <tr>
        <td id="L275" class="blob-line-num js-line-number" data-line-number="275"></td>
        <td id="LC275" class="blob-line-code js-file-line"><span class="k">\begin</span><span class="nb">{</span>equation<span class="nb">}</span> <span class="k">\label</span><span class="nb">{</span>eq:Corrected-First-ISI<span class="nb">}</span></td>
      </tr>
      <tr>
        <td id="L276" class="blob-line-num js-line-number" data-line-number="276"></td>
        <td id="LC276" class="blob-line-code js-file-line">    x<span class="nb">_{</span><span class="k">\mathrm</span> <span class="nb">{</span>ISI<span class="nb">}}</span> (0) = <span class="k">\max</span> ( t<span class="nb">_</span>1, t<span class="nb">_</span>2 - t<span class="nb">_</span>1 )	</td>
      </tr>
      <tr>
        <td id="L277" class="blob-line-num js-line-number" data-line-number="277"></td>
        <td id="LC277" class="blob-line-code js-file-line"><span class="k">\end</span><span class="nb">{</span>equation<span class="nb">}</span></td>
      </tr>
      <tr>
        <td id="L278" class="blob-line-num js-line-number" data-line-number="278"></td>
        <td id="LC278" class="blob-line-code js-file-line"><span class="c">%</span></td>
      </tr>
      <tr>
        <td id="L279" class="blob-line-num js-line-number" data-line-number="279"></td>
        <td id="LC279" class="blob-line-code js-file-line">where we use the length of the first known interspike interval <span class="s">$</span><span class="nb">t_</span><span class="m">2</span><span class="o">-</span><span class="nb">t_</span><span class="m">1</span><span class="s">$</span> as an upper limit of the inverse firing rate. This way we get at least a crude estimate of how much longer the first interspike interval could be.</td>
      </tr>
      <tr>
        <td id="L280" class="blob-line-num js-line-number" data-line-number="280"></td>
        <td id="LC280" class="blob-line-code js-file-line">
</td>
      </tr>
      <tr>
        <td id="L281" class="blob-line-num js-line-number" data-line-number="281"></td>
        <td id="LC281" class="blob-line-code js-file-line">
</td>
      </tr>
      <tr>
        <td id="L282" class="blob-line-num js-line-number" data-line-number="282"></td>
        <td id="LC282" class="blob-line-code js-file-line"><span class="k">\subsubsection</span><span class="nb">{</span><span class="k">\label</span><span class="nb">{</span>sss:Sampling<span class="nb">}</span> Increase of memory efficiency by avoiding sampling<span class="nb">}</span></td>
      </tr>
      <tr>
        <td id="L283" class="blob-line-num js-line-number" data-line-number="283"></td>
        <td id="LC283" class="blob-line-code js-file-line">
</td>
      </tr>
      <tr>
        <td id="L284" class="blob-line-num js-line-number" data-line-number="284"></td>
        <td id="LC284" class="blob-line-code js-file-line">Earlier versions of the codes for calculating the ISI- and the SPIKE-distance relied on sampled dissimilarity profiles. Typically the precision was set to the sampling interval of the neuronal recording. Since the dissimilarity profile has to be calculated and stored for each pair of spike trains, this resulted, for each measure, in a matrix of order `number of sampled time instants&#39; <span class="s">$</span><span class="nv">\times</span><span class="s">$</span> `number of spike train pairs&#39; (i.e., <span class="s">$</span><span class="nv">\#</span><span class="nb"> </span><span class="o">(</span><span class="nb">t_s</span><span class="o">)</span><span class="nb"> </span><span class="nv">\times</span><span class="nb"> N</span><span class="o">(</span><span class="nb">N</span><span class="o">-</span><span class="m">1</span><span class="o">)/</span><span class="m">2</span><span class="s">$</span>). For small sampling intervals and a large number of recorded spike trains this lead to memory problems.</td>
      </tr>
      <tr>
        <td id="L285" class="blob-line-num js-line-number" data-line-number="285"></td>
        <td id="LC285" class="blob-line-code js-file-line">In SPIKY we use an optimized and more memory-efficient way of storing the data where we make use of the fact that the dissimilarity profile <span class="s">$</span><span class="nb">I </span><span class="o">(</span><span class="nb">t</span><span class="o">)</span><span class="s">$</span> of the ISI-distance is piecewise constant and the dissimilarity profile <span class="s">$</span><span class="nb">S </span><span class="o">(</span><span class="nb">t</span><span class="o">)</span><span class="s">$</span> of the SPIKE-distance is piecewise linear with each interval running from one spike of the pooled spike train to the next one. Thus for each such interval (and for each pair of spike trains) we have to store only one value for the ISI-distance and two values for the SPIKE-distance, one at the beginning and one at the end of the interval. Typically the storage space required will be much smaller than for dissimilarity profiles sampled with a reasonable precision.</td>
      </tr>
      <tr>
        <td id="L286" class="blob-line-num js-line-number" data-line-number="286"></td>
        <td id="LC286" class="blob-line-code js-file-line">
</td>
      </tr>
      <tr>
        <td id="L287" class="blob-line-num js-line-number" data-line-number="287"></td>
        <td id="LC287" class="blob-line-code js-file-line">For both dissimilarity profiles there are instantaneous jumps at the times of the spikes since this is where the lengths of the interspike intervals and the identity of the previous and the following spikes change abruptly. In contrast to the calculation based on sampling we get the exact result since each spike is both the previous and the next spike and there is no need anymore to `cut the corners&#39; of the dissimilarity profiles as had to be done for the sampled dissimilarity profiles. The dissimilarity profiles <span class="s">$</span><span class="nb">S_r</span><span class="o">(</span><span class="nb">t</span><span class="o">)</span><span class="s">$</span> and <span class="s">$</span><span class="nb">S_f </span><span class="o">(</span><span class="nb">t</span><span class="o">)</span><span class="s">$</span> of the real-time and the future SPIKE-distances are hyperbolic but for these measures the exact result can be obtained by piecewise integration over all intervals of the pooled spike train.</td>
      </tr>
      <tr>
        <td id="L288" class="blob-line-num js-line-number" data-line-number="288"></td>
        <td id="LC288" class="blob-line-code js-file-line">
</td>
      </tr>
      <tr>
        <td id="L289" class="blob-line-num js-line-number" data-line-number="289"></td>
        <td id="LC289" class="blob-line-code js-file-line">
</td>
      </tr>
      <tr>
        <td id="L290" class="blob-line-num js-line-number" data-line-number="290"></td>
        <td id="LC290" class="blob-line-code js-file-line"><span class="k">\subsection</span><span class="nb">{</span><span class="k">\label</span><span class="nb">{</span>ss:Information-reduction<span class="nb">}</span> Levels of information reduction<span class="nb">}</span></td>
      </tr>
      <tr>
        <td id="L291" class="blob-line-num js-line-number" data-line-number="291"></td>
        <td id="LC291" class="blob-line-code js-file-line">
</td>
      </tr>
      <tr>
        <td id="L292" class="blob-line-num js-line-number" data-line-number="292"></td>
        <td id="LC292" class="blob-line-code js-file-line">The ISI- and the SPIKE-distance combine a variety of properties that make them well suited for applications to real data. In particular, they are conceptually simple, computationally efficient, and easy to visualize in a time-resolved manner. By taking into account only the preceding and the following spike in each spike train, these distances rely on local information only. They are also time-scale-adaptive since the information used is not contained within a window of fixed size but rather within a time frame whose size depends on the local rate of each spike train.</td>
      </tr>
      <tr>
        <td id="L293" class="blob-line-num js-line-number" data-line-number="293"></td>
        <td id="LC293" class="blob-line-code js-file-line">
</td>
      </tr>
      <tr>
        <td id="L294" class="blob-line-num js-line-number" data-line-number="294"></td>
        <td id="LC294" class="blob-line-code js-file-line">Moreover, the sensitivity to spike timing and the instantaneous reliability achieved by the SPIKE-distance opens up many new possibilities in multi-neuron spike train analysis <span class="k">\citep</span><span class="nb">{</span>Kreuz13<span class="nb">}</span>. These build upon the fact that there are several levels of information reduction all of which we describe in the following. As illustration we use the detailed analysis of an artificially generated spike train dataset (see upper subplot of Fig. <span class="k">\ref</span><span class="nb">{</span>fig:Fig2-SPIKE-Representations<span class="nb">}</span>A for the rasterplot).</td>
      </tr>
      <tr>
        <td id="L295" class="blob-line-num js-line-number" data-line-number="295"></td>
        <td id="LC295" class="blob-line-code js-file-line"><span class="c">%</span></td>
      </tr>
      <tr>
        <td id="L296" class="blob-line-num js-line-number" data-line-number="296"></td>
        <td id="LC296" class="blob-line-code js-file-line"><span class="c">% #########################################################################################</span></td>
      </tr>
      <tr>
        <td id="L297" class="blob-line-num js-line-number" data-line-number="297"></td>
        <td id="LC297" class="blob-line-code js-file-line"><span class="c">% #########################################################################################</span></td>
      </tr>
      <tr>
        <td id="L298" class="blob-line-num js-line-number" data-line-number="298"></td>
        <td id="LC298" class="blob-line-code js-file-line"><span class="c">% #################################################### Figure 2 ###########################</span></td>
      </tr>
      <tr>
        <td id="L299" class="blob-line-num js-line-number" data-line-number="299"></td>
        <td id="LC299" class="blob-line-code js-file-line"><span class="c">% #########################################################################################</span></td>
      </tr>
      <tr>
        <td id="L300" class="blob-line-num js-line-number" data-line-number="300"></td>
        <td id="LC300" class="blob-line-code js-file-line"><span class="c">% #########################################################################################</span></td>
      </tr>
      <tr>
        <td id="L301" class="blob-line-num js-line-number" data-line-number="301"></td>
        <td id="LC301" class="blob-line-code js-file-line"><span class="c">%</span></td>
      </tr>
      <tr>
        <td id="L302" class="blob-line-num js-line-number" data-line-number="302"></td>
        <td id="LC302" class="blob-line-code js-file-line"><span class="k">\begin</span><span class="nb">{</span>figure<span class="nb">}</span></td>
      </tr>
      <tr>
        <td id="L303" class="blob-line-num js-line-number" data-line-number="303"></td>
        <td id="LC303" class="blob-line-code js-file-line">    <span class="k">\includegraphics</span><span class="na">[width=85mm]</span><span class="nb">{</span>Fig2<span class="nb">_</span>SPIKE<span class="nb">_</span>Representations.eps<span class="nb">}</span></td>
      </tr>
      <tr>
        <td id="L304" class="blob-line-num js-line-number" data-line-number="304"></td>
        <td id="LC304" class="blob-line-code js-file-line">    <span class="k">\caption</span><span class="nb">{</span><span class="k">\abb\label</span><span class="nb">{</span>fig:Fig2-SPIKE-Representations<span class="nb">}</span> The different levels of information reduction for the SPIKE-distance.  A. Top: Spike rasterplot of <span class="s">$</span><span class="m">20</span><span class="s">$</span> artificially generated spike trains divided in <span class="s">$</span><span class="m">4</span><span class="s">$</span> spike train groups of <span class="s">$</span><span class="m">5</span><span class="s">$</span> spike trains each. The clustering behavior changes every <span class="s">$</span><span class="m">500</span><span class="s">$</span> ms. Bottom: Dissimilarity profiles of the SPIKE-distance for the four spike train groups (thin color-coded lines) and for all spike trains (thick black line). The overall dissimilarity is defined as the temporal average of the dissimilarity profile of all spike trains (<span class="s">$</span><span class="m">0</span><span class="nb">.</span><span class="m">192</span><span class="s">$</span> in this case) and is marked by a black horizontal line.  B-E. Matrices of pairwise instantaneous dissimilarity values for a single time instant, for two selective averages and for a triggered average.  F. Matrices of overall pairwise instantaneous dissimilarity values for the <span class="s">$</span><span class="m">4</span><span class="s">$</span> spike train groups.  G. Dendrogram of spike train group matrix in F.  H/I. Dendrogram of spike train matrices in D/E. Note that the triggered averages in contrast to the overall average captures the local similarity between <span class="s">$</span><span class="m">5</span><span class="s">$</span> of the spike trains.<span class="nb">}</span></td>
      </tr>
      <tr>
        <td id="L305" class="blob-line-num js-line-number" data-line-number="305"></td>
        <td id="LC305" class="blob-line-code js-file-line"><span class="k">\end</span><span class="nb">{</span>figure<span class="nb">}</span></td>
      </tr>
      <tr>
        <td id="L306" class="blob-line-num js-line-number" data-line-number="306"></td>
        <td id="LC306" class="blob-line-code js-file-line"><span class="c">%</span></td>
      </tr>
      <tr>
        <td id="L307" class="blob-line-num js-line-number" data-line-number="307"></td>
        <td id="LC307" class="blob-line-code js-file-line"><span class="c">% #########################################################################################</span></td>
      </tr>
      <tr>
        <td id="L308" class="blob-line-num js-line-number" data-line-number="308"></td>
        <td id="LC308" class="blob-line-code js-file-line"><span class="c">% #########################################################################################</span></td>
      </tr>
      <tr>
        <td id="L309" class="blob-line-num js-line-number" data-line-number="309"></td>
        <td id="LC309" class="blob-line-code js-file-line"><span class="c">% #################################################### Figure 2 ###########################</span></td>
      </tr>
      <tr>
        <td id="L310" class="blob-line-num js-line-number" data-line-number="310"></td>
        <td id="LC310" class="blob-line-code js-file-line"><span class="c">% #########################################################################################</span></td>
      </tr>
      <tr>
        <td id="L311" class="blob-line-num js-line-number" data-line-number="311"></td>
        <td id="LC311" class="blob-line-code js-file-line"><span class="c">% #########################################################################################</span></td>
      </tr>
      <tr>
        <td id="L312" class="blob-line-num js-line-number" data-line-number="312"></td>
        <td id="LC312" class="blob-line-code js-file-line">
</td>
      </tr>
      <tr>
        <td id="L313" class="blob-line-num js-line-number" data-line-number="313"></td>
        <td id="LC313" class="blob-line-code js-file-line"><span class="k">\subsubsection</span><span class="nb">{</span><span class="k">\label</span><span class="nb">{</span>sss:Full-matrix-and-cross-sections<span class="nb">}</span> Full matrix and cross sections<span class="nb">}</span></td>
      </tr>
      <tr>
        <td id="L314" class="blob-line-num js-line-number" data-line-number="314"></td>
        <td id="LC314" class="blob-line-code js-file-line">
</td>
      </tr>
      <tr>
        <td id="L315" class="blob-line-num js-line-number" data-line-number="315"></td>
        <td id="LC315" class="blob-line-code js-file-line">The starting point is the most detailed representation in which one instantaneous value is obtained for each pair of spike trains (see Eq. <span class="k">\ref</span><span class="nb">{</span>eq:Bi-Spike-Diss-Improved<span class="nb">}</span>). This representation could be viewed as a movie of a symmetric pairwise dissimilarity matrix in which each frame corresponds to one time instant (an example can be found in the supplementary material of <span class="k">\citeauthor</span><span class="nb">{</span>Kreuz13<span class="nb">}</span>, <span class="k">\citeyear</span><span class="nb">{</span>Kreuz13<span class="nb">}</span>). For a movie of finite length the time axis necessarily has to be sampled but in principle this most detailed representation is continuous and consists of an infinite number of values. However, since all dissimilarity profiles are piecewise linear (with potential discontinuities only at the times of the spikes) there is a lot of redundancy. Using the most compact and memory-efficient representation one can store all pairwise dissimilarity profiles in a matrix of size `number of unique spikes in the pooled spike train&#39; <span class="s">$</span><span class="nv">\times</span><span class="s">$</span> `number of spike train pairs&#39; (<span class="s">$</span><span class="nv">\times</span><span class="nb"> </span><span class="m">2</span><span class="s">$</span> for the SPIKE-distance, see Section <span class="k">\ref</span><span class="nb">{</span>sss:Sampling<span class="nb">}</span>).</td>
      </tr>
      <tr>
        <td id="L316" class="blob-line-num js-line-number" data-line-number="316"></td>
        <td id="LC316" class="blob-line-code js-file-line">
</td>
      </tr>
      <tr>
        <td id="L317" class="blob-line-num js-line-number" data-line-number="317"></td>
        <td id="LC317" class="blob-line-code js-file-line">From this matrix, it is possible to extract any desired information. By selecting a pair of spike trains, one obtains the bivariate dissimilarity profile <span class="s">$</span><span class="nb">S </span><span class="o">(</span><span class="nb">t</span><span class="o">)</span><span class="s">$</span> for this pair of spike trains. An average dissimilarity profile <span class="s">$</span><span class="nb">S </span><span class="o">(</span><span class="nb">t</span><span class="o">)</span><span class="s">$</span> is obtained upon the selection of a subgroup or of all pairs of spike trains. Selecting a time instant <span class="s">$</span><span class="nb">t_s</span><span class="s">$</span> (and using linear interpolation for time instants in between spikes) yields an instantaneous matrix of pairwise spike train dissimilarities <span class="s">$</span><span class="nb">S_{mn}</span><span class="o">(</span><span class="nb">t_s</span><span class="o">)</span><span class="s">$</span>  (see Fig. <span class="k">\ref</span><span class="nb">{</span>fig:Fig2-SPIKE-Representations<span class="nb">}</span>B). This matrix can be used to divide the spike trains into instantaneous clusters, that is, groups of spike trains with low intra-group and high inter-group dissimilarity.</td>
      </tr>
      <tr>
        <td id="L318" class="blob-line-num js-line-number" data-line-number="318"></td>
        <td id="LC318" class="blob-line-code js-file-line">
</td>
      </tr>
      <tr>
        <td id="L319" class="blob-line-num js-line-number" data-line-number="319"></td>
        <td id="LC319" class="blob-line-code js-file-line"><span class="k">\subsubsection</span><span class="nb">{</span><span class="k">\label</span><span class="nb">{</span>sss:Spatial-and-temporal-Averaging<span class="nb">}</span> Spatial and temporal averaging<span class="nb">}</span></td>
      </tr>
      <tr>
        <td id="L320" class="blob-line-num js-line-number" data-line-number="320"></td>
        <td id="LC320" class="blob-line-code js-file-line">
</td>
      </tr>
      <tr>
        <td id="L321" class="blob-line-num js-line-number" data-line-number="321"></td>
        <td id="LC321" class="blob-line-code js-file-line">Another way to reduce the information of the dissimilarity matrix is averaging. There are two possibilities that commute: the spatial average over spike train pairs and the temporal average.</td>
      </tr>
      <tr>
        <td id="L322" class="blob-line-num js-line-number" data-line-number="322"></td>
        <td id="LC322" class="blob-line-code js-file-line">
</td>
      </tr>
      <tr>
        <td id="L323" class="blob-line-num js-line-number" data-line-number="323"></td>
        <td id="LC323" class="blob-line-code js-file-line">The local average over spike train pairs yields a dissimilarity profile for the whole population. Examples for <span class="s">$</span><span class="m">4</span><span class="s">$</span> different spike train groups as well as all spike trains are shown in the lower subplot of Fig. <span class="k">\ref</span><span class="nb">{</span>fig:Fig2-SPIKE-Representations<span class="nb">}</span>A. Temporal averaging over certain intervals on the other hand leads to a bivariate distance matrix (see Fig. <span class="k">\ref</span><span class="nb">{</span>fig:Fig2-SPIKE-Representations<span class="nb">}</span>C and D for examples of non-continuous and continuous intervals). In real data, these intervals could be chosen to correspond to different external conditions such as normal vs. pathological, asleep vs. awake, target vs. non-target stimulus, or presence/absence of a certain channel blocker. Finally, in both cases, application of the respective remaining average results in one distance value that describes the overall level of synchrony for a group of spike trains over a given time interval. In Fig. <span class="k">\ref</span><span class="nb">{</span>fig:Fig2-SPIKE-Representations<span class="nb">}</span>A this value is stated in the upper right of the lower subplot.</td>
      </tr>
      <tr>
        <td id="L324" class="blob-line-num js-line-number" data-line-number="324"></td>
        <td id="LC324" class="blob-line-code js-file-line">
</td>
      </tr>
      <tr>
        <td id="L325" class="blob-line-num js-line-number" data-line-number="325"></td>
        <td id="LC325" class="blob-line-code js-file-line"><span class="k">\subsubsection</span><span class="nb">{</span><span class="k">\label</span><span class="nb">{</span>sss:Triggered-Averaging<span class="nb">}</span> Triggered averaging<span class="nb">}</span></td>
      </tr>
      <tr>
        <td id="L326" class="blob-line-num js-line-number" data-line-number="326"></td>
        <td id="LC326" class="blob-line-code js-file-line">
</td>
      </tr>
      <tr>
        <td id="L327" class="blob-line-num js-line-number" data-line-number="327"></td>
        <td id="LC327" class="blob-line-code js-file-line">The fact that there are no limits to the temporal resolution allows further analyses such as internally or externally triggered temporal averaging. Here, the matrices are averaged over certain trigger time instants only. The idea is to check whether this triggered temporal average is significantly different from the global average since this would indicate that something peculiar is happening at these trigger instants. The trigger times can either be obtained from external influences (such as the occurrence of certain features in a stimulus) or from internal conditions (such as the spike times of a certain spike train). External triggering is a standard tool to address questions of neuronal coding, for example, it can be used to evaluate the influence of localized stimulus features on the reliability of neurons under repeated stimulation. In multi-neuron data, internal triggering might help to uncover the connectivity in neural networks or to detect converging or diverging patterns of firing propagation. An example is shown in Fig. <span class="k">\ref</span><span class="nb">{</span>fig:Fig2-SPIKE-Representations<span class="nb">}</span>E. Here, neurons <span class="s">$</span><span class="m">2</span><span class="s">$</span>, <span class="s">$</span><span class="m">9</span><span class="s">$</span>, <span class="s">$</span><span class="m">14</span><span class="s">$</span>, and <span class="s">$</span><span class="m">19</span><span class="s">$</span> follow neuron <span class="s">$</span><span class="m">7</span><span class="s">$</span> #####	</td>
      </tr>
      <tr>
        <td id="L328" class="blob-line-num js-line-number" data-line-number="328"></td>
        <td id="LC328" class="blob-line-code js-file-line">
</td>
      </tr>
      <tr>
        <td id="L329" class="blob-line-num js-line-number" data-line-number="329"></td>
        <td id="LC329" class="blob-line-code js-file-line"> </td>
      </tr>
      <tr>
        <td id="L330" class="blob-line-num js-line-number" data-line-number="330"></td>
        <td id="LC330" class="blob-line-code js-file-line">A last possibility is spatial averaging such that the spike trains are manually assigned to subgroups, and a block matrix (and the corresponding dendrogram XXXXX) is obtained by averaging over the respective submatrices of the original dissimilarity matrix. In applications to real data, these groups could be different neuronal populations or responses to different stimuli, depending on whether the spike trains were recorded simultaneously or successively.</td>
      </tr>
      <tr>
        <td id="L331" class="blob-line-num js-line-number" data-line-number="331"></td>
        <td id="LC331" class="blob-line-code js-file-line">
</td>
      </tr>
      <tr>
        <td id="L332" class="blob-line-num js-line-number" data-line-number="332"></td>
        <td id="LC332" class="blob-line-code js-file-line"><span class="c">%</span></td>
      </tr>
      <tr>
        <td id="L333" class="blob-line-num js-line-number" data-line-number="333"></td>
        <td id="LC333" class="blob-line-code js-file-line"><span class="c">%</span></td>
      </tr>
      <tr>
        <td id="L334" class="blob-line-num js-line-number" data-line-number="334"></td>
        <td id="LC334" class="blob-line-code js-file-line"><span class="c">% *************************************************************************************</span></td>
      </tr>
      <tr>
        <td id="L335" class="blob-line-num js-line-number" data-line-number="335"></td>
        <td id="LC335" class="blob-line-code js-file-line"><span class="c">% **************************************************************** SPIKY **************</span></td>
      </tr>
      <tr>
        <td id="L336" class="blob-line-num js-line-number" data-line-number="336"></td>
        <td id="LC336" class="blob-line-code js-file-line"><span class="c">% *************************************************************************************</span></td>
      </tr>
      <tr>
        <td id="L337" class="blob-line-num js-line-number" data-line-number="337"></td>
        <td id="LC337" class="blob-line-code js-file-line"><span class="c">%</span></td>
      </tr>
      <tr>
        <td id="L338" class="blob-line-num js-line-number" data-line-number="338"></td>
        <td id="LC338" class="blob-line-code js-file-line"><span class="c">%</span></td>
      </tr>
      <tr>
        <td id="L339" class="blob-line-num js-line-number" data-line-number="339"></td>
        <td id="LC339" class="blob-line-code js-file-line"><span class="k">\section</span><span class="nb">{</span><span class="k">\label</span><span class="nb">{</span>s:SPIKY<span class="nb">}</span> SPIKY<span class="nb">}</span></td>
      </tr>
      <tr>
        <td id="L340" class="blob-line-num js-line-number" data-line-number="340"></td>
        <td id="LC340" class="blob-line-code js-file-line">
</td>
      </tr>
      <tr>
        <td id="L341" class="blob-line-num js-line-number" data-line-number="341"></td>
        <td id="LC341" class="blob-line-code js-file-line">SPIKY is a graphical user interface (GUI) for monitoring synchrony between artificially simulated or experimentally recorded neuronal spike trains. It is an implementation of the ISI- and the SPIKE-distance (including simulations of its realtime and future variants) which allows  interactive access to all the different representations described in Section <span class="k">\ref</span><span class="nb">{</span>ss:Information-reduction<span class="nb">}</span>. SPIKY is a free software package programmed by Thomas Kreuz and Nebojsa Bozanic. All source codes are written in Matlab (MathWorks Inc, Natick, MA, USA) with the most time-consuming loops coded in MEX-files <span class="k">\footnote</span><span class="nb">{</span>In our case these are subroutines written in C. However, as some users may not have access to a suitable C compiler, SPIKY contains the (slower) pure Matlab code as well.<span class="nb">}</span>. SPIKY is not stand-alone but requires Matlab to run. XXXXX Update, now stand-alone? XXXXX</td>
      </tr>
      <tr>
        <td id="L342" class="blob-line-num js-line-number" data-line-number="342"></td>
        <td id="LC342" class="blob-line-code js-file-line">
</td>
      </tr>
      <tr>
        <td id="L343" class="blob-line-num js-line-number" data-line-number="343"></td>
        <td id="LC343" class="blob-line-code js-file-line">
</td>
      </tr>
      <tr>
        <td id="L344" class="blob-line-num js-line-number" data-line-number="344"></td>
        <td id="LC344" class="blob-line-code js-file-line"><span class="k">\subsection</span><span class="nb">{</span><span class="k">\label</span><span class="nb">{</span>ss:Access<span class="nb">}</span> Access to SPIKY and how to get started<span class="nb">}</span></td>
      </tr>
      <tr>
        <td id="L345" class="blob-line-num js-line-number" data-line-number="345"></td>
        <td id="LC345" class="blob-line-code js-file-line">
</td>
      </tr>
      <tr>
        <td id="L346" class="blob-line-num js-line-number" data-line-number="346"></td>
        <td id="LC346" class="blob-line-code js-file-line">SPIKY is distributed under a BSD licence (Copyright (c) 2014, Thomas Kreuz, Nebojsa Bozanic. All rights reserved.). A zip-package containing all the necessary files can be accessed for free on the <span class="k">\href</span><span class="nb">{</span>http://www.fi.isc.cnr.it/users/thomas.kreuz/Source-Code/SPIKY.html<span class="nb">}{</span>download page<span class="nb">}</span> (<span class="k">\url</span><span class="nb">{</span>http://www.fi.isc.cnr.it/users/thomas.kreuz/Source-Code/SPIKY.html<span class="nb">}</span>). This package also contains a folder with lots of documentation (such as a FAQ-file and an introduction to all individual elements and all individual files of SPIKY). Further information and many demonstrations (both images and movies) can be found on the download page and on the <span class="k">\href</span><span class="nb">{</span>https://www.facebook.com/SPIKYgui<span class="nb">}{</span>SPIKY Facebook-page<span class="nb">}</span> (<span class="k">\url</span><span class="nb">{</span>https://www.facebook.com/SPIKYgui<span class="nb">}</span>). Both of these pages are used to announce updates and distribute the latest information about new features. They also provide the user with an opportunity to provide feedback and ask questions.</td>
      </tr>
      <tr>
        <td id="L347" class="blob-line-num js-line-number" data-line-number="347"></td>
        <td id="LC347" class="blob-line-code js-file-line">
</td>
      </tr>
      <tr>
        <td id="L348" class="blob-line-num js-line-number" data-line-number="348"></td>
        <td id="LC348" class="blob-line-code js-file-line">The Facebook-page include various screen recordings with voice-over in which the user is guided step by step through some of the most important features of SPIKY. All of these movies can also be viewed on the <span class="k">\href</span><span class="nb">{</span>https://www.youtube.com/channel/UCgSz0YQ5lWdVF0<span class="nb">_</span>Z1FNN0Bw<span class="nb">}{</span>SPIKY Youtube-channel<span class="nb">}</span> (<span class="k">\url</span><span class="nb">{</span>https://www.youtube.com/channel/UCgSz0YQ5lWdVF0<span class="nb">_</span>Z1FNN0Bw<span class="nb">}</span>).</td>
      </tr>
      <tr>
        <td id="L349" class="blob-line-num js-line-number" data-line-number="349"></td>
        <td id="LC349" class="blob-line-code js-file-line">
</td>
      </tr>
      <tr>
        <td id="L350" class="blob-line-num js-line-number" data-line-number="350"></td>
        <td id="LC350" class="blob-line-code js-file-line">After having downloaded SPIKY from the download page the user has to first extract the zip-package which leaves all files in one folder named `SPIKY&#39;. If the system has a suitable MEX-compiler installed, the MEX-files can be compiled from within this folder by running the m-file `SPIKY<span class="k">\_</span>compile<span class="k">\_</span>MEX&#39;. The last step is to run the m-file `SPIKY&#39;. XXXXX Stand-alone version? XXXXX</td>
      </tr>
      <tr>
        <td id="L351" class="blob-line-num js-line-number" data-line-number="351"></td>
        <td id="LC351" class="blob-line-code js-file-line">
</td>
      </tr>
      <tr>
        <td id="L352" class="blob-line-num js-line-number" data-line-number="352"></td>
        <td id="LC352" class="blob-line-code js-file-line">When SPIKY is running, the user can find quick information about the individual elements of the graphical user interface by activating the `Hints&#39;-checkbox in the `Options&#39;-Menu. Hovering with the mouse cursor above the elements of interest will then show short hints. An overview of all the information contained in the hints can be found in the documentation file `SPIKY-Elements.doc&#39;. Furthermore, at each step the suggested element for the next user action is highlighted by a bold font. </td>
      </tr>
      <tr>
        <td id="L353" class="blob-line-num js-line-number" data-line-number="353"></td>
        <td id="LC353" class="blob-line-code js-file-line">
</td>
      </tr>
      <tr>
        <td id="L354" class="blob-line-num js-line-number" data-line-number="354"></td>
        <td id="LC354" class="blob-line-code js-file-line">To get the user started quickly, SPIKY provides a few (artificial) example datasets from previous publications. The most useful example is the entry `Clustering&#39; in the `Selection: Data&#39;-listbox. This dataset has already been used in several figures as well as in the supplementary movie of <span class="k">\cite</span><span class="nb">{</span>Kreuz13<span class="nb">}</span>. A good introduction to SPIKY would be to follow this example through till the end advancing from panel to panel by pressing the highlighted button. In a second step one could reset and run the same example again while changing some parameters in order to see the consequences. Note that it is not necessary to set all the parameters each time when SPIKY is started. Rather it is possible to use the file `SPIKY<span class="k">\_</span>f<span class="k">\_</span>user<span class="k">\_</span>interface&#39; to set and modify the spike train data as well as the parameters (again refer to the dataset `Clustering&#39; for an example).</td>
      </tr>
      <tr>
        <td id="L355" class="blob-line-num js-line-number" data-line-number="355"></td>
        <td id="LC355" class="blob-line-code js-file-line">
</td>
      </tr>
      <tr>
        <td id="L356" class="blob-line-num js-line-number" data-line-number="356"></td>
        <td id="LC356" class="blob-line-code js-file-line">
</td>
      </tr>
      <tr>
        <td id="L357" class="blob-line-num js-line-number" data-line-number="357"></td>
        <td id="LC357" class="blob-line-code js-file-line"><span class="k">\subsection</span><span class="nb">{</span><span class="k">\label</span><span class="nb">{</span>ss:Structure<span class="nb">}</span> Structure and workflow of SPIKY<span class="nb">}</span></td>
      </tr>
      <tr>
        <td id="L358" class="blob-line-num js-line-number" data-line-number="358"></td>
        <td id="LC358" class="blob-line-code js-file-line">
</td>
      </tr>
      <tr>
        <td id="L359" class="blob-line-num js-line-number" data-line-number="359"></td>
        <td id="LC359" class="blob-line-code js-file-line">Overall, SPIKY has a rather linear workflow, however, it is much more interactive than previous implementations of the ISI- and the SPIKE-distance and there are many potential shortcuts and loops along the way. As you can see in the SPIKY-flowchart in Fig. <span class="k">\ref</span><span class="nb">{</span>fig:Fig3-SPIKY-Flowchart<span class="nb">}</span>, the general flow is clearly directed from the input of spike train data to the output of results. So the first step the user has to do is to give SPIKY spike train data (i.e. sequences of spike times) to work with. There are three possibilities to do so: one can make use of predefined examples, load data from a file, or employ the spike train generator (see Section <span class="k">\ref</span><span class="nb">{</span>sss:Input<span class="nb">}</span> for more details).</td>
      </tr>
      <tr>
        <td id="L360" class="blob-line-num js-line-number" data-line-number="360"></td>
        <td id="LC360" class="blob-line-code js-file-line"><span class="c">%</span></td>
      </tr>
      <tr>
        <td id="L361" class="blob-line-num js-line-number" data-line-number="361"></td>
        <td id="LC361" class="blob-line-code js-file-line"><span class="c">% #########################################################################################</span></td>
      </tr>
      <tr>
        <td id="L362" class="blob-line-num js-line-number" data-line-number="362"></td>
        <td id="LC362" class="blob-line-code js-file-line"><span class="c">% #########################################################################################</span></td>
      </tr>
      <tr>
        <td id="L363" class="blob-line-num js-line-number" data-line-number="363"></td>
        <td id="LC363" class="blob-line-code js-file-line"><span class="c">% #################################################### Figure 3 ###########################</span></td>
      </tr>
      <tr>
        <td id="L364" class="blob-line-num js-line-number" data-line-number="364"></td>
        <td id="LC364" class="blob-line-code js-file-line"><span class="c">% #########################################################################################</span></td>
      </tr>
      <tr>
        <td id="L365" class="blob-line-num js-line-number" data-line-number="365"></td>
        <td id="LC365" class="blob-line-code js-file-line"><span class="c">% #########################################################################################</span></td>
      </tr>
      <tr>
        <td id="L366" class="blob-line-num js-line-number" data-line-number="366"></td>
        <td id="LC366" class="blob-line-code js-file-line"><span class="c">%</span></td>
      </tr>
      <tr>
        <td id="L367" class="blob-line-num js-line-number" data-line-number="367"></td>
        <td id="LC367" class="blob-line-code js-file-line"><span class="k">\begin</span><span class="nb">{</span>figure<span class="nb">}</span></td>
      </tr>
      <tr>
        <td id="L368" class="blob-line-num js-line-number" data-line-number="368"></td>
        <td id="LC368" class="blob-line-code js-file-line">    <span class="k">\includegraphics</span><span class="na">[width=85mm]</span><span class="nb">{</span>Fig3<span class="nb">_</span>SPIKY<span class="nb">_</span>Flowchart.eps<span class="nb">}</span></td>
      </tr>
      <tr>
        <td id="L369" class="blob-line-num js-line-number" data-line-number="369"></td>
        <td id="LC369" class="blob-line-code js-file-line">    <span class="k">\caption</span><span class="nb">{</span><span class="k">\abb\label</span><span class="nb">{</span>fig:Fig3-SPIKY-Flowchart<span class="nb">}</span> Flowchart describing the workflow of SPIKY from 	the input of spike train data to the output of results. A typical SPIKY-session begins at the top with `Get data&#39;, then goes clockwise and ends on the left with `Get results&#39;. In the center of the circle the SPIKY-logo is depicted.<span class="nb">}</span></td>
      </tr>
      <tr>
        <td id="L370" class="blob-line-num js-line-number" data-line-number="370"></td>
        <td id="LC370" class="blob-line-code js-file-line"><span class="k">\end</span><span class="nb">{</span>figure<span class="nb">}</span></td>
      </tr>
      <tr>
        <td id="L371" class="blob-line-num js-line-number" data-line-number="371"></td>
        <td id="LC371" class="blob-line-code js-file-line"><span class="c">%</span></td>
      </tr>
      <tr>
        <td id="L372" class="blob-line-num js-line-number" data-line-number="372"></td>
        <td id="LC372" class="blob-line-code js-file-line"><span class="c">% #########################################################################################</span></td>
      </tr>
      <tr>
        <td id="L373" class="blob-line-num js-line-number" data-line-number="373"></td>
        <td id="LC373" class="blob-line-code js-file-line"><span class="c">% #########################################################################################</span></td>
      </tr>
      <tr>
        <td id="L374" class="blob-line-num js-line-number" data-line-number="374"></td>
        <td id="LC374" class="blob-line-code js-file-line"><span class="c">% #################################################### Figure 3 ###########################</span></td>
      </tr>
      <tr>
        <td id="L375" class="blob-line-num js-line-number" data-line-number="375"></td>
        <td id="LC375" class="blob-line-code js-file-line"><span class="c">% #########################################################################################</span></td>
      </tr>
      <tr>
        <td id="L376" class="blob-line-num js-line-number" data-line-number="376"></td>
        <td id="LC376" class="blob-line-code js-file-line"><span class="c">% #########################################################################################</span></td>
      </tr>
      <tr>
        <td id="L377" class="blob-line-num js-line-number" data-line-number="377"></td>
        <td id="LC377" class="blob-line-code js-file-line">
</td>
      </tr>
      <tr>
        <td id="L378" class="blob-line-num js-line-number" data-line-number="378"></td>
        <td id="LC378" class="blob-line-code js-file-line">Once the full dataset is there, modification is still possible. One can restrict the analysis to a specific subset, e.g., select a smaller time window and/or a subset of spike trains. It is also possible to impose some external structure on the rasterplot (spike trains vs time). SPIKY allows the definition of two types of spike train separators (e.g. neurons from the left vs. neurons from the right hemisphere) and two types of time markers (e.g. specific events such as seizure onset and offset in epilepsy, trigger onset during stimulation etc.). It also enables the user to define spike train groups. Depending on the setup these could be spike trains recorded in different brain regions or upon presentation of different kinds of stimuli. Fig. <span class="k">\ref</span><span class="nb">{</span>fig:Fig4-Movie-Screenshot<span class="nb">}</span> shows an example of a raster plot with annotations marking all these different elements.</td>
      </tr>
      <tr>
        <td id="L379" class="blob-line-num js-line-number" data-line-number="379"></td>
        <td id="LC379" class="blob-line-code js-file-line"><span class="c">%</span></td>
      </tr>
      <tr>
        <td id="L380" class="blob-line-num js-line-number" data-line-number="380"></td>
        <td id="LC380" class="blob-line-code js-file-line"><span class="c">% #########################################################################################</span></td>
      </tr>
      <tr>
        <td id="L381" class="blob-line-num js-line-number" data-line-number="381"></td>
        <td id="LC381" class="blob-line-code js-file-line"><span class="c">% #########################################################################################</span></td>
      </tr>
      <tr>
        <td id="L382" class="blob-line-num js-line-number" data-line-number="382"></td>
        <td id="LC382" class="blob-line-code js-file-line"><span class="c">% #################################################### Figure 4 ###########################</span></td>
      </tr>
      <tr>
        <td id="L383" class="blob-line-num js-line-number" data-line-number="383"></td>
        <td id="LC383" class="blob-line-code js-file-line"><span class="c">% #########################################################################################</span></td>
      </tr>
      <tr>
        <td id="L384" class="blob-line-num js-line-number" data-line-number="384"></td>
        <td id="LC384" class="blob-line-code js-file-line"><span class="c">% #########################################################################################</span></td>
      </tr>
      <tr>
        <td id="L385" class="blob-line-num js-line-number" data-line-number="385"></td>
        <td id="LC385" class="blob-line-code js-file-line"><span class="c">%</span></td>
      </tr>
      <tr>
        <td id="L386" class="blob-line-num js-line-number" data-line-number="386"></td>
        <td id="LC386" class="blob-line-code js-file-line"><span class="k">\begin</span><span class="nb">{</span>figure<span class="nb">}</span></td>
      </tr>
      <tr>
        <td id="L387" class="blob-line-num js-line-number" data-line-number="387"></td>
        <td id="LC387" class="blob-line-code js-file-line">    <span class="k">\includegraphics</span><span class="na">[width=85mm]</span><span class="nb">{</span>Fig4<span class="nb">_</span>Movie<span class="nb">_</span>Screenshot.eps<span class="nb">}</span></td>
      </tr>
      <tr>
        <td id="L388" class="blob-line-num js-line-number" data-line-number="388"></td>
        <td id="LC388" class="blob-line-code js-file-line">    <span class="k">\caption</span><span class="nb">{</span><span class="k">\abb\label</span><span class="nb">{</span>fig:Fig4-Movie-Screenshot<span class="nb">}</span> Annotated screenshot from a movie.   A.				Artificially generated spike trains.   B. Dissimilarity matrices obtained by averaging 			over two separate time intervals for both the regular and the real-time SPIKE-distance 			as well as their averages over subgroups of spike trains (denoted by <span class="s">$</span><span class="nb">&lt;</span><span class="nv">\cdot</span><span class="nb">&gt;_G</span><span class="s">$</span>).   C. 				Corresponding dendrograms.<span class="nb">}</span></td>
      </tr>
      <tr>
        <td id="L389" class="blob-line-num js-line-number" data-line-number="389"></td>
        <td id="LC389" class="blob-line-code js-file-line"><span class="k">\end</span><span class="nb">{</span>figure<span class="nb">}</span></td>
      </tr>
      <tr>
        <td id="L390" class="blob-line-num js-line-number" data-line-number="390"></td>
        <td id="LC390" class="blob-line-code js-file-line"><span class="c">%</span></td>
      </tr>
      <tr>
        <td id="L391" class="blob-line-num js-line-number" data-line-number="391"></td>
        <td id="LC391" class="blob-line-code js-file-line"><span class="c">% #########################################################################################</span></td>
      </tr>
      <tr>
        <td id="L392" class="blob-line-num js-line-number" data-line-number="392"></td>
        <td id="LC392" class="blob-line-code js-file-line"><span class="c">% #########################################################################################</span></td>
      </tr>
      <tr>
        <td id="L393" class="blob-line-num js-line-number" data-line-number="393"></td>
        <td id="LC393" class="blob-line-code js-file-line"><span class="c">% #################################################### Figure 4 ###########################</span></td>
      </tr>
      <tr>
        <td id="L394" class="blob-line-num js-line-number" data-line-number="394"></td>
        <td id="LC394" class="blob-line-code js-file-line"><span class="c">% #########################################################################################</span></td>
      </tr>
      <tr>
        <td id="L395" class="blob-line-num js-line-number" data-line-number="395"></td>
        <td id="LC395" class="blob-line-code js-file-line"><span class="c">% #########################################################################################</span></td>
      </tr>
      <tr>
        <td id="L396" class="blob-line-num js-line-number" data-line-number="396"></td>
        <td id="LC396" class="blob-line-code js-file-line">
</td>
      </tr>
      <tr>
        <td id="L397" class="blob-line-num js-line-number" data-line-number="397"></td>
        <td id="LC397" class="blob-line-code js-file-line">After updating all of these data parameters the next step is to select the measures to be calculated. In addition to the measures described in Section <span class="k">\ref</span><span class="nb">{</span>s:Measures<span class="nb">}</span> we also include the Peri-Stimulus Time Histogram (PSTH) as a time-resolved standard approach widely used in spike train analysis. However, note that the PSTH is a complementary approach which describes overall firing rate and (to some precision) timing. It is not a measure of spike train synchrony <span class="k">\citep</span><span class="nb">{</span>Kreuz11<span class="nb">}</span> since it is invariant to shuffling spikes among the spike trains yielding the same value regardless of how spikes are distributed among the different spike trains. At the same time the user can select successive frames for a temporal analysis of spike train patterns. These can be individual time instants for cross sections, temporal intervals for selective averages and sequences of time instants for triggered averages (see Section <span class="k">\ref</span><span class="nb">{</span>ss:Information-reduction<span class="nb">}</span>).</td>
      </tr>
      <tr>
        <td id="L398" class="blob-line-num js-line-number" data-line-number="398"></td>
        <td id="LC398" class="blob-line-code js-file-line">
</td>
      </tr>
      <tr>
        <td id="L399" class="blob-line-num js-line-number" data-line-number="399"></td>
        <td id="LC399" class="blob-line-code js-file-line">Now the actual calculation of the measures takes place. For reasonably sized datasets this should take at most a few seconds. Larger datasets are divided in smaller subintervals and the calculation will be performed in a loop which might take longer (cf. Section <span class="k">\ref</span><span class="nb">{</span>ss:Limitations<span class="nb">}</span>). It is at this point that SPIKY becomes truly interactive. Now the user can switch between different representations of the results (such as dissimilarity profiles, dissimilarity matrices and dendrograms). Regarding matrices and dendrograms one can decide whether one wants to compare different representations or look at them in sequence.</td>
      </tr>
      <tr>
        <td id="L400" class="blob-line-num js-line-number" data-line-number="400"></td>
        <td id="LC400" class="blob-line-code js-file-line">
</td>
      </tr>
      <tr>
        <td id="L401" class="blob-line-num js-line-number" data-line-number="401"></td>
        <td id="LC401" class="blob-line-code js-file-line">The presentation can be restricted to smaller time windows and/or subsets of spike train, and temporal and spatial averaging (for example moving average and average over spike train groups) can be performed. It is possible to add further figure elements such as spike number histograms, overall averages, or dissimilarity profiles for individual spike train groups. At this stage the user can also retrospectively change the appearance of all the individual elements of the figure (see Section <span class="k">\ref</span><span class="nb">{</span>sss:Figure-Layout<span class="nb">}</span> for more details). Finally, SPIKY allows to extract both data and results to the Matlab workspace for further analysis, and it is also possible to save individual figures as postscript-file or a sequence of images as an `avi&#39;-movie (for more details see Section <span class="k">\ref</span><span class="nb">{</span>sss:Output<span class="nb">}</span>).</td>
      </tr>
      <tr>
        <td id="L402" class="blob-line-num js-line-number" data-line-number="402"></td>
        <td id="LC402" class="blob-line-code js-file-line">
</td>
      </tr>
      <tr>
        <td id="L403" class="blob-line-num js-line-number" data-line-number="403"></td>
        <td id="LC403" class="blob-line-code js-file-line">
</td>
      </tr>
      <tr>
        <td id="L404" class="blob-line-num js-line-number" data-line-number="404"></td>
        <td id="LC404" class="blob-line-code js-file-line"><span class="k">\subsubsection</span><span class="nb">{</span><span class="k">\label</span><span class="nb">{</span>sss:Input<span class="nb">}</span> Input<span class="nb">}</span></td>
      </tr>
      <tr>
        <td id="L405" class="blob-line-num js-line-number" data-line-number="405"></td>
        <td id="LC405" class="blob-line-code js-file-line">
</td>
      </tr>
      <tr>
        <td id="L406" class="blob-line-num js-line-number" data-line-number="406"></td>
        <td id="LC406" class="blob-line-code js-file-line">There are three different possibilities to input spike train data into SPIKY.</td>
      </tr>
      <tr>
        <td id="L407" class="blob-line-num js-line-number" data-line-number="407"></td>
        <td id="LC407" class="blob-line-code js-file-line">
</td>
      </tr>
      <tr>
        <td id="L408" class="blob-line-num js-line-number" data-line-number="408"></td>
        <td id="LC408" class="blob-line-code js-file-line">The first option is to select one of the predefined examples which are generated using Matlab-code. Initially these are the examples used in <span class="k">\citep</span><span class="nb">{</span>Kreuz13<span class="nb">}</span> but one can also define new examples.</td>
      </tr>
      <tr>
        <td id="L409" class="blob-line-num js-line-number" data-line-number="409"></td>
        <td id="LC409" class="blob-line-code js-file-line">
</td>
      </tr>
      <tr>
        <td id="L410" class="blob-line-num js-line-number" data-line-number="410"></td>
        <td id="LC410" class="blob-line-code js-file-line">The second option is to load spike train data from a file. Two different file formats are allowed, `.mat&#39; and `.txt&#39; (ASCII) files. For the mat-files SPIKY currently allows three different kinds of input formats (further formats can be added on demand).</td>
      </tr>
      <tr>
        <td id="L411" class="blob-line-num js-line-number" data-line-number="411"></td>
        <td id="LC411" class="blob-line-code js-file-line">
</td>
      </tr>
      <tr>
        <td id="L412" class="blob-line-num js-line-number" data-line-number="412"></td>
        <td id="LC412" class="blob-line-code js-file-line"><span class="k">\begin</span><span class="nb">{</span>itemize<span class="nb">}</span></td>
      </tr>
      <tr>
        <td id="L413" class="blob-line-num js-line-number" data-line-number="413"></td>
        <td id="LC413" class="blob-line-code js-file-line"><span class="k">\item</span> cell arrays (ca) with just the spike times. This is the preferred format used by SPIKY since it is most memory efficient. The two other formats will internally be converted into this format.</td>
      </tr>
      <tr>
        <td id="L414" class="blob-line-num js-line-number" data-line-number="414"></td>
        <td id="LC414" class="blob-line-code js-file-line"><span class="k">\item</span> regular matrices with each row being a spike train and zero padding (zp) in case the spike numbers are different.</td>
      </tr>
      <tr>
        <td id="L415" class="blob-line-num js-line-number" data-line-number="415"></td>
        <td id="LC415" class="blob-line-code js-file-line"><span class="k">\item</span> matrices representing time bins where each zero/one (01) indicates the absence/presence of a spike</td>
      </tr>
      <tr>
        <td id="L416" class="blob-line-num js-line-number" data-line-number="416"></td>
        <td id="LC416" class="blob-line-code js-file-line"><span class="k">\end</span><span class="nb">{</span>itemize<span class="nb">}</span></td>
      </tr>
      <tr>
        <td id="L417" class="blob-line-num js-line-number" data-line-number="417"></td>
        <td id="LC417" class="blob-line-code js-file-line">
</td>
      </tr>
      <tr>
        <td id="L418" class="blob-line-num js-line-number" data-line-number="418"></td>
        <td id="LC418" class="blob-line-code js-file-line">If case of a mat-file SPIKY looks for a variable called `spikes&#39;, if it cannot find it you have the chance to select the variable name (or field name) which contains the spikes via an input mask which provides a hierarchical structure tree of the variable structure. In the text format spike times should be written as a matrix with each row being one spike train. The SPIKY-package contains one example file for all four formats (`testdata<span class="k">\_</span>ca.mat&#39;, `testdata<span class="k">\_</span>zp.mat&#39;, `testdata<span class="k">\_</span>01.mat&#39; and `testdata.txt&#39;).</td>
      </tr>
      <tr>
        <td id="L419" class="blob-line-num js-line-number" data-line-number="419"></td>
        <td id="LC419" class="blob-line-code js-file-line">
</td>
      </tr>
      <tr>
        <td id="L420" class="blob-line-num js-line-number" data-line-number="420"></td>
        <td id="LC420" class="blob-line-code js-file-line">The third option is to create new spike train data via the spike train generator. After setting some defining variables (number of spike trains, start and end time, sampling interval) you can build your spike trains by using predefined spike train patterns (such as periodic, splay, uniform or Poisson) and/or by manually adding, shifting and deleting individual spikes or groups of spikes.</td>
      </tr>
      <tr>
        <td id="L421" class="blob-line-num js-line-number" data-line-number="421"></td>
        <td id="LC421" class="blob-line-code js-file-line">
</td>
      </tr>
      <tr>
        <td id="L422" class="blob-line-num js-line-number" data-line-number="422"></td>
        <td id="LC422" class="blob-line-code js-file-line">
</td>
      </tr>
      <tr>
        <td id="L423" class="blob-line-num js-line-number" data-line-number="423"></td>
        <td id="LC423" class="blob-line-code js-file-line"><span class="k">\subsubsection</span><span class="nb">{</span><span class="k">\label</span><span class="nb">{</span>sss:Figure-Layout<span class="nb">}</span> Figure-Layout<span class="nb">}</span></td>
      </tr>
      <tr>
        <td id="L424" class="blob-line-num js-line-number" data-line-number="424"></td>
        <td id="LC424" class="blob-line-code js-file-line">
</td>
      </tr>
      <tr>
        <td id="L425" class="blob-line-num js-line-number" data-line-number="425"></td>
        <td id="LC425" class="blob-line-code js-file-line">SPIKY was designed in a way that allows to directly generate figures suitable for publication. To this aim the user is given control over the appearance of every individual element (e.g. fonts, lines etc.) in each type of figure. There are two ways to determine essential properties such as color, font size or line width. Most conveniently, one can use the file `SPIKY<span class="k">\_</span>f<span class="k">\_</span>user<span class="k">\_</span>interface&#39; to define the standard values for all the parameters that describe the principal layout of the figure. But it is also possible to change elements in the active figure while the program is already running. To do so the user has to simply click the right mouse button on the element to be changed. A context menu will appear which lets the user either edit either the properties of individual elements or of all elements of a certain type. This also includes the string property of any font (title, x- and y-labels etc.).</td>
      </tr>
      <tr>
        <td id="L426" class="blob-line-num js-line-number" data-line-number="426"></td>
        <td id="LC426" class="blob-line-code js-file-line">
</td>
      </tr>
      <tr>
        <td id="L427" class="blob-line-num js-line-number" data-line-number="427"></td>
        <td id="LC427" class="blob-line-code js-file-line">If a figure contains more than one subplot (besides the combined subplot containing the spike rasterplot and dissimilarity profiles these are typically subplots with dissimilarity matrices and dendrograms), it is also possible to change their position and size. To do so just move the cursor to the respective axis (either just left or just below the subplot) and click the right mouse button. Now one can edit all position variables by hand or change the x-position, the y-position, the width and the height individually. In case there are several dissimilarity matrices / dendrograms one can do this either for an individual matrix / dendrogram or for all of them at the same time.</td>
      </tr>
      <tr>
        <td id="L428" class="blob-line-num js-line-number" data-line-number="428"></td>
        <td id="LC428" class="blob-line-code js-file-line">
</td>
      </tr>
      <tr>
        <td id="L429" class="blob-line-num js-line-number" data-line-number="429"></td>
        <td id="LC429" class="blob-line-code js-file-line">
</td>
      </tr>
      <tr>
        <td id="L430" class="blob-line-num js-line-number" data-line-number="430"></td>
        <td id="LC430" class="blob-line-code js-file-line"><span class="k">\subsubsection</span><span class="nb">{</span><span class="k">\label</span><span class="nb">{</span>sss:Output<span class="nb">}</span> Output<span class="nb">}</span></td>
      </tr>
      <tr>
        <td id="L431" class="blob-line-num js-line-number" data-line-number="431"></td>
        <td id="LC431" class="blob-line-code js-file-line">
</td>
      </tr>
      <tr>
        <td id="L432" class="blob-line-num js-line-number" data-line-number="432"></td>
        <td id="LC432" class="blob-line-code js-file-line">From within SPIKY it is possible to extract the spike trains and the results of the analyses (measure profiles, matrices, dendrograms) to the Matlab workspace for further processing. When one clicks the right mouse button on the element whose data one wishes to extract results will be stored in variables such as `SPIKY<span class="k">\_</span>spikes&#39;, `SPIKY<span class="k">\_</span>profile<span class="k">\_</span>X<span class="k">\_</span>1&#39;, `SPIKY<span class="k">\_</span>profile<span class="k">\_</span>Y<span class="k">\_</span>1&#39;, `SPIKY<span class="k">\_</span>profile<span class="k">\_</span>name<span class="k">\_</span>1&#39; as well as `SPIKY<span class="k">\_</span>matrix<span class="k">\_</span>1&#39; and `SPIKY<span class="k">\_</span>matrix<span class="k">\_</span>name<span class="k">\_</span>1&#39;. In addition, the results obtained during an analysis will automatically be stored in the output structure `SPIKY<span class="k">\_</span>results&#39; which will have one field for each measure selected. Depending on the parameter selection within SPIKY, for each measure the structure can contains the following subfields which largely correspond to the different representations identified in Section <span class="k">\ref</span><span class="nb">{</span>ss:Information-reduction<span class="nb">}</span>:</td>
      </tr>
      <tr>
        <td id="L433" class="blob-line-num js-line-number" data-line-number="433"></td>
        <td id="LC433" class="blob-line-code js-file-line">
</td>
      </tr>
      <tr>
        <td id="L434" class="blob-line-num js-line-number" data-line-number="434"></td>
        <td id="LC434" class="blob-line-code js-file-line"><span class="k">\begin</span><span class="nb">{</span>itemize<span class="nb">}</span></td>
      </tr>
      <tr>
        <td id="L435" class="blob-line-num js-line-number" data-line-number="435"></td>
        <td id="LC435" class="blob-line-code js-file-line"><span class="k">\item</span> SPIKY<span class="k">\_</span>results.<span class="s">$</span><span class="nv">\langle</span><span class="s">$</span>Measure<span class="s">$</span><span class="nv">\rangle</span><span class="s">$</span>.name: Name of selected measures (helps to identify the order within all other variables)</td>
      </tr>
      <tr>
        <td id="L436" class="blob-line-num js-line-number" data-line-number="436"></td>
        <td id="LC436" class="blob-line-code js-file-line"><span class="k">\item</span> SPIKY<span class="k">\_</span>results.<span class="s">$</span><span class="nv">\langle</span><span class="s">$</span>Measure<span class="s">$</span><span class="nv">\rangle</span><span class="s">$</span>.distance: Level of dissimilarity over all spike trains and the whole interval. This is just one value, obtained by averaging over both spike trains and time</td>
      </tr>
      <tr>
        <td id="L437" class="blob-line-num js-line-number" data-line-number="437"></td>
        <td id="LC437" class="blob-line-code js-file-line"><span class="k">\item</span> SPIKY<span class="k">\_</span>results.<span class="s">$</span><span class="nv">\langle</span><span class="s">$</span>Measure<span class="s">$</span><span class="nv">\rangle</span><span class="s">$</span>.matrix: Pairwise distance matrices, obtained by averaging over time</td>
      </tr>
      <tr>
        <td id="L438" class="blob-line-num js-line-number" data-line-number="438"></td>
        <td id="LC438" class="blob-line-code js-file-line"><span class="k">\item</span> SPIKY<span class="k">\_</span>results.<span class="s">$</span><span class="nv">\langle</span><span class="s">$</span>Measure<span class="s">$</span><span class="nv">\rangle</span><span class="s">$</span>.x: Time-values of overall dissimilarity profile</td>
      </tr>
      <tr>
        <td id="L439" class="blob-line-num js-line-number" data-line-number="439"></td>
        <td id="LC439" class="blob-line-code js-file-line"><span class="k">\item</span> SPIKY<span class="k">\_</span>results.<span class="s">$</span><span class="nv">\langle</span><span class="s">$</span>Measure<span class="s">$</span><span class="nv">\rangle</span><span class="s">$</span>.y: Overall dissimilarity profile obtained by averaging over spike train pairs</td>
      </tr>
      <tr>
        <td id="L440" class="blob-line-num js-line-number" data-line-number="440"></td>
        <td id="LC440" class="blob-line-code js-file-line"><span class="k">\end</span><span class="nb">{</span>itemize<span class="nb">}</span></td>
      </tr>
      <tr>
        <td id="L441" class="blob-line-num js-line-number" data-line-number="441"></td>
        <td id="LC441" class="blob-line-code js-file-line">
</td>
      </tr>
      <tr>
        <td id="L442" class="blob-line-num js-line-number" data-line-number="442"></td>
        <td id="LC442" class="blob-line-code js-file-line">Note that the dissimilarity profiles are not equidistantly sampled. Rather they are stored as memory-efficiently as possible which means just one value for each interval of the pooled spike train for the ISI- and two values for the SPIKE-distance. Since this format can be more difficult to process, the functions `SPIKY<span class="k">\_</span>f<span class="k">\_</span>selective<span class="k">\_</span>averaging&#39;, `SPIKY<span class="k">\_</span>f<span class="k">\_</span>triggered<span class="k">\_</span>averaging&#39;, and `SPIKY<span class="k">\_</span>f<span class="k">\_</span>average<span class="k">\_</span>pi&#39; are provided in order to compute the selective average over time intervals, the triggered over time instants, or the average over many dissimilarity profiles, respectively. Furthermore, for the ISI-distance the function `SPIKY<span class="k">\_</span>f<span class="k">\_</span>pico.m&#39; can be used to obtain the average value as well as the x- and y-vectors for plotting.</td>
      </tr>
      <tr>
        <td id="L443" class="blob-line-num js-line-number" data-line-number="443"></td>
        <td id="LC443" class="blob-line-code js-file-line">
</td>
      </tr>
      <tr>
        <td id="L444" class="blob-line-num js-line-number" data-line-number="444"></td>
        <td id="LC444" class="blob-line-code js-file-line">Besides the standard way to work with Matlab-figures SPIKY also offers the opportunity to save each figure as a postscript-file. Finally, it is possible to save a sequence of images as an `avi&#39;-movie.</td>
      </tr>
      <tr>
        <td id="L445" class="blob-line-num js-line-number" data-line-number="445"></td>
        <td id="LC445" class="blob-line-code js-file-line">
</td>
      </tr>
      <tr>
        <td id="L446" class="blob-line-num js-line-number" data-line-number="446"></td>
        <td id="LC446" class="blob-line-code js-file-line">
</td>
      </tr>
      <tr>
        <td id="L447" class="blob-line-num js-line-number" data-line-number="447"></td>
        <td id="LC447" class="blob-line-code js-file-line"><span class="k">\subsection</span><span class="nb">{</span><span class="k">\label</span><span class="nb">{</span>ss:GUI-vs-loop<span class="nb">}</span> GUI vs. loop<span class="nb">}</span></td>
      </tr>
      <tr>
        <td id="L448" class="blob-line-num js-line-number" data-line-number="448"></td>
        <td id="LC448" class="blob-line-code js-file-line">
</td>
      </tr>
      <tr>
        <td id="L449" class="blob-line-num js-line-number" data-line-number="449"></td>
        <td id="LC449" class="blob-line-code js-file-line">SPIKY was mainly designed to facilitate the detailed analysis of one dataset. It enables the user to switch between different representations (see Section <span class="k">\ref</span><span class="nb">{</span>ss:Information-reduction<span class="nb">}</span>) and to zoom in on both spatial and temporal features of interest. However, SPIKY is not very suitable for the collective analysis of many different datasets when e.g. the statistics of a certain quantity such as an average over certain time intervals should be evaluated over all available datasets in some kind of loop. For these purposes the SPIKY-package contains a program called `SPIKY<span class="k">\_</span>loop&#39; which is complementary to SPIKY. It is not a graphical user interface but it should be simple enough (and plenty of examples are provided) to allow everyone to run the same kind of analysis for many different datasets and to evaluate and compare their `SPIKY<span class="k">\_</span>results&#39;. `SPIKY<span class="k">\_</span>loop&#39; uses the full functionality of SPIKY such as access to time instants, selective and triggered averages as well as averages over spike train groups.</td>
      </tr>
      <tr>
        <td id="L450" class="blob-line-num js-line-number" data-line-number="450"></td>
        <td id="LC450" class="blob-line-code js-file-line">
</td>
      </tr>
      <tr>
        <td id="L451" class="blob-line-num js-line-number" data-line-number="451"></td>
        <td id="LC451" class="blob-line-code js-file-line">So combining these two programs it is possible to first use SPIKY for a rather exploratory but detailed analysis of a limited number of individual datasets and then use `SPIKY<span class="k">\_</span>loop&#39; and its output structure `SPIKY<span class="k">\_</span>loop<span class="k">\_</span>results&#39; to verify whether any effect discovered on the example dataset is consistently present within all of the datasets.</td>
      </tr>
      <tr>
        <td id="L452" class="blob-line-num js-line-number" data-line-number="452"></td>
        <td id="LC452" class="blob-line-code js-file-line">
</td>
      </tr>
      <tr>
        <td id="L453" class="blob-line-num js-line-number" data-line-number="453"></td>
        <td id="LC453" class="blob-line-code js-file-line">
</td>
      </tr>
      <tr>
        <td id="L454" class="blob-line-num js-line-number" data-line-number="454"></td>
        <td id="LC454" class="blob-line-code js-file-line"><span class="k">\subsection</span><span class="nb">{</span><span class="k">\label</span><span class="nb">{</span>ss:Spike-train-surrogates<span class="nb">}</span> Spike train surrogates and significance<span class="nb">}</span></td>
      </tr>
      <tr>
        <td id="L455" class="blob-line-num js-line-number" data-line-number="455"></td>
        <td id="LC455" class="blob-line-code js-file-line">
</td>
      </tr>
      <tr>
        <td id="L456" class="blob-line-num js-line-number" data-line-number="456"></td>
        <td id="LC456" class="blob-line-code js-file-line">An important question that has not yet been asked is the one of statistical significance. Given a certain value of the SPIKE-distance how can one judge whether it reflects a significant decrease or increase in spike train synchrony and does not just lie within the range of values obtained for random fluctuations. One way to address this question is the use of spike train surrogates <span class="k">\citep</span><span class="nb">{</span>Kass05, Gruen09, Louis10<span class="nb">}</span>. The idea is to compare the results obtained for the original dataset versus the results obtained for spike train surrogates generated from that dataset. If the value obtained for the original lies outside the range of values for the surrogates this value can be assumed to be significant to a level defined by the number of surrogates used (e.g. <span class="s">$</span><span class="nv">\alpha</span><span class="nb"> </span><span class="o">=</span><span class="nb"> </span><span class="m">0</span><span class="nb">.</span><span class="m">05</span><span class="s">$</span> for <span class="s">$</span><span class="m">19</span><span class="s">$</span> surrogates or <span class="s">$</span><span class="nv">\alpha</span><span class="nb"> </span><span class="o">=</span><span class="nb"> </span><span class="m">0</span><span class="nb">.</span><span class="m">001</span><span class="s">$</span> for <span class="s">$</span><span class="m">999</span><span class="s">$</span> surrogates).</td>
      </tr>
      <tr>
        <td id="L457" class="blob-line-num js-line-number" data-line-number="457"></td>
        <td id="LC457" class="blob-line-code js-file-line">
</td>
      </tr>
      <tr>
        <td id="L458" class="blob-line-num js-line-number" data-line-number="458"></td>
        <td id="LC458" class="blob-line-code js-file-line">The SPIKY-package contains a program `SPIKY<span class="k">\_</span>loop<span class="k">\_</span>surro&#39; which was designed to look at significance.</td>
      </tr>
      <tr>
        <td id="L459" class="blob-line-num js-line-number" data-line-number="459"></td>
        <td id="LC459" class="blob-line-code js-file-line">So far it includes four different types of spike train surrogates. They differ in the properties that are preserved and maintain either the individual spike numbers (obtained by shuffling the spikes), the individual interspike interval distribution (obtained by shuffling the interspike intervals), the pooled spike train (obtained by shuffling spikes among the spike trains) or the peri-stimulus time histogram (PSTH) (obtained by drawing from the probability density function (PDF) of the data).</td>
      </tr>
      <tr>
        <td id="L460" class="blob-line-num js-line-number" data-line-number="460"></td>
        <td id="LC460" class="blob-line-code js-file-line">
</td>
      </tr>
      <tr>
        <td id="L461" class="blob-line-num js-line-number" data-line-number="461"></td>
        <td id="LC461" class="blob-line-code js-file-line">
</td>
      </tr>
      <tr>
        <td id="L462" class="blob-line-num js-line-number" data-line-number="462"></td>
        <td id="LC462" class="blob-line-code js-file-line"><span class="c">% Daniel&#39;s Email 13.07.2011:</span></td>
      </tr>
      <tr>
        <td id="L463" class="blob-line-num js-line-number" data-line-number="463"></td>
        <td id="LC463" class="blob-line-code js-file-line"><span class="c">%</span></td>
      </tr>
      <tr>
        <td id="L464" class="blob-line-num js-line-number" data-line-number="464"></td>
        <td id="LC464" class="blob-line-code js-file-line"><span class="c">% Regarding the use of Poisson processes to have a benchmark it will have different implications depending on the structure of your data. If the rate is almost constant then comparing to processes with the same estimated average rate is the best option. If the rate is time-dependent you may want to compare with processes which are Poisson but mimic the same time-dependent profile of the rate. These are just alternatives that give you complementary information on the structure of the reliability.</span></td>
      </tr>
      <tr>
        <td id="L465" class="blob-line-num js-line-number" data-line-number="465"></td>
        <td id="LC465" class="blob-line-code js-file-line"><span class="c">% Surrogate data with the same time dependent rate can be obtained by randomly reassigning each spike to any of the spike trains.</span></td>
      </tr>
      <tr>
        <td id="L466" class="blob-line-num js-line-number" data-line-number="466"></td>
        <td id="LC466" class="blob-line-code js-file-line">
</td>
      </tr>
      <tr>
        <td id="L467" class="blob-line-num js-line-number" data-line-number="467"></td>
        <td id="LC467" class="blob-line-code js-file-line">
</td>
      </tr>
      <tr>
        <td id="L468" class="blob-line-num js-line-number" data-line-number="468"></td>
        <td id="LC468" class="blob-line-code js-file-line"><span class="k">\subsection</span><span class="nb">{</span><span class="k">\label</span><span class="nb">{</span>ss:Comparison<span class="nb">}</span> Comparison with other implementations<span class="nb">}</span></td>
      </tr>
      <tr>
        <td id="L469" class="blob-line-num js-line-number" data-line-number="469"></td>
        <td id="LC469" class="blob-line-code js-file-line">
</td>
      </tr>
      <tr>
        <td id="L470" class="blob-line-num js-line-number" data-line-number="470"></td>
        <td id="LC470" class="blob-line-code js-file-line"><span class="k">\href</span><span class="nb">{</span>https://github.com/modulus-metric/spike-train-metrics<span class="nb">}{</span>C++ implementations<span class="nb">}</span> (<span class="k">\url</span><span class="nb">{</span>https://github.com/modulus-metric/spike-train-metrics<span class="nb">}</span>) of several types of spike train metrics courtesy of R<span class="nb">{</span><span class="k">\v</span> a<span class="nb">}</span>zvan Florian <span class="k">\citep</span><span class="nb">{</span>Rusu14<span class="nb">}</span></td>
      </tr>
      <tr>
        <td id="L471" class="blob-line-num js-line-number" data-line-number="471"></td>
        <td id="LC471" class="blob-line-code js-file-line">
</td>
      </tr>
      <tr>
        <td id="L472" class="blob-line-num js-line-number" data-line-number="472"></td>
        <td id="LC472" class="blob-line-code js-file-line">
</td>
      </tr>
      <tr>
        <td id="L473" class="blob-line-num js-line-number" data-line-number="473"></td>
        <td id="LC473" class="blob-line-code js-file-line">XXX Nebojsa figure (Performance comparison) XXX</td>
      </tr>
      <tr>
        <td id="L474" class="blob-line-num js-line-number" data-line-number="474"></td>
        <td id="LC474" class="blob-line-code js-file-line">
</td>
      </tr>
      <tr>
        <td id="L475" class="blob-line-num js-line-number" data-line-number="475"></td>
        <td id="LC475" class="blob-line-code js-file-line">
</td>
      </tr>
      <tr>
        <td id="L476" class="blob-line-num js-line-number" data-line-number="476"></td>
        <td id="LC476" class="blob-line-code js-file-line"><span class="k">\href</span><span class="nb">{</span>http://jeremy.fix.free.fr/Softwares/spike.html<span class="nb">}{</span>Python-Implementation<span class="nb">}</span> (<span class="k">\url</span><span class="nb">{</span>http://jeremy.fix.free.fr/Softwares/spike.html<span class="nb">}</span>) of SPIKE-distance courtesy of Jeremy Fix</td>
      </tr>
      <tr>
        <td id="L477" class="blob-line-num js-line-number" data-line-number="477"></td>
        <td id="LC477" class="blob-line-code js-file-line">
</td>
      </tr>
      <tr>
        <td id="L478" class="blob-line-num js-line-number" data-line-number="478"></td>
        <td id="LC478" class="blob-line-code js-file-line"><span class="k">\href</span><span class="nb">{</span>https://pypi.python.org/pypi/ISIpy/1.0.1<span class="nb">}{</span>Python-Implementation<span class="nb">}</span> (<span class="k">\url</span><span class="nb">{</span>https://pypi.python.org/pypi/ISIpy/1.0.1<span class="nb">}</span>) of the pairwise ISI-distance courtesy of Michael Chary</td>
      </tr>
      <tr>
        <td id="L479" class="blob-line-num js-line-number" data-line-number="479"></td>
        <td id="LC479" class="blob-line-code js-file-line">
</td>
      </tr>
      <tr>
        <td id="L480" class="blob-line-num js-line-number" data-line-number="480"></td>
        <td id="LC480" class="blob-line-code js-file-line"><span class="k">\href</span><span class="nb">{</span>https://github.com/HRLAnalysis/HRLAnalysis<span class="nb">}{</span>mixed C++ and Python software suite<span class="nb">}</span> (<span class="k">\url</span><span class="nb">{</span>https://github.com/HRLAnalysis/HRLAnalysis<span class="nb">}</span>) for analyzing spiking neural data courtesy of Corey M. Thibeault and Michael J. O&#39;Brien <span class="k">\citep</span><span class="nb">{</span>Thibeault14<span class="nb">}</span></td>
      </tr>
      <tr>
        <td id="L481" class="blob-line-num js-line-number" data-line-number="481"></td>
        <td id="LC481" class="blob-line-code js-file-line">
</td>
      </tr>
      <tr>
        <td id="L482" class="blob-line-num js-line-number" data-line-number="482"></td>
        <td id="LC482" class="blob-line-code js-file-line">While all of these previous implementations are restricted to the overall dissimilarity profile and its temporal average (the overall dissimilarity), SPIKY also allows the user to interactively access all of the many different levels of information reduction introduced in Section <span class="k">\ref</span><span class="nb">{</span>ss:Information-reduction<span class="nb">}</span>.</td>
      </tr>
      <tr>
        <td id="L483" class="blob-line-num js-line-number" data-line-number="483"></td>
        <td id="LC483" class="blob-line-code js-file-line">
</td>
      </tr>
      <tr>
        <td id="L484" class="blob-line-num js-line-number" data-line-number="484"></td>
        <td id="LC484" class="blob-line-code js-file-line">
</td>
      </tr>
      <tr>
        <td id="L485" class="blob-line-num js-line-number" data-line-number="485"></td>
        <td id="LC485" class="blob-line-code js-file-line">
</td>
      </tr>
      <tr>
        <td id="L486" class="blob-line-num js-line-number" data-line-number="486"></td>
        <td id="LC486" class="blob-line-code js-file-line">
</td>
      </tr>
      <tr>
        <td id="L487" class="blob-line-num js-line-number" data-line-number="487"></td>
        <td id="LC487" class="blob-line-code js-file-line">
</td>
      </tr>
      <tr>
        <td id="L488" class="blob-line-num js-line-number" data-line-number="488"></td>
        <td id="LC488" class="blob-line-code js-file-line"><span class="c">%</span></td>
      </tr>
      <tr>
        <td id="L489" class="blob-line-num js-line-number" data-line-number="489"></td>
        <td id="LC489" class="blob-line-code js-file-line"><span class="c">%\begin{figure}</span></td>
      </tr>
      <tr>
        <td id="L490" class="blob-line-num js-line-number" data-line-number="490"></td>
        <td id="LC490" class="blob-line-code js-file-line"><span class="c">%    \includegraphics[width=85mm]{FigXX_Performance_Comparison.eps}</span></td>
      </tr>
      <tr>
        <td id="L491" class="blob-line-num js-line-number" data-line-number="491"></td>
        <td id="LC491" class="blob-line-code js-file-line"><span class="c">%    \caption{\abb\label{fig:FigXX-Performance-Comparison} XXX Add correct figure XXX Performance comparison: Time %needed for computing various distances, as a function of the number of spikes in the spike trains. For the ISI- and %the SPIKE-distance we show the curves for both the old (sampling-based) and the new algorithm.}</span></td>
      </tr>
      <tr>
        <td id="L492" class="blob-line-num js-line-number" data-line-number="492"></td>
        <td id="LC492" class="blob-line-code js-file-line"><span class="c">%\end{figure}</span></td>
      </tr>
      <tr>
        <td id="L493" class="blob-line-num js-line-number" data-line-number="493"></td>
        <td id="LC493" class="blob-line-code js-file-line">
</td>
      </tr>
      <tr>
        <td id="L494" class="blob-line-num js-line-number" data-line-number="494"></td>
        <td id="LC494" class="blob-line-code js-file-line"><span class="c">%</span></td>
      </tr>
      <tr>
        <td id="L495" class="blob-line-num js-line-number" data-line-number="495"></td>
        <td id="LC495" class="blob-line-code js-file-line"><span class="c">%</span></td>
      </tr>
      <tr>
        <td id="L496" class="blob-line-num js-line-number" data-line-number="496"></td>
        <td id="LC496" class="blob-line-code js-file-line"><span class="c">% *************************************************************************************</span></td>
      </tr>
      <tr>
        <td id="L497" class="blob-line-num js-line-number" data-line-number="497"></td>
        <td id="LC497" class="blob-line-code js-file-line"><span class="c">% ********************************************************** Discussion ***************</span></td>
      </tr>
      <tr>
        <td id="L498" class="blob-line-num js-line-number" data-line-number="498"></td>
        <td id="LC498" class="blob-line-code js-file-line"><span class="c">% *************************************************************************************</span></td>
      </tr>
      <tr>
        <td id="L499" class="blob-line-num js-line-number" data-line-number="499"></td>
        <td id="LC499" class="blob-line-code js-file-line"><span class="c">%</span></td>
      </tr>
      <tr>
        <td id="L500" class="blob-line-num js-line-number" data-line-number="500"></td>
        <td id="LC500" class="blob-line-code js-file-line"><span class="c">%</span></td>
      </tr>
      <tr>
        <td id="L501" class="blob-line-num js-line-number" data-line-number="501"></td>
        <td id="LC501" class="blob-line-code js-file-line"><span class="k">\section</span><span class="nb">{</span><span class="k">\label</span><span class="nb">{</span>s:Discussion<span class="nb">}</span> Discussion<span class="nb">}</span></td>
      </tr>
      <tr>
        <td id="L502" class="blob-line-num js-line-number" data-line-number="502"></td>
        <td id="LC502" class="blob-line-code js-file-line">
</td>
      </tr>
      <tr>
        <td id="L503" class="blob-line-num js-line-number" data-line-number="503"></td>
        <td id="LC503" class="blob-line-code js-file-line">So far the SPIKE-distance has been applied in the following papers:</td>
      </tr>
      <tr>
        <td id="L504" class="blob-line-num js-line-number" data-line-number="504"></td>
        <td id="LC504" class="blob-line-code js-file-line">
</td>
      </tr>
      <tr>
        <td id="L505" class="blob-line-num js-line-number" data-line-number="505"></td>
        <td id="LC505" class="blob-line-code js-file-line"><span class="k">\citep</span><span class="nb">{</span>Papoutsi13, DiPoppa13, Rusu14, Sacre14<span class="nb">}</span></td>
      </tr>
      <tr>
        <td id="L506" class="blob-line-num js-line-number" data-line-number="506"></td>
        <td id="LC506" class="blob-line-code js-file-line">
</td>
      </tr>
      <tr>
        <td id="L507" class="blob-line-num js-line-number" data-line-number="507"></td>
        <td id="LC507" class="blob-line-code js-file-line">XXXXX Update at the end XXXXX</td>
      </tr>
      <tr>
        <td id="L508" class="blob-line-num js-line-number" data-line-number="508"></td>
        <td id="LC508" class="blob-line-code js-file-line">
</td>
      </tr>
      <tr>
        <td id="L509" class="blob-line-num js-line-number" data-line-number="509"></td>
        <td id="LC509" class="blob-line-code js-file-line">Realtime algorithm for the SPIKE-distance</td>
      </tr>
      <tr>
        <td id="L510" class="blob-line-num js-line-number" data-line-number="510"></td>
        <td id="LC510" class="blob-line-code js-file-line">
</td>
      </tr>
      <tr>
        <td id="L511" class="blob-line-num js-line-number" data-line-number="511"></td>
        <td id="LC511" class="blob-line-code js-file-line"><span class="k">\subsection</span><span class="nb">{</span><span class="k">\label</span><span class="nb">{</span>ss:Summary<span class="nb">}</span> Summary<span class="nb">}</span></td>
      </tr>
      <tr>
        <td id="L512" class="blob-line-num js-line-number" data-line-number="512"></td>
        <td id="LC512" class="blob-line-code js-file-line">
</td>
      </tr>
      <tr>
        <td id="L513" class="blob-line-num js-line-number" data-line-number="513"></td>
        <td id="LC513" class="blob-line-code js-file-line"><span class="k">\subsection</span><span class="nb">{</span><span class="k">\label</span><span class="nb">{</span>ss:Limitations<span class="nb">}</span> Limitations<span class="nb">}</span></td>
      </tr>
      <tr>
        <td id="L514" class="blob-line-num js-line-number" data-line-number="514"></td>
        <td id="LC514" class="blob-line-code js-file-line">
</td>
      </tr>
      <tr>
        <td id="L515" class="blob-line-num js-line-number" data-line-number="515"></td>
        <td id="LC515" class="blob-line-code js-file-line">The calculation of the SPIKE-distance consists of three steps: First for each spike the distance to the nearest spike in all the other spike trains is calculated. Successively, for each time instant and each pair of spike trains, the distances of the four corner spikes are first locally weighted and then normalized. These latter steps involve matrices of the order `number of time instants&#39; <span class="s">$</span><span class="nv">\times</span><span class="s">$</span> `number of spike train pairs&#39;, which for very long datasets with many spike trains can lead to memory problems. The solution to this problem is to make the calculation sequential, i.e., to cut the recording interval into smaller segments, and to perform the averaging over all pairs of spike trains for each segment separately. In the end the dissimilarity profiles for the different segments (already averaged over pairs of spike trains) are concatenated, and its temporal average yields the distance value for the whole recording interval. If such a sequential calculation is required the user is continuously informed about the progress via a waitbar.</td>
      </tr>
      <tr>
        <td id="L516" class="blob-line-num js-line-number" data-line-number="516"></td>
        <td id="LC516" class="blob-line-code js-file-line">
</td>
      </tr>
      <tr>
        <td id="L517" class="blob-line-num js-line-number" data-line-number="517"></td>
        <td id="LC517" class="blob-line-code js-file-line">
</td>
      </tr>
      <tr>
        <td id="L518" class="blob-line-num js-line-number" data-line-number="518"></td>
        <td id="LC518" class="blob-line-code js-file-line">
</td>
      </tr>
      <tr>
        <td id="L519" class="blob-line-num js-line-number" data-line-number="519"></td>
        <td id="LC519" class="blob-line-code js-file-line"><span class="k">\subsection</span><span class="nb">{</span><span class="k">\label</span><span class="nb">{</span>ss:Outlook<span class="nb">}</span> Outlook<span class="nb">}</span></td>
      </tr>
      <tr>
        <td id="L520" class="blob-line-num js-line-number" data-line-number="520"></td>
        <td id="LC520" class="blob-line-code js-file-line">
</td>
      </tr>
      <tr>
        <td id="L521" class="blob-line-num js-line-number" data-line-number="521"></td>
        <td id="LC521" class="blob-line-code js-file-line">This manuscript is concerned with a new method and a new program primarily designed to analyze electrophysiological data such as neuronal spike trains. But in principle it is also applicable to any other kind of discrete data which comes in the form of sequences of time stamps (such as times of bouncing basketballs and coded parent and child acts during children&#39;s tantrums just to mention two examples which we already dealt with).</td>
      </tr>
      <tr>
        <td id="L522" class="blob-line-num js-line-number" data-line-number="522"></td>
        <td id="LC522" class="blob-line-code js-file-line">
</td>
      </tr>
      <tr>
        <td id="L523" class="blob-line-num js-line-number" data-line-number="523"></td>
        <td id="LC523" class="blob-line-code js-file-line">We chose to write SPIKY in Matlab because of its ease of use, popularity in the neuroscience community, and the well-developed MEX-interface for integrating C functions for performance enhancements. However, in the future we plan to export at least some of its functionality into an open-source environment which could for example be based on Python code within a C++ core. XXXXX Check advice on alternatives again XXXXX</td>
      </tr>
      <tr>
        <td id="L524" class="blob-line-num js-line-number" data-line-number="524"></td>
        <td id="LC524" class="blob-line-code js-file-line">
</td>
      </tr>
      <tr>
        <td id="L525" class="blob-line-num js-line-number" data-line-number="525"></td>
        <td id="LC525" class="blob-line-code js-file-line">Stand-Alone, </td>
      </tr>
      <tr>
        <td id="L526" class="blob-line-num js-line-number" data-line-number="526"></td>
        <td id="LC526" class="blob-line-code js-file-line">
</td>
      </tr>
      <tr>
        <td id="L527" class="blob-line-num js-line-number" data-line-number="527"></td>
        <td id="LC527" class="blob-line-code js-file-line"><span class="c">%\begin{appendix} \label{Appendix}</span></td>
      </tr>
      <tr>
        <td id="L528" class="blob-line-num js-line-number" data-line-number="528"></td>
        <td id="LC528" class="blob-line-code js-file-line"><span class="c">%</span></td>
      </tr>
      <tr>
        <td id="L529" class="blob-line-num js-line-number" data-line-number="529"></td>
        <td id="LC529" class="blob-line-code js-file-line"><span class="c">%\end{appendix}</span></td>
      </tr>
      <tr>
        <td id="L530" class="blob-line-num js-line-number" data-line-number="530"></td>
        <td id="LC530" class="blob-line-code js-file-line">
</td>
      </tr>
      <tr>
        <td id="L531" class="blob-line-num js-line-number" data-line-number="531"></td>
        <td id="LC531" class="blob-line-code js-file-line">Regarding the application to real neuronal data within any of the three different scenarios listed in the introduction,</td>
      </tr>
      <tr>
        <td id="L532" class="blob-line-num js-line-number" data-line-number="532"></td>
        <td id="LC532" class="blob-line-code js-file-line">
</td>
      </tr>
      <tr>
        <td id="L533" class="blob-line-num js-line-number" data-line-number="533"></td>
        <td id="LC533" class="blob-line-code js-file-line">The question whether synchrony on the single-neuron level is relevant for neuronal coding is not discussed here.</td>
      </tr>
      <tr>
        <td id="L534" class="blob-line-num js-line-number" data-line-number="534"></td>
        <td id="LC534" class="blob-line-code js-file-line">
</td>
      </tr>
      <tr>
        <td id="L535" class="blob-line-num js-line-number" data-line-number="535"></td>
        <td id="LC535" class="blob-line-code js-file-line">However, if there are changes in overall spike train synchrony or in spike train clustering SPIKY should be able to detect them.</td>
      </tr>
      <tr>
        <td id="L536" class="blob-line-num js-line-number" data-line-number="536"></td>
        <td id="LC536" class="blob-line-code js-file-line">
</td>
      </tr>
      <tr>
        <td id="L537" class="blob-line-num js-line-number" data-line-number="537"></td>
        <td id="LC537" class="blob-line-code js-file-line">
</td>
      </tr>
      <tr>
        <td id="L538" class="blob-line-num js-line-number" data-line-number="538"></td>
        <td id="LC538" class="blob-line-code js-file-line"><span class="k">\vspace</span><span class="nb">{</span>1cm<span class="nb">}</span></td>
      </tr>
      <tr>
        <td id="L539" class="blob-line-num js-line-number" data-line-number="539"></td>
        <td id="LC539" class="blob-line-code js-file-line">
</td>
      </tr>
      <tr>
        <td id="L540" class="blob-line-num js-line-number" data-line-number="540"></td>
        <td id="LC540" class="blob-line-code js-file-line"><span class="k">\begin</span><span class="nb">{</span>thanks<span class="nb">}</span></td>
      </tr>
      <tr>
        <td id="L541" class="blob-line-num js-line-number" data-line-number="541"></td>
        <td id="LC541" class="blob-line-code js-file-line"><span class="k">\section</span><span class="nb">{</span><span class="k">\label</span><span class="nb">{</span>s:Acknowledgement<span class="nb">}</span> <span class="k">\textbf</span><span class="nb">{</span>Acknowledgements<span class="nb">}}</span></td>
      </tr>
      <tr>
        <td id="L542" class="blob-line-num js-line-number" data-line-number="542"></td>
        <td id="LC542" class="blob-line-code js-file-line">
</td>
      </tr>
      <tr>
        <td id="L543" class="blob-line-num js-line-number" data-line-number="543"></td>
        <td id="LC543" class="blob-line-code js-file-line">NB and TK acknowledge funding support from the European Commission through the Marie Curie Initial Training Network `Neural Engineering Transformative Technologies (NETT)&#39;, project 289146. TK also acknowledges the Italian Ministry of Foreign Affairs regarding the activity of the Joint Italian-Israeli Laboratory on Neuroscience.</td>
      </tr>
      <tr>
        <td id="L544" class="blob-line-num js-line-number" data-line-number="544"></td>
        <td id="LC544" class="blob-line-code js-file-line">
</td>
      </tr>
      <tr>
        <td id="L545" class="blob-line-num js-line-number" data-line-number="545"></td>
        <td id="LC545" class="blob-line-code js-file-line">TK thanks Marcus Kaiser for hosting him at the University of Newcastle, UK.</td>
      </tr>
      <tr>
        <td id="L546" class="blob-line-num js-line-number" data-line-number="546"></td>
        <td id="LC546" class="blob-line-code js-file-line">     </td>
      </tr>
      <tr>
        <td id="L547" class="blob-line-num js-line-number" data-line-number="547"></td>
        <td id="LC547" class="blob-line-code js-file-line">We thank Ralph Andrzejak, Emily Caporello, Daniel Chicharro, Tim Gentner, Conor Houghton, Jutta Kretzberg, Stefano Luccioli, Florian Mormann, Leon Paz, Friederice Pirschel, Alessandro Torcini, Jonathan Victor, and Sid Visser for useful discussions.</td>
      </tr>
      <tr>
        <td id="L548" class="blob-line-num js-line-number" data-line-number="548"></td>
        <td id="LC548" class="blob-line-code js-file-line">
</td>
      </tr>
      <tr>
        <td id="L549" class="blob-line-num js-line-number" data-line-number="549"></td>
        <td id="LC549" class="blob-line-code js-file-line">We also thank Thomas Alderson, Black Square, Mayte Bonilla Quintana, Hamid Charkhkar, Didier Desaintjan, Mario DiPoppa, Mahboubeh Etemadi, Marion Najac, Matthew Phillips, Eugenio Piasini, Robert Rein, Rodrigo Salazar, Michael Schaub, Eitan Schechtman, Matthew Williams, Yunguo Yu for advice and user feedback.</td>
      </tr>
      <tr>
        <td id="L550" class="blob-line-num js-line-number" data-line-number="550"></td>
        <td id="LC550" class="blob-line-code js-file-line"><span class="k">\end</span><span class="nb">{</span>thanks<span class="nb">}</span></td>
      </tr>
      <tr>
        <td id="L551" class="blob-line-num js-line-number" data-line-number="551"></td>
        <td id="LC551" class="blob-line-code js-file-line">
</td>
      </tr>
      <tr>
        <td id="L552" class="blob-line-num js-line-number" data-line-number="552"></td>
        <td id="LC552" class="blob-line-code js-file-line"><span class="k">\begin</span><span class="nb">{</span>thebibliography<span class="nb">}{</span>99<span class="nb">}</span></td>
      </tr>
      <tr>
        <td id="L553" class="blob-line-num js-line-number" data-line-number="553"></td>
        <td id="LC553" class="blob-line-code js-file-line">
</td>
      </tr>
      <tr>
        <td id="L554" class="blob-line-num js-line-number" data-line-number="554"></td>
        <td id="LC554" class="blob-line-code js-file-line"><span class="k">\bibitem</span><span class="na">[{Bower et al.(2012)}]</span><span class="nb">{</span>Bower12<span class="nb">}</span></td>
      </tr>
      <tr>
        <td id="L555" class="blob-line-num js-line-number" data-line-number="555"></td>
        <td id="LC555" class="blob-line-code js-file-line">Bower MR, Stead M, Meyer FB, Marsh WR, Worrell GA. Spatiotemporal neuronal correlates of seizure generation in focal epilepsy. Epilepsia, 2012;53:807-16.</td>
      </tr>
      <tr>
        <td id="L556" class="blob-line-num js-line-number" data-line-number="556"></td>
        <td id="LC556" class="blob-line-code js-file-line">
</td>
      </tr>
      <tr>
        <td id="L557" class="blob-line-num js-line-number" data-line-number="557"></td>
        <td id="LC557" class="blob-line-code js-file-line"><span class="k">\bibitem</span><span class="na">[{{Di Poppa} and Gutkin(2013)}]</span><span class="nb">{</span>DiPoppa13<span class="nb">}</span></td>
      </tr>
      <tr>
        <td id="L558" class="blob-line-num js-line-number" data-line-number="558"></td>
        <td id="LC558" class="blob-line-code js-file-line"><span class="nb">{</span>DiPoppa<span class="nb">}</span> M, Gutkin BS. Correlations in background activity control persistent state stability and allow execution of working memory tasks. Front Comp Neurosci, 2013;7:139.</td>
      </tr>
      <tr>
        <td id="L559" class="blob-line-num js-line-number" data-line-number="559"></td>
        <td id="LC559" class="blob-line-code js-file-line">
</td>
      </tr>
      <tr>
        <td id="L560" class="blob-line-num js-line-number" data-line-number="560"></td>
        <td id="LC560" class="blob-line-code js-file-line"><span class="k">\bibitem</span><span class="na">[{Gruen(2009)}]</span><span class="nb">{</span>Gruen09<span class="nb">}</span></td>
      </tr>
      <tr>
        <td id="L561" class="blob-line-num js-line-number" data-line-number="561"></td>
        <td id="LC561" class="blob-line-code js-file-line">Gr<span class="nb">{</span><span class="k">\&quot;</span>u<span class="nb">}</span>n S. Data-Driven Significance Estimation for Precise Spike Correlation. J Neurophysiol, 2009;101:1126-40.</td>
      </tr>
      <tr>
        <td id="L562" class="blob-line-num js-line-number" data-line-number="562"></td>
        <td id="LC562" class="blob-line-code js-file-line">
</td>
      </tr>
      <tr>
        <td id="L563" class="blob-line-num js-line-number" data-line-number="563"></td>
        <td id="LC563" class="blob-line-code js-file-line"><span class="k">\bibitem</span><span class="na">[{Hochberg et al.(2006)}]</span><span class="nb">{</span>Hochberg06<span class="nb">}</span></td>
      </tr>
      <tr>
        <td id="L564" class="blob-line-num js-line-number" data-line-number="564"></td>
        <td id="LC564" class="blob-line-code js-file-line">Hochberg LR, Serruya MD, Friehs GM, Mukand JA, Saleh M, Caplan AH, Branner A, Chen D, Penn RD, Donoghue JP. Neuronal ensemble control of prosthetic devices by a human with tetraplegia. Nature, 2006;442:164-71.</td>
      </tr>
      <tr>
        <td id="L565" class="blob-line-num js-line-number" data-line-number="565"></td>
        <td id="LC565" class="blob-line-code js-file-line">
</td>
      </tr>
      <tr>
        <td id="L566" class="blob-line-num js-line-number" data-line-number="566"></td>
        <td id="LC566" class="blob-line-code js-file-line"><span class="k">\bibitem</span><span class="na">[{Kass et al.(2005)}]</span><span class="nb">{</span>Kass05<span class="nb">}</span></td>
      </tr>
      <tr>
        <td id="L567" class="blob-line-num js-line-number" data-line-number="567"></td>
        <td id="LC567" class="blob-line-code js-file-line">Kass RS, Ventura V, Brown EN. Statistical issues in the Analysis of Neuronal Data. J Neurophysiol, 2005;94:8-25.</td>
      </tr>
      <tr>
        <td id="L568" class="blob-line-num js-line-number" data-line-number="568"></td>
        <td id="LC568" class="blob-line-code js-file-line">
</td>
      </tr>
      <tr>
        <td id="L569" class="blob-line-num js-line-number" data-line-number="569"></td>
        <td id="LC569" class="blob-line-code js-file-line"><span class="k">\bibitem</span><span class="na">[{Kreuz et al.(2009)}]</span><span class="nb">{</span>Kreuz09<span class="nb">}</span></td>
      </tr>
      <tr>
        <td id="L570" class="blob-line-num js-line-number" data-line-number="570"></td>
        <td id="LC570" class="blob-line-code js-file-line">Kreuz T, Chicharro D, Andrzejak RG, Haas JS, Abarbanel HDI.</td>
      </tr>
      <tr>
        <td id="L571" class="blob-line-num js-line-number" data-line-number="571"></td>
        <td id="LC571" class="blob-line-code js-file-line">  Measuring multiple spike train synchrony. J Neurosci Methods, 2009;183:287-99.</td>
      </tr>
      <tr>
        <td id="L572" class="blob-line-num js-line-number" data-line-number="572"></td>
        <td id="LC572" class="blob-line-code js-file-line">
</td>
      </tr>
      <tr>
        <td id="L573" class="blob-line-num js-line-number" data-line-number="573"></td>
        <td id="LC573" class="blob-line-code js-file-line"><span class="k">\bibitem</span><span class="na">[{Kreuz et al.(2011)}]</span><span class="nb">{</span>Kreuz11<span class="nb">}</span></td>
      </tr>
      <tr>
        <td id="L574" class="blob-line-num js-line-number" data-line-number="574"></td>
        <td id="LC574" class="blob-line-code js-file-line">Kreuz T, Chicharro D, Greschner M, Andrzejak RG.</td>
      </tr>
      <tr>
        <td id="L575" class="blob-line-num js-line-number" data-line-number="575"></td>
        <td id="LC575" class="blob-line-code js-file-line">  Time-resolved and time-scale adaptive measures of spike train synchrony. J Neurosci Methods, 2011;195:92-106.</td>
      </tr>
      <tr>
        <td id="L576" class="blob-line-num js-line-number" data-line-number="576"></td>
        <td id="LC576" class="blob-line-code js-file-line">
</td>
      </tr>
      <tr>
        <td id="L577" class="blob-line-num js-line-number" data-line-number="577"></td>
        <td id="LC577" class="blob-line-code js-file-line"><span class="c">%\bibitem[{Kreuz(2011)}]{Kreuz11b}</span></td>
      </tr>
      <tr>
        <td id="L578" class="blob-line-num js-line-number" data-line-number="578"></td>
        <td id="LC578" class="blob-line-code js-file-line"><span class="c">%Kreuz T. Measures of spike train synchrony. Scholarpedia, 2011;6:11934.</span></td>
      </tr>
      <tr>
        <td id="L579" class="blob-line-num js-line-number" data-line-number="579"></td>
        <td id="LC579" class="blob-line-code js-file-line">
</td>
      </tr>
      <tr>
        <td id="L580" class="blob-line-num js-line-number" data-line-number="580"></td>
        <td id="LC580" class="blob-line-code js-file-line"><span class="k">\bibitem</span><span class="na">[{Kreuz et al.(2013)}]</span><span class="nb">{</span>Kreuz13<span class="nb">}</span></td>
      </tr>
      <tr>
        <td id="L581" class="blob-line-num js-line-number" data-line-number="581"></td>
        <td id="LC581" class="blob-line-code js-file-line">Kreuz T, Chicharro D, Houghton C, Andrzejak RG, Mormann F.</td>
      </tr>
      <tr>
        <td id="L582" class="blob-line-num js-line-number" data-line-number="582"></td>
        <td id="LC582" class="blob-line-code js-file-line">  Monitoring spike train synchrony. J Neurophysiol, 2013;109:1457-72.</td>
      </tr>
      <tr>
        <td id="L583" class="blob-line-num js-line-number" data-line-number="583"></td>
        <td id="LC583" class="blob-line-code js-file-line">
</td>
      </tr>
      <tr>
        <td id="L584" class="blob-line-num js-line-number" data-line-number="584"></td>
        <td id="LC584" class="blob-line-code js-file-line"><span class="k">\bibitem</span><span class="na">[{Kreuz et al.(2007)}]</span><span class="nb">{</span>Kreuz07c<span class="nb">}</span></td>
      </tr>
      <tr>
        <td id="L585" class="blob-line-num js-line-number" data-line-number="585"></td>
        <td id="LC585" class="blob-line-code js-file-line">Kreuz T, Haas JS, Morelli A, Abarbanel HDI, Politi A.</td>
      </tr>
      <tr>
        <td id="L586" class="blob-line-num js-line-number" data-line-number="586"></td>
        <td id="LC586" class="blob-line-code js-file-line">  Measuring spike train synchrony. J Neurosci Methods, 2007;165:151-61.</td>
      </tr>
      <tr>
        <td id="L587" class="blob-line-num js-line-number" data-line-number="587"></td>
        <td id="LC587" class="blob-line-code js-file-line">
</td>
      </tr>
      <tr>
        <td id="L588" class="blob-line-num js-line-number" data-line-number="588"></td>
        <td id="LC588" class="blob-line-code js-file-line"><span class="k">\bibitem</span><span class="na">[{Kumar et al.(2010)}]</span><span class="nb">{</span>Kumar10<span class="nb">}</span></td>
      </tr>
      <tr>
        <td id="L589" class="blob-line-num js-line-number" data-line-number="589"></td>
        <td id="LC589" class="blob-line-code js-file-line">Kumar A, Rotter S, Aertsen A. Spiking activity propagation in neuronal networks: reconciling different perspectives on neural coding. Nature Rev Neurosci, 2010;11:615-27.</td>
      </tr>
      <tr>
        <td id="L590" class="blob-line-num js-line-number" data-line-number="590"></td>
        <td id="LC590" class="blob-line-code js-file-line">
</td>
      </tr>
      <tr>
        <td id="L591" class="blob-line-num js-line-number" data-line-number="591"></td>
        <td id="LC591" class="blob-line-code js-file-line"><span class="k">\bibitem</span><span class="na">[{Louis et al.(2010)}]</span><span class="nb">{</span>Louis10<span class="nb">}</span></td>
      </tr>
      <tr>
        <td id="L592" class="blob-line-num js-line-number" data-line-number="592"></td>
        <td id="LC592" class="blob-line-code js-file-line">Louis S, Gerstein GL, Gr<span class="nb">{</span><span class="k">\&quot;</span>u<span class="nb">}</span>n S, Diesmann M. Surrogate spike train generation through dithering in operational time. Front Comp Neurosci, 2010;4:127.</td>
      </tr>
      <tr>
        <td id="L593" class="blob-line-num js-line-number" data-line-number="593"></td>
        <td id="LC593" class="blob-line-code js-file-line">
</td>
      </tr>
      <tr>
        <td id="L594" class="blob-line-num js-line-number" data-line-number="594"></td>
        <td id="LC594" class="blob-line-code js-file-line"><span class="k">\bibitem</span><span class="na">[{Mainen and Sejnowski(1995)}]</span><span class="nb">{</span>Mainen95<span class="nb">}</span></td>
      </tr>
      <tr>
        <td id="L595" class="blob-line-num js-line-number" data-line-number="595"></td>
        <td id="LC595" class="blob-line-code js-file-line">Mainen Z, Sejnowski TJ. Reliability of spike timing in neocortical neurons. Science 1995;268:1503-6.</td>
      </tr>
      <tr>
        <td id="L596" class="blob-line-num js-line-number" data-line-number="596"></td>
        <td id="LC596" class="blob-line-code js-file-line">
</td>
      </tr>
      <tr>
        <td id="L597" class="blob-line-num js-line-number" data-line-number="597"></td>
        <td id="LC597" class="blob-line-code js-file-line"><span class="k">\bibitem</span><span class="na">[{Miller and Wilson(2008)}]</span><span class="nb">{</span>Miller08<span class="nb">}</span></td>
      </tr>
      <tr>
        <td id="L598" class="blob-line-num js-line-number" data-line-number="598"></td>
        <td id="LC598" class="blob-line-code js-file-line">Miller EK, Wilson MA. All My Circuits: Using Multiple Electrodes to Understand Functioning Neural Networks. Neuron, 2008;60:483-8.</td>
      </tr>
      <tr>
        <td id="L599" class="blob-line-num js-line-number" data-line-number="599"></td>
        <td id="LC599" class="blob-line-code js-file-line">
</td>
      </tr>
      <tr>
        <td id="L600" class="blob-line-num js-line-number" data-line-number="600"></td>
        <td id="LC600" class="blob-line-code js-file-line"><span class="k">\bibitem</span><span class="na">[{Mormann et~al.(2007)}]</span><span class="nb">{</span>Mormann07<span class="nb">}</span></td>
      </tr>
      <tr>
        <td id="L601" class="blob-line-num js-line-number" data-line-number="601"></td>
        <td id="LC601" class="blob-line-code js-file-line">Mormann F, Andrzejak RG, Elger CE, Lehnertz K. Seizure prediction: the long and winding road. Brain, 2007;130:314-33.</td>
      </tr>
      <tr>
        <td id="L602" class="blob-line-num js-line-number" data-line-number="602"></td>
        <td id="LC602" class="blob-line-code js-file-line">
</td>
      </tr>
      <tr>
        <td id="L603" class="blob-line-num js-line-number" data-line-number="603"></td>
        <td id="LC603" class="blob-line-code js-file-line"><span class="k">\bibitem</span><span class="na">[{Nirenberg and Victor(2007)}]</span><span class="nb">{</span>Nirenberg07<span class="nb">}</span></td>
      </tr>
      <tr>
        <td id="L604" class="blob-line-num js-line-number" data-line-number="604"></td>
        <td id="LC604" class="blob-line-code js-file-line">Nirenberg S, Victor JD. Analyzing the activity of large populations of neurons: how tractable is the problem? Curr Opin Neurobiol 2007;17:397-400.</td>
      </tr>
      <tr>
        <td id="L605" class="blob-line-num js-line-number" data-line-number="605"></td>
        <td id="LC605" class="blob-line-code js-file-line">
</td>
      </tr>
      <tr>
        <td id="L606" class="blob-line-num js-line-number" data-line-number="606"></td>
        <td id="LC606" class="blob-line-code js-file-line"><span class="k">\bibitem</span><span class="na">[{Papoutsi et~al.(2013)}]</span><span class="nb">{</span>Papoutsi13<span class="nb">}</span></td>
      </tr>
      <tr>
        <td id="L607" class="blob-line-num js-line-number" data-line-number="607"></td>
        <td id="LC607" class="blob-line-code js-file-line">Papoutsi A, Sidiropoulou K, Cutsuridis V, Poirazi P. Induction and modulation of persistent activity in a layer v pfc microcircuit model. Front Neural Circuits 2013;7:161.</td>
      </tr>
      <tr>
        <td id="L608" class="blob-line-num js-line-number" data-line-number="608"></td>
        <td id="LC608" class="blob-line-code js-file-line">
</td>
      </tr>
      <tr>
        <td id="L609" class="blob-line-num js-line-number" data-line-number="609"></td>
        <td id="LC609" class="blob-line-code js-file-line"><span class="k">\bibitem</span><span class="na">[{Pikovsky et~al.(2001)}]</span><span class="nb">{</span>Pikovsky01<span class="nb">}</span></td>
      </tr>
      <tr>
        <td id="L610" class="blob-line-num js-line-number" data-line-number="610"></td>
        <td id="LC610" class="blob-line-code js-file-line">Pikovsky AS, Rosenblum MG, Kurths J. Synchronization. A universal concept in nonlinear sciences. Cambridge Univ. Press, Cambridge, UK 2001.</td>
      </tr>
      <tr>
        <td id="L611" class="blob-line-num js-line-number" data-line-number="611"></td>
        <td id="LC611" class="blob-line-code js-file-line">
</td>
      </tr>
      <tr>
        <td id="L612" class="blob-line-num js-line-number" data-line-number="612"></td>
        <td id="LC612" class="blob-line-code js-file-line"><span class="k">\bibitem</span><span class="na">[{Rusu and Florian(2014)}]</span><span class="nb">{</span>Rusu14<span class="nb">}</span></td>
      </tr>
      <tr>
        <td id="L613" class="blob-line-num js-line-number" data-line-number="613"></td>
        <td id="LC613" class="blob-line-code js-file-line">Rusu CV, RV Florian. A new class of metrics for spike trains. Neural Computation, 2014;26:306-48.</td>
      </tr>
      <tr>
        <td id="L614" class="blob-line-num js-line-number" data-line-number="614"></td>
        <td id="LC614" class="blob-line-code js-file-line">
</td>
      </tr>
      <tr>
        <td id="L615" class="blob-line-num js-line-number" data-line-number="615"></td>
        <td id="LC615" class="blob-line-code js-file-line"><span class="k">\bibitem</span><span class="na">[{Sacr\&#39;{e} and Sepulchre(2014)}]</span><span class="nb">{</span>Sacre14<span class="nb">}</span></td>
      </tr>
      <tr>
        <td id="L616" class="blob-line-num js-line-number" data-line-number="616"></td>
        <td id="LC616" class="blob-line-code js-file-line">Sacr<span class="k">\&#39;</span><span class="nb">{</span>e<span class="nb">}</span> P, Sepulchre R. Sensitivity analysis of oscillator models in the space of phase-response curves: Oscillators as open systems. Control Systems, IEEE 2014;34:50–74.</td>
      </tr>
      <tr>
        <td id="L617" class="blob-line-num js-line-number" data-line-number="617"></td>
        <td id="LC617" class="blob-line-code js-file-line">
</td>
      </tr>
      <tr>
        <td id="L618" class="blob-line-num js-line-number" data-line-number="618"></td>
        <td id="LC618" class="blob-line-code js-file-line"><span class="k">\bibitem</span><span class="na">[{Sanchez(2008)}]</span><span class="nb">{</span>Sanchez08<span class="nb">}</span></td>
      </tr>
      <tr>
        <td id="L619" class="blob-line-num js-line-number" data-line-number="619"></td>
        <td id="LC619" class="blob-line-code js-file-line">Sanchez JC, Principe JC, Nishida T, Bashirullah R, Harris JG, Fortes JAB. Technology and Signal Processing for Brain-Machine Interfaces. IEEE Signal Processing, 2008;25:29-40.</td>
      </tr>
      <tr>
        <td id="L620" class="blob-line-num js-line-number" data-line-number="620"></td>
        <td id="LC620" class="blob-line-code js-file-line">
</td>
      </tr>
      <tr>
        <td id="L621" class="blob-line-num js-line-number" data-line-number="621"></td>
        <td id="LC621" class="blob-line-code js-file-line"><span class="k">\bibitem</span><span class="na">[{Shlens et al.(2008)}]</span><span class="nb">{</span>Shlens08<span class="nb">}</span></td>
      </tr>
      <tr>
        <td id="L622" class="blob-line-num js-line-number" data-line-number="622"></td>
        <td id="LC622" class="blob-line-code js-file-line">Shlens J, Rieke F, Chichilnisky EJ. Synchronized firing in the retina. Curr Opin Neurobiol, 2008;18:396-402.</td>
      </tr>
      <tr>
        <td id="L623" class="blob-line-num js-line-number" data-line-number="623"></td>
        <td id="LC623" class="blob-line-code js-file-line">
</td>
      </tr>
      <tr>
        <td id="L624" class="blob-line-num js-line-number" data-line-number="624"></td>
        <td id="LC624" class="blob-line-code js-file-line"><span class="k">\bibitem</span><span class="na">[{Thibeault et al.(2014)}]</span><span class="nb">{</span>Thibeault14<span class="nb">}</span></td>
      </tr>
      <tr>
        <td id="L625" class="blob-line-num js-line-number" data-line-number="625"></td>
        <td id="LC625" class="blob-line-code js-file-line">Thibeault CM, <span class="nb">{</span>O&#39;Brien<span class="nb">}</span> MJ, Srinivasa N. Analyzing large-scale spiking neural data with HRLAnalysis. Front. Neuroinform., 2014;8:17.</td>
      </tr>
      <tr>
        <td id="L626" class="blob-line-num js-line-number" data-line-number="626"></td>
        <td id="LC626" class="blob-line-code js-file-line">
</td>
      </tr>
      <tr>
        <td id="L627" class="blob-line-num js-line-number" data-line-number="627"></td>
        <td id="LC627" class="blob-line-code js-file-line"><span class="k">\bibitem</span><span class="na">[{Tiesinga et al.(2008)}]</span><span class="nb">{</span>Tiesinga08<span class="nb">}</span></td>
      </tr>
      <tr>
        <td id="L628" class="blob-line-num js-line-number" data-line-number="628"></td>
        <td id="LC628" class="blob-line-code js-file-line">Tiesinga PHE, Fellous JM, Sejnowski TJ. Regulation of spike timing in visual cortical circuits. Nature Reviews Neuroscience 2008;9:97-107.</td>
      </tr>
      <tr>
        <td id="L629" class="blob-line-num js-line-number" data-line-number="629"></td>
        <td id="LC629" class="blob-line-code js-file-line">
</td>
      </tr>
      <tr>
        <td id="L630" class="blob-line-num js-line-number" data-line-number="630"></td>
        <td id="LC630" class="blob-line-code js-file-line"><span class="k">\bibitem</span><span class="na">[{Truccolo et al.(2011)}]</span><span class="nb">{</span>Truccolo11<span class="nb">}</span></td>
      </tr>
      <tr>
        <td id="L631" class="blob-line-num js-line-number" data-line-number="631"></td>
        <td id="LC631" class="blob-line-code js-file-line">Truccolo W, Donoghue JP, Hochberg LR, Eskandar EN, Madsen JR, Anderson WS, Brown EN, Halgren E, Cash SS. Single-neuron dynamics in human focal epilepsy. Nature Neurosci, 2011;14:635-41.</td>
      </tr>
      <tr>
        <td id="L632" class="blob-line-num js-line-number" data-line-number="632"></td>
        <td id="LC632" class="blob-line-code js-file-line">
</td>
      </tr>
      <tr>
        <td id="L633" class="blob-line-num js-line-number" data-line-number="633"></td>
        <td id="LC633" class="blob-line-code js-file-line"><span class="k">\bibitem</span><span class="na">[{Victor(2005)}]</span><span class="nb">{</span>Victor05<span class="nb">}</span></td>
      </tr>
      <tr>
        <td id="L634" class="blob-line-num js-line-number" data-line-number="634"></td>
        <td id="LC634" class="blob-line-code js-file-line">Victor JD. Spike train metrics. Current Opinion in Neurobiology 2005;15:585-592.</td>
      </tr>
      <tr>
        <td id="L635" class="blob-line-num js-line-number" data-line-number="635"></td>
        <td id="LC635" class="blob-line-code js-file-line">
</td>
      </tr>
      <tr>
        <td id="L636" class="blob-line-num js-line-number" data-line-number="636"></td>
        <td id="LC636" class="blob-line-code js-file-line"><span class="k">\end</span><span class="nb">{</span>thebibliography<span class="nb">}</span></td>
      </tr>
      <tr>
        <td id="L637" class="blob-line-num js-line-number" data-line-number="637"></td>
        <td id="LC637" class="blob-line-code js-file-line">
</td>
      </tr>
      <tr>
        <td id="L638" class="blob-line-num js-line-number" data-line-number="638"></td>
        <td id="LC638" class="blob-line-code js-file-line"><span class="k">\bibliographystyle</span><span class="nb">{</span>elsart-harv<span class="nb">}</span></td>
      </tr>
      <tr>
        <td id="L639" class="blob-line-num js-line-number" data-line-number="639"></td>
        <td id="LC639" class="blob-line-code js-file-line">
</td>
      </tr>
      <tr>
        <td id="L640" class="blob-line-num js-line-number" data-line-number="640"></td>
        <td id="LC640" class="blob-line-code js-file-line"><span class="k">\end</span><span class="nb">{</span>document<span class="nb">}</span></td>
      </tr>
</table>

  </div>

  </div>
</div>

<a href="#jump-to-line" rel="facebox[.linejump]" data-hotkey="l" style="display:none">Jump to Line</a>
<div id="jump-to-line" style="display:none">
  <form accept-charset="UTF-8" class="js-jump-to-line-form">
    <input class="linejump-input js-jump-to-line-field" type="text" placeholder="Jump to line&hellip;" autofocus>
    <button type="submit" class="button">Go</button>
  </form>
</div>

        </div>

      </div><!-- /.repo-container -->
      <div class="modal-backdrop"></div>
    </div><!-- /.container -->
  </div><!-- /.site -->


    </div><!-- /.wrapper -->

      <div class="container">
  <div class="site-footer">
    <ul class="site-footer-links right">
      <li><a href="https://status.github.com/">Status</a></li>
      <li><a href="http://developer.github.com">API</a></li>
      <li><a href="http://training.github.com">Training</a></li>
      <li><a href="http://shop.github.com">Shop</a></li>
      <li><a href="/blog">Blog</a></li>
      <li><a href="/about">About</a></li>

    </ul>

    <a href="/" aria-label="Homepage">
      <span class="mega-octicon octicon-mark-github" title="GitHub"></span>
    </a>

    <ul class="site-footer-links">
      <li>&copy; 2014 <span title="0.03856s from github-fe123-cp1-prd.iad.github.net">GitHub</span>, Inc.</li>
        <li><a href="/site/terms">Terms</a></li>
        <li><a href="/site/privacy">Privacy</a></li>
        <li><a href="/security">Security</a></li>
        <li><a href="/contact">Contact</a></li>
    </ul>
  </div><!-- /.site-footer -->
</div><!-- /.container -->


    <div class="fullscreen-overlay js-fullscreen-overlay" id="fullscreen_overlay">
  <div class="fullscreen-container js-suggester-container">
    <div class="textarea-wrap">
      <textarea name="fullscreen-contents" id="fullscreen-contents" class="fullscreen-contents js-fullscreen-contents js-suggester-field" placeholder=""></textarea>
    </div>
  </div>
  <div class="fullscreen-sidebar">
    <a href="#" class="exit-fullscreen js-exit-fullscreen tooltipped tooltipped-w" aria-label="Exit Zen Mode">
      <span class="mega-octicon octicon-screen-normal"></span>
    </a>
    <a href="#" class="theme-switcher js-theme-switcher tooltipped tooltipped-w"
      aria-label="Switch themes">
      <span class="octicon octicon-color-mode"></span>
    </a>
  </div>
</div>



    <div id="ajax-error-message" class="flash flash-error">
      <span class="octicon octicon-alert"></span>
      <a href="#" class="octicon octicon-x close js-ajax-error-dismiss" aria-label="Dismiss error"></a>
      Something went wrong with that request. Please try again.
    </div>


      <script crossorigin="anonymous" src="https://assets-cdn.github.com/assets/frameworks-2b4202fc62ef88e1a007a9ed05df10607b189f42.js" type="text/javascript"></script>
      <script async="async" crossorigin="anonymous" src="https://assets-cdn.github.com/assets/github-19ba694dbe2930d7f4bf2f69535a9a7fd3a4be48.js" type="text/javascript"></script>
      
      
        <script async src="https://www.google-analytics.com/analytics.js"></script>
  </body>
</html>

